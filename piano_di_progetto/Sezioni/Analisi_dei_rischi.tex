\section{Analisi dei rischi}
Nel corso dello sviluppo del progetto è naturale incontrare vari tipi di problematiche, che con un’attenta e continua analisi dei rischi possono essere mitigate.
Il piano per la gestione dei rischi viene suddiviso in 4 attività:

\begin{itemize}
\item \textbf{Individuazione} dei possibili eventi che possono portare a dei problemi durante l’avanzamento;

\item \textbf{Analisi} del problema, in particolare la probabilità con cui si possa verificare e le conseguenze negative che comporta;

\item \textbf{Pianificazione} di misure da prendere per impedire il verificarsi dei rischi e comportamenti da seguire nel caso in cui essi dovessero presentarsi. In questo modo si evita che un rischio possa diventare insostenibile;

\item \textbf{Monitoraggio} continuo dei rischi, cercando di prevenirli o minimizzando l’effetto negativo di quest’ultimi.
\end{itemize}

\subsection{Rischi tecnologici}

\begin{table}[H]
\rowcolors{2}{gray!25}{white}
\renewcommand{\arraystretch}{1.5}
\begin{tabular}{m{0.3\textwidth}<\centering m{0.65\textwidth}<\centering}
\rowcolor{darkblue} \multicolumn{2}{c}{\textcolor{white}{\textbf{Problemi hardware}}} \\
\hline
\textbf{Descrizione} & Ogni membro dispone di un computer da cui lavorare, il quale può essere soggetto a guasti \\
\textbf{Conseguenze} &  Possibili ritardi nell'avanzamento del progetto \\
\textbf{Probabilità di manifestarsi} & Bassa\\
\textbf{Pericolosità} & Media \\
\textbf{Precauzioni} & Ogni tipo di modifica a file riguardanti il progetto viene sottoposta a backup tramite sistema di versionamento\glo{} remoto \\
\textbf{Contingenza} & L'azienda si offre di ospitare i componenti del gruppo offrendo delle postazioni di lavoro. Inoltre, l’ateneo mette a disposizione laboratori, utilizzabili in queste situazioni. \\
\end{tabular}
\end{table}


\begin{table}[H]
\rowcolors{2}{gray!25}{white}
\renewcommand{\arraystretch}{1.5}
\begin{tabular}{m{0.3\textwidth}<\centering m{0.65\textwidth}<\centering}
\rowcolor{darkblue} \multicolumn{2}{c}{\textcolor{white}{\textbf{Problemi software}}}\\
\hline
\textbf{Descrizione} & Il gruppo fa uso di software di terze parti, che può essere soggetto a malfunzionamenti e bug \\
\textbf{Conseguenze} &  Possibile inconsistenza dei dati e ritardi nello sviluppo \\
\textbf{Probabilità di manifestarsi} & Bassa\\
\textbf{Pericolosità} & Alta \\
\textbf{Precauzioni} & Per evitare l’inconsistenza, il responsabile di progetto si incarica di effettuare il backup dei dati \\
\textbf{Contingenza} & Il responsabile, in accordo con l’azienda, deciderà una tecnologia simile\\
\end{tabular}
\end{table}

\begin{table}[H]
\rowcolors{2}{gray!25}{white}
\renewcommand{\arraystretch}{1.5}
\begin{tabular}{m{0.3\textwidth}<\centering m{0.65\textwidth}<\centering}
\rowcolor{darkblue} \multicolumn{2}{c}{\textcolor{white}{\textbf{Modifica della piattaforma\glo{} Instagram}}}\\
\hline
\textbf{Descrizione} & Il gruppo estrapola dati dalla piattaforma Instagram che potrebbe essere soggetta a modifiche periodiche \\
\textbf{Conseguenze} &  Malfunzionamenti del software di crawling\glo{} \\
\textbf{Probabilità di manifestarsi} & Bassa \\
\textbf{Pericolosità} & Alta \\
\textbf{Precauzioni} & Per evitare il verificarsi di tale problema il responsabile, periodicamente, realizza un video dimostrativo del buon funzionamento del prodotto software realizzato\\
\textbf{Contingenza} & Il responsabile, in accordo con l’azienda, deciderà una tecnologia simile\\
\end{tabular}
\end{table}

\begin{table}[H]
\rowcolors{2}{gray!25}{white}
\renewcommand{\arraystretch}{1.5}
\begin{tabular}{m{0.3\textwidth}<\centering m{0.65\textwidth}<\centering}
\rowcolor{darkblue} \multicolumn{2}{c}{\textcolor{white}{\textbf{Modifica della piattaforma TikTok}}}\\
\hline
\textbf{Descrizione} & Il gruppo estrapola dati dalla piattaforma TikTok, che potrebbe essere soggetta a modifiche periodiche \\
\textbf{Conseguenze} &  Malfunzionamenti del software di crawling\glo{} \\
\textbf{Probabilità di manifestarsi} & Alta \\
\textbf{Pericolosità} & Alta \\
\textbf{Precauzioni} & Per evitare il verificarsi di tale problema il responsabile periodicamente realizza un video dimostrativo del buon funzionamento del prodotto software realizzato\\
\textbf{Contingenza} & Il responsabile, in accordo con l’azienda, deciderà una tecnologia simile\\
\end{tabular}
\end{table}

\subsection{Rischi personali}

\begin{table}[H]
\rowcolors{2}{gray!25}{white}
\renewcommand{\arraystretch}{1.5}
\begin{tabular}{m{0.3\textwidth}<\centering m{0.65\textwidth}<\centering}
\rowcolor{darkblue} \multicolumn{2}{c}{\textcolor{white}{\textbf{Conflitti decisionali}}}\\
\hline
\textbf{Descrizione} & I membri del gruppo possono essere in disaccordo sulle tecnologie da utilizzare laddove l’azienda da libera scelta\\
\textbf{Conseguenze} & Malessere all’interno del gruppo  \\
\textbf{Probabilità di manifestarsi} & Bassa\\
\textbf{Pericolosità} & Media \\
\textbf{Precauzioni} & Il componente del gruppo comunicherà la sua disapprovazione al responsabile di progetto \\
\textbf{Contingenza} & Scelta della tecnologie tramite un'opportuna indagine tra i componenti del gruppo\\
\end{tabular}
\end{table}

\begin{table}[H]
\rowcolors{2}{gray!25}{white}
\renewcommand{\arraystretch}{1.5}
\begin{tabular}{m{0.3\textwidth}<\centering m{0.65\textwidth}<\centering}
\rowcolor{darkblue} \multicolumn{2}{c}{\textcolor{white}{\textbf{Inesperienza Tecnologica}}}\\
\hline
\textbf{Descrizione} & I membri del gruppo non hanno esperienza con le varie tecnologie scelte\\
\textbf{Conseguenze} & Ogni membro del gruppo può avere delle tempistiche di apprendimento differenti  \\
\textbf{Probabilità di manifestarsi} & Alta\\
\textbf{Pericolosità} & Alta \\
\textbf{Precauzioni} & Il componente del gruppo che si identifica in questa situazione comunicherà tempestivamente il suo stato al gruppo \\
\textbf{Contingenza} & I membri del gruppo che hanno già appreso al meglio la tecnologia forniranno supporto per aiutare lo sviluppo\\
\end{tabular}
\end{table}


\begin{table}[H]
\rowcolors{2}{gray!25}{white}
\renewcommand{\arraystretch}{1.5}
\begin{tabular}{m{0.3\textwidth}<\centering m{0.65\textwidth}<\centering}
\rowcolor{darkblue} \multicolumn{2}{c}{\textcolor{white}{\textbf{Disponibilità dei membri}}}\\
\hline
\textbf{Descrizione} & I membri del gruppo hanno impegni extra-universitari i quali possono renderli indisponibili nelle varie fasi del progetto\\
\textbf{Conseguenze} & Possibile ritardo sull'avanzamento individuale di gruppo \\
\textbf{Probabilità di manifestarsi} & Media \\
\textbf{Pericolosità} & Media \\
\textbf{Precauzioni} & Ogni membro del gruppo è tenuto a comunicare tempestivamente la proprio indisponibilità in modo da garantire un’organizzazione ottimale \\
\textbf{Contingenza} & In caso di assenze prolungate, il responsabile di progetto provvederà a ridistribuire i compiti \\
\end{tabular}
\end{table}


\begin{table}[H]
\rowcolors{2}{gray!25}{white}
\renewcommand{\arraystretch}{1.5}
\begin{tabular}{m{0.3\textwidth}<\centering m{0.65\textwidth}<\centering}
\rowcolor{darkblue} \multicolumn{2}{c}{\textcolor{white}{\textbf{Difficoltà di comunicazione}}}\\
\hline
\textbf{Descrizione} & La maggior parte degli incontri sarà svolta in via telematica. Il gruppo quindi può avere difficoltà nel comunicare sia internamente, sia con il proponente esterno\\
\textbf{Conseguenze} & Possibile ritardo sull'avanzamento individuale e di gruppo \\
\textbf{Probabilità di manifestarsi} & Media \\
\textbf{Pericolosità} & Media \\
\textbf{Precauzioni} & Per la comunicazione esterna ed interna, il gruppo si avvale di più strumenti di supporto \\
\textbf{Contingenza} & Il responsabile di progetto, in accordo con il proponente esterno, si occuperà di comunicare lo strumento designato a svolgere il compito per ogni incontro stabilito \\
\end{tabular}
\end{table}

\subsection{Rischi organizzativi}


\begin{table}[H]
\rowcolors{2}{gray!25}{white}
\renewcommand{\arraystretch}{1.5}
\begin{tabular}{m{0.3\textwidth}<\centering m{0.65\textwidth}<\centering}
\rowcolor{darkblue} \multicolumn{2}{c}{\textcolor{white}{\textbf{Calcolo delle tempistiche}}}\\
\hline
\textbf{Descrizione} & I membri del team, a causa di inesperienza o impegni personali, possono non essere in grado di rispettare le milestones\glo{} prefissate\\
\textbf{Conseguenze} & Possibile ritardo sull'avanzamento individuale e di gruppo \\
\textbf{Probabilità di manifestarsi} & Media \\
\textbf{Pericolosità} & Alta \\
\textbf{Precauzioni} & Nel momento in cui verranno decise le milestones\glo{}, ogni singolo membro del team dovrà portare alla luce eventuali difficoltà, che verranno prese in considerazione nella decisione delle scadenze \\
\textbf{Contingenza} & In caso non si riesca in nessun modo a rispettare le scadenze, il responsabile di progetto dovrà riassegnare le risorse ed i compiti. Inoltre, ogni membro del gruppo deve operare in modo che questa problematica non si verifichi \\
\end{tabular}
\end{table}


\begin{table}[H]
\rowcolors{2}{gray!25}{white}
\renewcommand{\arraystretch}{1.5}
\begin{tabular}{m{0.3\textwidth}<\centering m{0.65\textwidth}<\centering}
\rowcolor{darkblue} \multicolumn{2}{c}{\textcolor{white}{\textbf{Calcolo dei costi}}}\\
\hline
\textbf{Descrizione} & La stima dei costi preventivata potrebbe non essere corretta a causa dell’inesperienza del team\\
\textbf{Conseguenze} & In caso di sovrastima si avrebbe del tempo non necessario a disposizione, nel caso di sottostima si avrebbero dei ritardi nella consegna finale rispetto a quanto preventivato \\
\textbf{Probabilità di manifestarsi} & Media \\
\textbf{Pericolosità} & Alta \\
\textbf{Precauzioni} & Ogni membro del gruppo dovrà attenersi il più possibile alla pianificazione fatta dal gruppo \\
\textbf{Contingenza} & In caso di sovrastima il gruppo potrà dedicarsi a tutte quelle attività che sono considerate opzionali e ad una verifica di durata più lunga. In caso di sottostima, il responsabile di progetto dovrà riassegnare le risorse e i ruoli in modo da rispettare il più possibile i costi preventivati \\
\end{tabular}
\end{table}

\subsection{Rischi legati ai requisiti}


\begin{table}[H]
\rowcolors{2}{gray!25}{white}
\renewcommand{\arraystretch}{1.5}
\begin{tabular}{m{0.3\textwidth}<\centering m{0.65\textwidth}<\centering}
\rowcolor{darkblue} \multicolumn{2}{c}{\textcolor{white}{\textbf{Errori nell’analisi dei requisiti}}}\\
\hline
\textbf{Descrizione} & Ritardi nella consegna, quantità di ore maggiore rispetto quanto stimato, con conseguenze come aumento dei costi o requisiti opzionali non soddisfatti\\
\textbf{Conseguenze} & Ritardi nella consegna, quantità di ore maggiore, con conseguenze come aumento dei costi o requisiti opzionali non soddisfatti \\
\textbf{Probabilità di manifestarsi} & Bassa \\
\textbf{Pericolosità} & Alta \\
\textbf{Precauzioni} & I verificatori si impegnano a controllare quanto più scrupolosamente possibile ogni requisito individuato dal team e dal committente\glo{}\\
\textbf{Contingenza} & Ogni errore trovato verrà segnalato e discusso con il committente, oltre ad essere gestito con la massima priorità \\
\end{tabular}
\end{table}



\begin{table}[H]
\rowcolors{2}{gray!25}{white}
\renewcommand{\arraystretch}{1.5}
\begin{tabular}{m{0.3\textwidth}<\centering m{0.65\textwidth}<\centering}
\rowcolor{darkblue} \multicolumn{2}{c}{\textcolor{white}{\textbf{Calcolo dei costi}}}\\
\hline
\textbf{Descrizione} & Il committente durante lo sviluppo del progetto può apportare modifiche o aggiungere requisiti obbligatori\\
\textbf{Conseguenze} & Ritardi nella consegna \\
\textbf{Probabilità di manifestarsi} & Bassa \\
\textbf{Pericolosità} & Alta \\
\textbf{Precauzioni} & Ogni incontro con il proponente verrà verbalizzato in modo da tenere traccia di ogni possibile modifica o aggiunta di requisiti \\
\textbf{Contingenza} & Nel caso di cambiamenti di minimo impatto, questi verranno gestiti il prima possibile. Modifiche di più elevata portata verranno discusse con il proponente in modo da trovare un comune accordo \\
\end{tabular}
\end{table}


