\section{Attualizzazione dei rischi}

Nella presente sezione viene esposto come il gruppo \textit{DreamTeam} ha affrontato le avversità (riportate in §2) insorte durante lo svolgimento del progetto di Ingegneria del Software. \\
L’attualizzazione al periodo corrente viene riportata per i soli rischi che si sono effettivamente verificati, così da rendere più fruibili e leggibili i dati esposti in tabella.

\subsection{Rischi legati alle tecnologie}

\begin{table}[H]
\rowcolors{2}{gray!25}{white}
\renewcommand{\arraystretch}{1.5}
\begin{tabular}{m{0.3\textwidth}<\centering m{0.65\textwidth}<\centering}
\rowcolor{darkblue} \multicolumn{2}{c}{\textcolor{white}{\textbf{RT2 - Problemi software}}}\\
\hline
\textbf{Periodo} & Progettazione e Codifica \\
\textbf{Descrizione} & Il software utilizzato per produrre gli arteffati è soggeto a bug, questo ha influito nella realizzazione soprattutto di prodotti di tipo software. Matteo e Francesco infatti si sono scontrati con problemi simili\\
\textbf{Risoluzione} & I soggetti interessati si sono interfacciati con l'azienda, che ha fornito spiegazioni chiare e concise portando il gruppo a superare questo scoglio.\\
\end{tabular}
\end{table}

\begin{table}[H]
\rowcolors{2}{gray!25}{white}
\renewcommand{\arraystretch}{1.5}
\begin{tabular}{m{0.3\textwidth}<\centering m{0.65\textwidth}<\centering}
\rowcolor{darkblue} \multicolumn{2}{c}{\textcolor{white}{\textbf{RT3 - Modifica della piattaforma\glo{} Instagram}}}\\
\hline
\textbf{Periodo} & Progettazione e Codifica \\
\textbf{Descrizione} & La libreria che il gruppo ha deciso di utilizzare, in accordo anche con l'azienda, è soggetta a limitazione da parte della piattaforma Instagram\glo{}.\\
\textbf{Risoluzione} & Dopo averne discusso con l'azienda, il gruppo è arrivato alla conclusione che, non essendo il crawling\glo{} una procedura ufficiale, questo tipo di problema può essere gestito solo cercando un'altra libreria, cosa che però richeidrebbe uno sforzo insostenibile, sia in termini di costi che di tempo.\\
\end{tabular}
\end{table}

\subsection{Rischi personali}

\textbf{DA RIVEDERE}
\begin{table}[H]
\rowcolors{2}{gray!25}{white}
\renewcommand{\arraystretch}{1.5}
\begin{tabular}{m{0.3\textwidth}<\centering m{0.65\textwidth}<\centering}
\rowcolor{darkblue} \multicolumn{2}{c}{\textcolor{white}{\textbf{RP1 - Conflitti decisionali}}}\\
\hline
\textbf{Occorenza} & Bassa \\
\textbf{Descrizione} & Il gruppo \\% TODO: aspettare risposta frontend perché in backend tutto apposto
\textbf{Risoluzione} & Dopo averne discusso con l'azienda, il gruppo è arrivato alla conclusione che, non essendo il crawling\glo{} una procedura ufficiale, questo tipo di problema può essere gestito solo cercando un'altra libreria, cosa che però richeidrebbe uno sforzo insostenibile, sia in termini di costi che di tempo.\\
\end{tabular}
\end{table}

\begin{table}[H]
\rowcolors{2}{gray!25}{white}
\renewcommand{\arraystretch}{1.5}
\begin{tabular}{m{0.3\textwidth}<\centering m{0.65\textwidth}<\centering}
\rowcolor{darkblue} \multicolumn{2}{c}{\textcolor{white}{\textbf{RP2 - Inesperienza Tecnologica}}}\\
\hline
\textbf{Periodo} & Progettazione e Codifica \\
\textbf{Descrizione} & Tutti i membri del gruppo hanno dovuto apprendere un linguaggio nuovo, come React\glo{} e Python\glo{}. Inoltre Alcuni di noi hanno dovuto approfondire l'utilizzo di tecnologie quali Serverless\glo{} e Servizi di AWS\glo{}.\\
\textbf{Risoluzione} & Individuato il problema i membri interessati, assieme al responsabile, solo dopo esausitve ricerche nel web, si sono accordati per chiedere un colloquio all'azienda per cercare una soluzione a quanto riscontrato.\\
\end{tabular}
\end{table}