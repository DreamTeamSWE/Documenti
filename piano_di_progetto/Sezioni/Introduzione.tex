\section{Introduzione}

\subsection{Scopo del Documento}
Nel seguente documento viene illustrato un prospetto di pianificazione in modo dettagliato e delle modalità attraverso le quali avverrà lo sviluppo del progetto. 

Il documento tratterà,  in ordine,  i seguenti punti:
\begin{itemize}
\item analisi dei rischi
\item descrizione del modello di sviluppo adottato
\item suddivisione delle varie fasi con conseguente assegnazione dei ruoli
\item stima dei costi e delle risorse necessarie
\end{itemize}

\subsection{Scopo del Prodotto}

L’obiettivo di Sweeat e dell’azienda Zero12 è la creazione di un sistema software costituito da una Webapp. Lo scopo del prodotto è di fornire all’utente una guida dei locali gastronomici sfruttando i numerosi contenuti digitali creati dagli utenti sulle principali piattaforme social (Instagram e TikTok). In questo modo, è possibile realizzare una classifica basata sulle impressioni e reazioni di chiunque usufruisca dei servizi dei locali, non solo da professionisti ed esperti del settore.

\subsection{Glossario}

Per evitare ambiguità relative alle terminologie utilizzate è stato creato un documento denominato “\textit{Glossario}”. Questo documento comprende tutti i termini tecnici scelti dai membri del gruppo e utilizzati nei vari documenti con le relative definizioni. Tutti i termini inclusi in questo glossario, vengono segnalati all’interno del documento con l’apice \textsuperscript{G} accanto alla parola.

\subsection{Riferimenti}
\subsubsection{Nomativi}
\begin{itemize}
    \item \textit{Norme di Progetto 1.0.0;}
    \item \textbf{Capitolato d'appalto C4:} \newline \href{"https://www.math.unipd.it/~tullio/IS-1/2021/Progetto/C4p.pdf"}.
\end{itemize}
\subsubsection{Informativi}
\begin{itemize}
    \item \textit{Piano di Qualifica 1.0.0;}
    \item \textbf{Software Engineering - Ian Sommerville - 9th Edition (2010):}
        \begin{itemize}
            \item Capitolo 22 - Project management;
            \item Capitolo 23 - Project planning.
        \end{itemize}
    \item \textbf{Il ciclo di vita del software - Materiale didattico del corso di Ingegneria del Software:} \newline \href{"https://www.math.unipd.it/~tullio/IS-1/2021/Dispense/T05.pdf"} 
        \begin{itemize}
            \item Modello Incrementale - slide 19, 20, 21 e 22.
        \end{itemize}
    \item \textbf{Gestione di progetto - Materiale didattico del corso di Ingegneria del Soft- ware:} \newline \href{"https://www.math.unipd.it/~tullio/IS-1/2021/Dispense/T06.pdf"}
\end{itemize}


