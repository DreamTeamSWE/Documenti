\section{Modello di sviluppo}
Come modello di sviluppo si è preferito adottare quello \textbf{incrementale}.

\subsection{Modello incrementale}
Con il modello incrementale ci aspettiamo di avere rilasci successivi dopo ogni incremento. In questo modo viene ridotto il rischio di fallimento ed il lavoro procederà solo dopo l’accettazione da parte del proponente\glo{}.
L’instabilità dei requisiti può essere gestita solo tra un rilascio e l’altro, ma comunque con l’approvazione da parte di \textit{Zero12}.
I principali vantaggi di questo modello sono:
\begin{itemize}
\item possibilità di presentare al proponente un prodotto sempre funzionante;
\item si combina bene con il versionamento\glo, rendendo più visibili le modifiche;
\item gestione delle priorità tra i vari requisiti, dando priorità a funzionalità primarie;
\item gli errori sono limitati all’incremento corrente e la loro correzione è più economica;
\item gli incrementi terminano solo quando verrà accettato il prodotto con quanto di nuovo introdotto, riducendo così la possibilità di trascinare errori durante lo sviluppo del progetto.

\end{itemize}
