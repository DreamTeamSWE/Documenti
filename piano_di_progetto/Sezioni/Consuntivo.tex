\section{Consuntivo}
Di seguito vengono indicate le spese sostenute confrontandole con quelle preventivate per ogni ruolo.  Il bilancio potrà essere:
\begin{itemize}
\item \textbf{positivo}: se la spesa effettiva è minore di quanto preventivato;
\item \textbf{pari}: se la spesa effettiva è uguale a quanto preventivato;
\item \textbf{negativo}: se la spesa effettiva è maggiore di quanto preventivato.
\end{itemize}

\subsection{Fase di analisi}
\subsubsection{Consuntivo}
Le ore di lavoro impiegate nella fase di analisi vengono considerate di investimento; quindi non verranno rendicontate.

\begin{table}[!htbp]
\begin{center}
\rowcolors{2}{gray!25}{white}
\renewcommand{\arraystretch}{1.5}
\begin{tabular}{ m{0.3\textwidth}<{\centering}  m{0.2\textwidth}<{\centering} m{0.2\textwidth}<{\centering}}
	\rowcolor{darkblue}
	\textcolor{white}{\textbf{Ruolo}}&\textcolor{white}{\textbf{Ore}}&\textcolor{white}{\textbf{Costo (\euro) }}\\ 

	Responsabile  & - & - \\	

	Amministratore & - & - \\
	
	Analista & - & - \\
	
	Progettista & - & - \\
	
	Programmatore & - & - \\
	
	Verificatore & - & - \\
	
	\textbf{Totale consuntivo } & - & - \\
	
	\textbf{Totale preventivo} & - & - \\
	
	\textbf{Differenza } & - & - \\
	
\end{tabular}
\caption{Consuntivo della fase di analisi}
\end{center}
\end{table}

\subsubsection{Conclusioni}