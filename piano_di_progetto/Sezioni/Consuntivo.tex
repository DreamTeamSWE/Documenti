\section{Consuntivi di periodo}
Di seguito vengono indicate le spese sostenute confrontandole con quelle preventivate per ogni ruolo.  Il bilancio potrà essere:
\begin{itemize}
\item \textbf{positivo}: se la spesa effettiva è minore di quanto preventivato;
\item \textbf{pari}: se la spesa effettiva è uguale a quanto preventivato;
\item \textbf{negativo}: se la spesa effettiva è maggiore di quanto preventivato.
\end{itemize}

\subsection{Fase di analisi}
<<<<<<< Updated upstream
\subsubsection{Consuntivo}
=======
\subsubsection{I Periodo}
\subsubsubsection{Consuntivo}
Le ore di lavoro che sono state sostenute durante questo periodo sono relative a quanto descritto in §4.1.2.1 dal 2021-11-19 al 2021-11-29.

\begin{table}[!htbp]
\begin{center}
\rowcolors{2}{gray!25}{white}
\renewcommand{\arraystretch}{1.5}
\begin{tabular}{ m{0.18\textwidth}<{\centering}  m{0.14\textwidth}<{\centering} m{0.14\textwidth}<{\centering} m{0.14\textwidth}<{\centering} m{0.16\textwidth}<{\centering} m{0.14\textwidth}<{\centering}}
	\rowcolor{darkblue}
	\textcolor{white}{\textbf{Ruolo}} & \textcolor{white}{\textbf{Ore Effettive}} & \textcolor{white}{\textbf{Ore Preventivate}}&\textcolor{white}{\textbf{Costo Effettivo (\euro) }}&\textcolor{white}{\textbf{Costo Preventivato (\euro)}}&\textcolor{white}{\textbf{Differenza (\euro)}}\\ 

	Responsabile  & 4 & 4 & 120 & 120 & 0\\	
	
	Progettista & 0 & 0 & 0 & 0 & 0\\
	
	Analista & 18 & 18 & 450 & 450 & 0\\
	
	Amministratore & 8 & 8 & 160 & 160 & 0\\
	
	Programmatore & 0 & 0 &0 &0 & 0\\
	
	Verificatore & 0 & 0 & 0 & 0 & 0\\
	
	\textbf{Totale} & 30 & 30 & 730 & 730 & 0\\
	
\end{tabular}
\caption{Consuntivo del I periodo della fase di analisi}
\end{center}
\end{table}


\subsubsection{II Periodo}
\subsubsubsection{Consuntivo}
Le ore di lavoro che sono state sostenute durante questo periodo sono relative a quanto descritto in §4.1.2.2 dal 2021-11-30 al 2022-01-09.

\begin{table}[!htbp]
\begin{center}
\rowcolors{2}{gray!25}{white}
\renewcommand{\arraystretch}{1.5}
\begin{tabular}{ m{0.18\textwidth}<{\centering}  m{0.14\textwidth}<{\centering} m{0.14\textwidth}<{\centering} m{0.14\textwidth}<{\centering} m{0.16\textwidth}<{\centering} m{0.14\textwidth}<{\centering}}
	\rowcolor{darkblue}
	\textcolor{white}{\textbf{Ruolo}} & \textcolor{white}{\textbf{Ore Effettive}} & \textcolor{white}{\textbf{Ore Preventivate}}&\textcolor{white}{\textbf{Costo Effettivo (\euro) }}&\textcolor{white}{\textbf{Costo Preventivato (\euro)}}&\textcolor{white}{\textbf{Differenza (\euro)}}\\ 

	Responsabile  & 5 & 5 & 150 & 150 & 0\\	
	
	Progettista & 0 & 0 & 0 & 0 & 0\\
	
	Analista & 45 & 40 & 1125 & 1000 & +125\\
	
	Amministratore & 20 & 10 & 400 & 200 & +200\\
	
	Programmatore & 0 & 0 &0 &0 & 0\\
	
	Verificatore & 17 & 17 & 255 & 255 & 0\\
	
	\textbf{Totale} & 87 & 72 & 1930 & 1605 & +325\\
	
\end{tabular}
\caption{Consuntivo del II periodo della fase di analisi}
\end{center}
\end{table}


\subsubsection{III Periodo}
\subsubsubsection{Consuntivo}
Le ore di lavoro che sono state sostenute durante questo periodo sono relative a quanto descritto in §4.1.2.3 dal 2022-01-10 al 2022-01-22.

\begin{table}[!htbp]
\begin{center}
\rowcolors{2}{gray!25}{white}
\renewcommand{\arraystretch}{1.5}
\begin{tabular}{ m{0.18\textwidth}<{\centering}  m{0.14\textwidth}<{\centering} m{0.14\textwidth}<{\centering} m{0.14\textwidth}<{\centering} m{0.16\textwidth}<{\centering} m{0.14\textwidth}<{\centering}}
	\rowcolor{darkblue}
	\textcolor{white}{\textbf{Ruolo}} & \textcolor{white}{\textbf{Ore Effettive}} & \textcolor{white}{\textbf{Ore Preventivate}}&\textcolor{white}{\textbf{Costo Effettivo (\euro) }}&\textcolor{white}{\textbf{Costo Preventivato (\euro)}}&\textcolor{white}{\textbf{Differenza (\euro)}}\\ 

	Responsabile  & 6 & 6 & 180 & 180 & 0\\	
	
	Progettista & 0 & 0 & 0 & 0 & 0\\
	
	Analista & 9 & 9 & 225 & 225 & 0\\
	
	Amministratore & 9 & 9 & 180 & 180 & 0\\
	
	Programmatore & 0 & 0 &0 &0 & 0\\
	
	Verificatore & 26 & 21 & 390 & 315 & +75\\
	
	\textbf{Totale} & 50 & 45 & 975 & 900 & +75\\
	
\end{tabular}
\caption{Consuntivo del III periodo della fase di analisi}
\end{center}
\end{table}

\subsubsection{Fase complessiva}
\subsubsubsection{Consuntivo}
>>>>>>> Stashed changes
Le ore di lavoro impiegate nella fase di analisi vengono considerate di investimento; quindi non verranno rendicontate.

\begin{table}[!htbp]
\begin{center}
\rowcolors{2}{gray!25}{white}
\renewcommand{\arraystretch}{1.5}
\begin{tabular}{ m{0.18\textwidth}<{\centering}  m{0.14\textwidth}<{\centering} m{0.14\textwidth}<{\centering} m{0.14\textwidth}<{\centering} m{0.16\textwidth}<{\centering} m{0.14\textwidth}<{\centering}}
	\rowcolor{darkblue}
	\textcolor{white}{\textbf{Ruolo}} & \textcolor{white}{\textbf{Ore Effettive}} & \textcolor{white}{\textbf{Ore Preventivate}}&\textcolor{white}{\textbf{Costo Effettivo (\euro) }}&\textcolor{white}{\textbf{Costo Preventivato (\euro)}}&\textcolor{white}{\textbf{Differenza (\euro)}}\\ 

	Responsabile  & 15 & 15 & 450 & 450 & 0\\	
	
	Progettista & 0 & 0 & 0 & 0 & 0\\
	
	Analista & 72 & 67 & 1800 & 1675 & +125\\
	
	Amministratore & 37 & 27 & 740 & 540 & +200\\
	
	Programmatore & 0 & 0 &0 &0 & 0\\
	
	Verificatore & 43 & 38 & 645 & 570 & +75\\
	
	\textbf{Totale} & 167 & 147 & 3635 & 3235 & +400\\
	
\end{tabular}
\caption{Consuntivo della fase di analisi}
\end{center}
\end{table}

\subsubsection{Conclusioni}
Il bilancio è negativo a causa di una maggior necessità di lavoro nei ruoli di Analista,  Amministratore e Verificatore.  Le motivazioni sono:
\begin{itemize}
\item Analista: in seguito al colloquio con il committente abbiamo dovuto effettuare diverse modifiche alla documentazione;
\item Amministratore: oltre ad aver calcolato male le ore necessarie,  abbiamo riscontrato una difficoltà maggiore del previsto nello stilare le metriche nelle Norme di Progetto;
\item Verificatore: le modiche alla documentazione hanno portato a più ore di verifica.
\end{itemize}

\subsubsection{Preventivo a finire}
Il preventivo a finire,  nonostante in questa fase siano state necessarie più ore di quelle preventivate,  è in linea.  Il surplus di 400,00\euro \xspace non è un problema perché le ore e i costi vengono considerati come investimento e quindi non verranno rendicontati.


\pagebreak


\subsection{Fase di produzione del proof of concept}
<<<<<<< Updated upstream
\subsubsection{Consuntivo}
=======

\subsubsection{I Periodo}
\subsubsubsection{Consuntivo}
Le ore di lavoro che sono state sostenute durante questo periodo sono relative a quanto descritto in §4.2.1.1 dal 2022-01-22 al 2022-01-25.

\begin{table}[!htbp]
\begin{center}
\rowcolors{2}{gray!25}{white}
\renewcommand{\arraystretch}{1.5}
\begin{tabular}{ m{0.18\textwidth}<{\centering}  m{0.14\textwidth}<{\centering} m{0.14\textwidth}<{\centering} m{0.14\textwidth}<{\centering} m{0.16\textwidth}<{\centering} m{0.14\textwidth}<{\centering}}
	\rowcolor{darkblue}
	\textcolor{white}{\textbf{Ruolo}} & \textcolor{white}{\textbf{Ore Effettive}} & \textcolor{white}{\textbf{Ore Preventivate}}&\textcolor{white}{\textbf{Costo Effettivo (\euro) }}&\textcolor{white}{\textbf{Costo Preventivato (\euro)}}&\textcolor{white}{\textbf{Differenza (\euro)}}\\ 
	
	Responsabile  & 3 & 3 & 90 & 90 & 0 \\	
	
	Progettista & 3 & 3 & 75 & 75 & 0 \\
	
	Analista & 7 & 7 & 175 & 175 & 0 \\

	Amministratore & 2 & 2 & 40 & 40 & 0 \\
	
	Programmatore & 6 & 6 & 90 & 90 &  0 \\
	
	Verificatore & 0 & 0 & 0 & 0 & 0 \\
	
	\textbf{Totale} & 21 & 21 & 470 & 470 & 0 \\
	
\end{tabular}
\caption{Consuntivo del I periodo della fase di produzione del proof of concept}
\end{center}
\end{table}


\subsubsection{II Periodo}
\subsubsubsection{Consuntivo}
Le ore di lavoro che sono state sostenute durante questo periodo sono relative a quanto descritto in §4.2.1.2 dal 2022-01-25 al 2022-02-12.

\begin{table}[!htbp]
\begin{center}
\rowcolors{2}{gray!25}{white}
\renewcommand{\arraystretch}{1.5}
\begin{tabular}{ m{0.18\textwidth}<{\centering}  m{0.14\textwidth}<{\centering} m{0.14\textwidth}<{\centering} m{0.14\textwidth}<{\centering} m{0.16\textwidth}<{\centering} m{0.14\textwidth}<{\centering}}
	\rowcolor{darkblue}
	\textcolor{white}{\textbf{Ruolo}} & \textcolor{white}{\textbf{Ore Effettive}} & \textcolor{white}{\textbf{Ore Preventivate}}&\textcolor{white}{\textbf{Costo Effettivo (\euro) }}&\textcolor{white}{\textbf{Costo Preventivato (\euro)}}&\textcolor{white}{\textbf{Differenza (\euro)}}\\ 
	
	Responsabile  & 6 & 6 & 180 & 180 & 0 \\	
	
	Progettista & 17 & 12 & 425 & 300 & +125 \\
	
	Analista & 15 & 15 & 375 & 375 & 0 \\

	Amministratore & 4 & 4 & 80 & 80 & 0 \\
	
	Programmatore & 23 & 17 & 345 & 255 &  +90 \\
	
	Verificatore & 0 & 0 & 0 & 0 & 0 \\
	
	\textbf{Totale} & 88 & 77 & 1750 & 1535 & +215 \\
	
\end{tabular}
\caption{Consuntivo del II periodo della fase di produzione del proof of concept}
\end{center}
\end{table}


\subsubsection{III Periodo}
\subsubsubsection{Consuntivo}
Le ore di lavoro che sono state sostenute durante questo periodo sono relative a quanto descritto in §4.2.1.3  dal 2022-02-12 al 2022-02-13.

\begin{table}[!htbp]
\begin{center}
\rowcolors{2}{gray!25}{white}
\renewcommand{\arraystretch}{1.5}
\begin{tabular}{ m{0.18\textwidth}<{\centering}  m{0.14\textwidth}<{\centering} m{0.14\textwidth}<{\centering} m{0.14\textwidth}<{\centering} m{0.16\textwidth}<{\centering} m{0.14\textwidth}<{\centering}}
	\rowcolor{darkblue}
	\textcolor{white}{\textbf{Ruolo}} & \textcolor{white}{\textbf{Ore Effettive}} & \textcolor{white}{\textbf{Ore Preventivate}}&\textcolor{white}{\textbf{Costo Effettivo (\euro) }}&\textcolor{white}{\textbf{Costo Preventivato (\euro)}}&\textcolor{white}{\textbf{Differenza (\euro)}}\\ 
	
	Responsabile  & 4 & 4 & 120 & 120 & 0\\	
	
	Progettista & 2 & 2 & 50 & 50 & 0\\
	
	Analista & 1  & 1 & 25 & 25 & 0 \\

	Amministratore & 1 & 1 & 20 & 20 & 0 \\
	
	Programmatore & 0 &0 &0 & 0 & 0 \\
	
	Verificatore & 1 & 1 & 15 & 15 & 0 \\
	
	\textbf{Totale} & 9 & 9 & 230 & 230 & 0 \\
	
\end{tabular}
\caption{Consuntivo del III periodo della fase di produzione del proof of concept}
\end{center}
\end{table}

\subsubsection{Fase complessiva}
\subsubsubsection{Consuntivo}
>>>>>>> Stashed changes
Le ore di lavoro impiegate nella fase di produzione del proof of concept,  data l'inesperienza nell'uso delle tecnologie scelte,  vengono considerate di investimento; quindi non verranno rendicontate.

\begin{table}[!htbp]
\begin{center}
\rowcolors{2}{gray!25}{white}
\renewcommand{\arraystretch}{1.5}
\begin{tabular}{ m{0.18\textwidth}<{\centering}  m{0.14\textwidth}<{\centering} m{0.14\textwidth}<{\centering} m{0.14\textwidth}<{\centering} m{0.16\textwidth}<{\centering} m{0.14\textwidth}<{\centering}}
	\rowcolor{darkblue}
	\textcolor{white}{\textbf{Ruolo}} & \textcolor{white}{\textbf{Ore Effettive}} & \textcolor{white}{\textbf{Ore Preventivate}}&\textcolor{white}{\textbf{Costo Effettivo (\euro) }}&\textcolor{white}{\textbf{Costo Preventivato (\euro)}}&\textcolor{white}{\textbf{Differenza (\euro)}}\\ 
	Responsabile  & 13 & 13 & 390 & 390 & 0\\	
	
	Progettista & 22 & 17 & 550 & 425 & +125\\
	
	Analista & 23 & 23 & 575 & 575 & 0\\
	
	Amministratore & 7 & 7 & 140 & 140 & 0\\
	
	Programmatore & 29 & 23 & 435 & 345 &  +90\\
	
	Verificatore & 24 & 24 & 360 & 360 & 0\\
	
<<<<<<< Updated upstream
	\textbf{Totale} & 112 & 107 & 2450 & 2235 & +215\\
=======
	\textbf{Totale} & 118 & 107 & 2450 & 2235 & +215\\
>>>>>>> Stashed changes
	
\end{tabular}
\caption{Consuntivo della fase di produzione del PoC}
\end{center}
\end{table}

\subsubsection{Conclusioni}
Il bilancio è negativo a causa di una maggior necessità di lavoro nel ruolo di Progettista e Programmatore.  Le motivazioni sono:
\begin{itemize}
\item Progettista: inizialmente abbiamo sbagliato a strutturare il PoC;
\item Programmatore: abbiamo riscontrato qualche problema nell'estrapolare i contenuti,  richiesti dal proponente,  dal social TikTok.
\end{itemize}

\subsubsection{Preventivo a finire}
Il preventivo a finire,  nonostante in questa fase siano state necessarie più ore di quelle preventivate,  è in linea.  Il surplus di 215,00\euro \xspace non è un problema perché le ore e i costi vengono considerati come investimento e quindi non verranno rendicontati.