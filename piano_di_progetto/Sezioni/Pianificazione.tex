\section{Pianificazione}
DreamTeam ha deciso di suddividere la pianificazione di progetto in 4 fasi differenti:

\begin{itemize}
\item \textbf{Analisi}
\item \textbf{Progettazione architetturale}
\item \textbf{Progettazione di dettaglio e codifica}
\item	 \textbf{Validazione e controllo}
\end{itemize}

Lo scopo è quello di mostrare come verrà suddiviso il lavoro, valutare i progressi e prevedere e anticipare i problemi che possono verificarsi.

\subsection{Analisi e progettazione architetturale}
\textbf{Periodo:} dal 2022-10-14 al 2022-01-16
\subsubsection{Studio di fattibilità}
Viene effettuata un’analisi dei capitolati proposti in modo da capire i punti di forza e punti critici di ognuno, in modo da identificare il capitolato scelto. Quest’attività risulta bloccante per l’Analisi dei requisiti
\subsubsection{Presentazione per assegnazione appalti}
Viene redatta la presentazione con cui il gruppo di candida per la realizzazione del progetto.
\subsubsection{Piano di progetto}
Presenta la suddivisione e assegnazione di attività, compiti e risorse precedentemente analizzate ai membri del team. È presente inoltre il calcolo del preventivo per la realizzazione del progetto.
\subsubsection{Analisi dei requisiti}
Analisi dei requisiti di cui si compone il capitolato scelto nella fase 4.1.1
\subsubsection{Piano di qualifica}
Definizione delle strategie da adoperare per garantire la qualità del prodotto.
\subsubsection{Glossario}
Vengono definite in modo preciso le definizioni di tutti i termini che possono venir mal interpretati all'interno e all’esterno del gruppo.
\subsubsection{Norme di progetto}
Vengono definite le norme di progetto che il gruppo DreamTeam seguirà durante lo sviluppo del progetto

\textbf{DA RIVEDERE}
\begin{enumerate}
\item \textbf{dal 2021-10-14 al 2021-11-19} In questa prima fase il gruppo esegue uno studio di fattibilità su i capitolati proposti, in questo modo può giungere ad una decisione su quale progetto dare la preferenza e cominciare dunque la stesura della presentazione per l’aggiudicazione dell’appalto e la stesura del preventivo in base anche alle preferenze di ruolo di ogni componente del gruppo. Inoltre sono state prese decisioni come il nome del team, il logo, l’ indirizzo email di riferimento, i giorni in cui riunirsi e gli strumenti con cui comunicare all’interno del gruppo.

\item \textbf{dal 2021-11-20 al 2021-12-31} Il gruppo si suddivide per iniziare a redigere i documenti mancanti: norme di progetto, utile per fissare le regole base del gruppo, il piano di progetto, con la suddivisione del lavoro che ogni membro deve svolgere per completare il progetto. Viene inoltre cominciata la stesura del Glossario.

\item \textbf{dal 2021-12-01 al 2021-12-07} Viene ultimata la stesura del Glossario con un processo di verifica atto a rimuovere termini ridondanti oppure inutili.

\end{enumerate}
