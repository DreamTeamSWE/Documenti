\section{Introduzione}

\subsection{Scopo del Documento}
Questo documento ha il fine di fissare degli standard e degli obiettivi che permettano di quantificare la qualità dei processi e dei prodotti mostrandone l’andamento nel corso dell’intero progetto.
Il documento definirà inoltre le modalità di validazione e verifica al fine di evitare che venga effettuata una verifica di tipo retrospettivo permettendo così di poter rilevare e correggere errori in modo tempestivo e con un costo minore.


\subsection{Scopo del Prodotto}

lo scopo generale del prodotto è quello di creare un sistema software costituito da webapp ed applicazione mobile in grado di fornire all’utente una guida di locali gastronomici. La guida verrà realizzata tramite l’analisi di contenuti digitali (foto, video, storie, commenti) pubblicati sui principali social (instagram e tiktok).  
\subsection{Glossario}

Per evitare ambiguità relative alle terminologie utilizzate è stato creato un documento denominato “\textit{Glossario}”. Questo documento comprende tutti i termini tecnici scelti dai membri del gruppo e utilizzati nei vari documenti con le relative definizioni. Tutti i termini inclusi in questo glossario, vengono segnalati all’interno del documento con l’apice \textsuperscript{G} accanto alla parola.

\subsection{Standard di progetto}
Per il progetto Sweeat, il gruppo DreamTeam ha pensato di adottare lo standard \textbf{ISO/IEC 9126} per la parte relativa alla qualità del software, mentre lo standard \textbf{ISO/IEC 12207} per la parte del ciclo di vita del software.

