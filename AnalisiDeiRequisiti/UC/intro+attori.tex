\section{Casi D'Uso}
\subsection{Introduzione}
In questa sezione verranno presentati i casi d’uso individuati dal gruppo DreamTeam, i quali fanno riferimento a tutte le funzionalità che la piattaforma Sweeat dovrà offrire ad ogni utente che vorrà interfacciarsi con essa.
\subsection{Attori primari}
\begin{itemize}
    \item \textbf{Utente non Autenticato}: utente che non ha ancora effettuato la fase di autenticazione sulla piattaforma. Può essere in possesso o meno delle credenziali per l’autenticazione. Avrà funzionalità limitate rispetto ad un utente autenticato.
    \item \textbf{Utente Autenticato}: utente che ha effettuato l’autenticazione alla piattaforma tramite le proprie credenziali. Ha accesso ad ogni funzionalità messa a disposizione dalla piattaforma.
    \item \textbf{Utente Generico}: può essere sia un utente autenticato che un utente non autenticato.
\end{itemize}
\subsection{Attori secondari}
\begin{itemize}
    \item \textbf{Instagram}: servizio di rete sociale statunitense che permette agli utenti di scattare foto, applicarvi filtri e condividerle via Internet.
    \item \textbf{TikTok}: social network cinese attraverso cui gli utenti possono creare brevi clip musicali.
    \item \textbf{Social Generico}: un qualsiasi social network utilizzato dalla piattaforma Sweeat.
\end{itemize}
\clearpage 