%\subsection{UCW11 - Visualizza informazioni locale}
\begin{figure}[!h]
\centering
\includegraphics[scale=0.5]{UC_images/UCW11.png} 
\caption{UCW11 - Visualizza informazioni locale}
\end{figure}
\begin{itemize}
    \item \textbf{Descrizione}: L'utente generico visualizza le informazioni di un locale.
    \item \textbf{Attore primario}: Utente generico.
    \item \textbf{Precondizione}: L'utente ha svolto la funzione di ricerca di un locale o sta visualizzando la classifica dei locali.
    \item \textbf{Postcondizione}: Vengono visualizzate le informazioni di base di un locale:
    \begin{enumerate}
        \item Visualizza informazioni generali;
        \item Visualizza punteggio locale;
        \item Visualizza dati estratti;
        \item Visualizza icona preferiti.
    \end{enumerate}   
	\item \textbf{Sottocasi}:    
	\begin{enumerate}  
		\item Visualizza informazioni generali (UCW11.1 \S{}3.12.1);
		\item Visualizza punteggio locale (UCW11.2 \S{}3.12.2);
		\item Visualizza contenuti estratti (UCW11.3 \S{}3.12.3);
		\item Visualizza icona preferiti (UCW11.4 \S{}3.12.4).
	\end{enumerate}
    \item \textbf{Scenario principale}: 
    \begin{enumerate}
	\item L'utente visualizza la classifica dei locali presenti nel sistema;
    \item Per ciascun locale, la lista mostrerà alcune informazioni di base.
    \end{enumerate}
\end{itemize}

\begin{figure}[!h]
	\centering
		\includegraphics[scale=0.5]{UC_images/UCW11-.png} 
		\caption{Sottocasi UCW11}
\end{figure}  

\subsubsection{UCW11.1 - Visualizza informazioni generali}
\begin{itemize}
    \item \textbf{Descrizione}: L'utente generico visualizza le informazioni di un locale.
    \item \textbf{Attore primario}: Utente generico.
    \item \textbf{Precondizione}: L'utente ha svolto la funzione di ricerca di un locale o sta visualizzando la classifica dei locali.
    \item \textbf{Postcondizione}: Vengono visualizzate le principali informazioni di base di ciascun locale.
	\item \textbf{Sottocasi}:
	\begin{enumerate}
		\item Visualizza nome locale (UCW11.1.1 \S{}3.12.5);
		\item Visualizza posizione locale (UCW11.1.2 \S{}3.12.6);
		\item Visualizza categoria locale (UCW11.1.3 \S{}3.12.7);
		\item Visualizza orari di apertura locale (UCW11.1.4 \S{}3.12.8);
		\item Visualizza numero di telefono locale (UCW11.1.5 \S{}3.12.9);
		\item Visualizza sito web locale (UCW11.1.6 \S{}3.12.10).
	\end{enumerate}
    \item \textbf{Scenario principale}: 
    \begin{enumerate}
	\item L'utente visualizza la classifica dei locali, inseriti sotto forma di lista, presenti nel sistema;
    \item L'utente identifica un locale tra quelli presenti nella lista;
    \item Il sistema mostrerà all'utente le principali informazioni relative al locale scelto.
    \end{enumerate}
\end{itemize}

\begin{figure}[!h]
	\centering
	\includegraphics[scale=0.5]{UC_images/UCW11-1.png} 
	\caption{Sottocasi UCW11.1}
\end{figure}	

\subsubsection{UCW11.2 - Visualizza punteggio locale}
\begin{itemize}
    \item \textbf{Descrizione}: L'utente generico visualizza il punteggio di un locale.
    \item \textbf{Attore primario}: Utente generico.
    \item \textbf{Precondizione}: L'utente ha svolto la funzione di ricerca di un locale o sta visualizzando la classifica dei locali.
    \item \textbf{Postcondizione}: Viene visualizzato il punteggio del locale selezionato, calcolato analizzando i contenuti pubblicati sui social.
    \item \textbf{Sottocasi}:
	\begin{enumerate}
		\item Visualizza punteggio locale (UCW11.2.1 \S{}3.12.8);
		\item Visualizza punteggio contenuti multimediali (UCW11.2.2 \S{}3.12.12);
		\item Visualizza punteggio testi (UCW11.2.3 \S{}3.12.13);
		\item Visualizza punteggio emoticon (UCW11.2.4 \S{}3.12.14).
	\end{enumerate}
    \item \textbf{Scenario principale}: 
    \begin{enumerate}
	\item L'utente visualizza la classifica dei locali, inseriti sotto forma di lista, presenti nel sistema;    
    \item L'utente identifica un locale tra quelli presenti nella lista;
    \item Il sistema mostrerà all'utente la valutazione complessiva del locale scelto.
    \end{enumerate}
\end{itemize}

\begin{figure}[!h]
	\centering
	\includegraphics[scale=0.5]{UC_images/UCW11-2.png} 
	\caption{Sottocasi UCW11.2}
\end{figure}

\subsubsection{UCW11.3 - Visualizza dati estratti}
\begin{itemize}
    \item \textbf{Descrizione}: L'utente generico visualizza le informazioni di un locale.
    \item \textbf{Attore primario}: Utente generico.
    \item \textbf{Precondizione}: L'utente ha svolto la funzione di ricerca di un locale o sta visualizzando la classifica dei locali.
    \item \textbf{Postcondizione}: Vengono visualizzati i dati del locale selezionato estratti dai social.
    \item \textbf{Sottocasi}:
	\begin{enumerate}
		\item Visualizza foto (UCW11.3.1 \S{}3.12.15);
		\item Visualizza testi dei post (UCW11.3.2 \S{}3.12.16);
		\item Visualizza tag (UCW11.3.3 \S{}3.12.17).
	\end{enumerate}
    \item \textbf{Scenario principale}: 
    \begin{enumerate}
	\item L'utente autenticato visualizza la classifica dei locali, inseriti sotto forma di lista, presenti nel sistema;
    \item L'utente identifica un locale tra quelli presenti nella lista;
    \item Il sistema mostrerà all'utente i dati relativi al locale scelto, che sono stati estratti dai social.
    \end{enumerate}
\end{itemize}

\begin{figure}[!h]
	\centering
	\includegraphics[scale=0.5]{UC_images/UCW11-3.png} 
	\caption{Sottocasi UCW11.3}
\end{figure}

\subsubsection{UCW11.4 - Visualizza icona preferiti}
\begin{itemize}
    \item \textbf{Descrizione}: L'utente autenticato visualizza l'icona “preferiti”.
    \item \textbf{Attore primario}: Utente autenticato.
    \item \textbf{Precondizione}: L'utente autenticato ha svolto la funzione di ricerca di un locale o sta visualizzando la classifica dei locali.
    \item \textbf{Postcondizione}: L'utente autenticato visualizza l'icona in questione per capire se ha inserito o meno il locale identificato nella lista dei preferiti e, nel caso il locale non sia presente nella lista, lo può aggiungere.
    \item \textbf{Scenario principale}: 
    \begin{enumerate}
    \item L'utente autenticato visualizza la classifica dei locali, inseriti sotto forma di lista, presenti nel sistema;
    \item L'utente autenticato identifica il locale di interesse tra quelli presenti nella lista;
    \item Il sistema mostrerà all'utente autenticato se quel locale è presente o meno nella sua lista dei preferiti e, nel caso non lo sia, lo può aggiungere (UCW12 \S{}3.13) o rimuovere (UCW13 \S{}3.14).
    \end{enumerate}
\end{itemize}

\subsubsection{UCW11.1.1 - Visualizza nome locale}
\begin{itemize}
    \item \textbf{Descrizione}: L'utente generico visualizza il nome di un locale.
    \item \textbf{Attore primario}: Utente generico.
    \item \textbf{Precondizione}: L'utente ha svolto la funzione di ricerca di un locale o sta visualizzando la classifica dei locali.
    \item \textbf{Postcondizione}: Viene visualizzato il nome del locale in questione.
    \item \textbf{Scenario principale}: 
    \begin{enumerate}
	\item L'utente visualizza la classifica dei locali, inseriti sotto forma di lista, presenti nel sistema;
    \item L'utente identifica un locale tra quelli presenti nella lista;
	\item Il sistema mostrerà all'utente il nome del locale identificato.
    \end{enumerate}
\end{itemize}

\subsubsection{UCW11.1.2 - Visualizza posizione locale}
\begin{itemize}
    \item \textbf{Descrizione}: L'utente generico visualizza la posizione di un locale.
    \item \textbf{Attore primario}: Utente generico.
    \item \textbf{Precondizione}: L'utente ha svolto la funzione di ricerca di un locale o sta visualizzando la classifica dei locali.
    \item \textbf{Postcondizione}: Viene visualizzata la posizione del locale in questione.
    \item \textbf{Scenario principale}: 
    \begin{enumerate}
	\item L'utente visualizza la classifica dei locali, inseriti sotto forma di lista, presenti nel sistema;
    \item L'utente identifica un locale tra quelli presenti nella lista;
	\item Il sistema mostrerà all'utente la posizione del locale identificato.
    \end{enumerate}
\end{itemize}

\subsubsection{UCW11.1.3 - Visualizza categoria locale}
\begin{itemize}
    \item \textbf{Descrizione}: L'utente generico visualizza la categoria di un locale.
    \item \textbf{Attore primario}: Utente generico.
    \item \textbf{Precondizione}: L'utente ha svolto la funzione di ricerca di un locale o sta visualizzando la classifica dei locali.
    \item \textbf{Postcondizione}: Viene visualizzata la categoria del locale in questione.
    \item \textbf{Scenario principale}: 
    \begin{enumerate}
	\item L'utente visualizza la classifica dei locali, inseriti sotto forma di lista, presenti nel sistema;
	\item L'utente identifica un locale tra quelli presenti nella lista;
	\item Il sistema mostrerà all'utente la categoria del locale identificato.
    \end{enumerate}
\end{itemize}

\subsubsection{UCW11.1.4 - Visualizza orari di apertura locale}
\begin{itemize}
    \item \textbf{Descrizione}: L'utente generico visualizza gli orari di apertura di un locale.
    \item \textbf{Attore primario}: Utente generico.
    \item \textbf{Precondizione}: L'utente ha svolto la funzione di ricerca di un locale o sta visualizzando la classifica dei locali.
    \item \textbf{Postcondizione}: Vengono visualizzati gli orari di apertura del locale selezionato.
    \item \textbf{Scenario principale}: 
    \begin{enumerate}
	\item L'utente visualizza la classifica dei locali, inseriti sotto forma di lista, presenti nel sistema;
	\item L'utente identifica un locale tra quelli presenti nella lista;
	\item Il sistema mostrerà all'utente gli orari di apertura del locale identificato.
    \end{enumerate}
\end{itemize}

\subsubsection{UCW11.1.5 - Visualizza numero di telefono locale}
\begin{itemize}
    \item \textbf{Descrizione}: L'utente generico visualizza il numero di telefono di un locale.
    \item \textbf{Attore primario}: Utente generico.
    \item \textbf{Precondizione}: L'utente ha svolto la funzione di ricerca di un locale o sta visualizzando la classifica dei locali.
    \item \textbf{Postcondizione}: Viene visualizzato il numero di telefono del locale selezionato.
    \item \textbf{Scenario principale}: 
    \begin{enumerate}
	\item L'utente visualizza la classifica dei locali, inseriti sotto forma di lista, presenti nel sistema;
	\item L'utente identifica un locale tra quelli presenti nella lista;
	\item Il sistema mostrerà all'utente il numero di telefono del locale identificato.
    \end{enumerate}
\end{itemize}

\subsubsection{UCW11.1.6 - Visualizza sito web locale}
\begin{itemize}
    \item \textbf{Descrizione}: L'utente generico visualizza il sito web di un locale.
    \item \textbf{Attore primario}: Utente generico.
    \item \textbf{Precondizione}: L'utente ha svolto la funzione di ricerca di un locale o sta visualizzando la classifica dei locali.
    \item \textbf{Postcondizione}: Viene visualizzato il sito web del locale in questione.
    \item \textbf{Scenario principale}: 
    \begin{enumerate}
	\item L'utente visualizza la classifica dei locali, inseriti sotto forma di lista, presenti nel sistema;
	\item L'utente identifica un locale tra quelli presenti nella lista;
	\item Il sistema mostrerà all'utente il sito web del locale identificato.
    \end{enumerate}
\end{itemize}

\subsubsection{UCW11.2.1 - Visualizza punteggio locale}
\begin{itemize}
    \item \textbf{Descrizione}: L'utente generico visualizza il punteggio totale di un locale.
    \item \textbf{Attore primario}: Utente generico.
    \item \textbf{Precondizione}: L'utente ha svolto la funzione di ricerca di un locale o sta visualizzando la classifica dei locali.
    \item \textbf{Postcondizione}: Viene visualizzato il punteggio totale del locale in questione.
    \item \textbf{Scenario principale}: 
    \begin{enumerate}
        \item L'utente visualizza la classifica dei locali, inseriti sotto forma di lista, presenti nel sistema;
        \item L'utente identifica un locale tra quelli presenti nella lista;
        \item Il sistema mostrerà all'utente il punteggio totale del locale identificato.
        \end{enumerate}
\end{itemize}
\subsubsection{UCW11.2.2 - Visualizza punteggio contenuti multimediali}
\begin{itemize}
    \item \textbf{Descrizione}: L'utente generico visualizza il punteggio dei contenuti multimediali di un locale.
    \item \textbf{Attore primario}: Utente generico.
    \item \textbf{Precondizione}: L'utente ha svolto la funzione di ricerca di un locale o sta visualizzando la classifica dei locali.
    \item \textbf{Postcondizione}: Viene visualizzato il punteggio dei contenuti multimediali del locale in questione.
    \item \textbf{Scenario principale}: 
    \begin{enumerate}
        \item L'utente visualizza la classifica dei locali, inseriti sotto forma di lista, presenti nel sistema;
        \item L'utente identifica un locale tra quelli presenti nella lista;
        \item Il sistema mostrerà all'utente il punteggio dei contenuti multimediali del locale identificato.
        \end{enumerate}
\end{itemize}
\subsubsection{UCW11.2.3 - Visualizza punteggio testi}
\begin{itemize}
    \item \textbf{Descrizione}: L'utente generico visualizza il punteggio dei testi di un locale.
    \item \textbf{Attore primario}: Utente generico.
    \item \textbf{Precondizione}: L'utente ha svolto la funzione di ricerca di un locale o sta visualizzando la classifica dei locali.
    \item \textbf{Postcondizione}: Viene visualizzato il punteggio dei testi del locale in questione.
    \item \textbf{Scenario principale}: 
    \begin{enumerate}
        \item L'utente visualizza la classifica dei locali, inseriti sotto forma di lista, presenti nel sistema;
        \item L'utente identifica un locale tra quelli presenti nella lista;
        \item Il sistema mostrerà all'utente il punteggio dei testi del locale identificato.
        \end{enumerate}
\end{itemize}

\subsubsection{UCW11.2.4 - Visualizza punteggio emoticon}
\begin{itemize}
    \item \textbf{Descrizione}: L'utente generico visualizza il punteggio delle emoticon di un locale.
    \item \textbf{Attore primario}: Utente generico.
    \item \textbf{Precondizione}: L'utente ha svolto la funzione di ricerca di un locale o sta visualizzando la classifica dei locali.
    \item \textbf{Postcondizione}: Viene visualizzato il punteggio delle emoticon del locale in questione.
    \item \textbf{Scenario principale}: 
    \begin{enumerate}
        \item L'utente visualizza la classifica dei locali, inseriti sotto forma di lista, presenti nel sistema;
        \item L'utente identifica un locale tra quelli presenti nella lista;
        \item Il sistema mostrerà all'utente il punteggio delle emoticon del locale identificato.
        \end{enumerate}
\end{itemize}

\subsubsection{UCW11.3.1 - Visualizza foto}
\begin{itemize}
    \item \textbf{Descrizione}: L'utente generico visualizza le foto di un locale.
    \item \textbf{Attore primario}: Utente generico.
    \item \textbf{Precondizione}: L'utente ha svolto la funzione di ricerca di un locale o sta visualizzando la classifica dei locali.
    \item \textbf{Postcondizione}: Viene visualizzato le foto del locale in questione.
    \item \textbf{Scenario principale}: 
    \begin{enumerate}
        \item L'utente visualizza la classifica dei locali, inseriti sotto forma di lista, presenti nel sistema;
        \item L'utente identifica un locale tra quelli presenti nella lista;
        \item Il sistema mostrerà all'utente le foto del locale identificato.
        \end{enumerate}
\end{itemize}
\subsubsection{UCW11.3.2 - Visualizza testi dei post}
\begin{itemize}
    \item \textbf{Descrizione}: L'utente generico visualizza i testi dei post di un locale.
    \item \textbf{Attore primario}: Utente generico.
    \item \textbf{Precondizione}: L'utente ha svolto la funzione di ricerca di un locale o sta visualizzando la classifica dei locali.
    \item \textbf{Postcondizione}: Viene visualizzato i testi dei post del locale in questione.
    \item \textbf{Scenario principale}: 
    \begin{enumerate}
        \item L'utente visualizza la classifica dei locali, inseriti sotto forma di lista, presenti nel sistema;
        \item L'utente identifica un locale tra quelli presenti nella lista;
        \item Il sistema mostrerà all'utente i testi dei post del locale identificato.
        \end{enumerate}
\end{itemize}
\subsubsection{UCW11.3.3 - Visualizza tag}
\begin{itemize}
    \item \textbf{Descrizione}: L'utente generico visualizza le tag di un locale.
    \item \textbf{Attore primario}: Utente generico.
    \item \textbf{Precondizione}: L'utente ha svolto la funzione di ricerca di un locale o sta visualizzando la classifica dei locali.
    \item \textbf{Postcondizione}: Viene visualizzato le tag del locale in questione.
    \item \textbf{Scenario principale}: 
    \begin{enumerate}
        \item L'utente visualizza la classifica dei locali, inseriti sotto forma di lista, presenti nel sistema;
        \item L'utente identifica un locale tra quelli presenti nella lista;
        \item Il sistema mostrerà all'utente le tag del locale identificato.
        \end{enumerate}
\end{itemize}