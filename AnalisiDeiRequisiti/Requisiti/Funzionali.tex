%\subsection{Requisiti Funzionali}

\definecolor{darkblue}{cmyk}{99, 99, 0, 71}

\renewcommand{\arraystretch}{1.5}
\rowcolors{2}{gray!25}{white}
\begin{longtable}{ m{0.15\textwidth}<{\centering}  m{0.4\textwidth}<{\centering}  m{0.16\textwidth}<{\centering}  m{0.19\textwidth}<{\centering}}
	\rowcolor{darkblue}
	\textcolor{white}{\textbf{Requisito}} &\textcolor{white}{\textbf{Descrizione}}& \textcolor{white}{\textbf{Classificazione}} & \textcolor{white}{\textbf{Fonti}}\\ 

	R1FW1 & L’utente deve riuscire ad inserire i propri dati personali (nome, cognome, indirizzo e-mail e password) per effettuare la registrazione & \Ob & UCW1 \\	
	 
	R1FW2 & L’utente deve riuscire ad inserire i propri dati (indirizzo e-mail e password) per effettuare il login & \Ob & UCW2\\	

	R2FW3 & L’utente deve riuscire a recuperare la password, nel caso l’avesse dimenticata & \De & UCW3\\	
	 
	R1FE1 & All’utente viene mostrato un errore nel caso non inserisca correttamente il nome in fase di registrazione & \Ob & UCE1\\	
	 
 	R1FE2 & All’utente viene mostrato un errore nel caso non inserisca correttamente il cognome in fase di registrazione & \Ob & UCE2\\	
	 
	R1FE3 & All’utente viene mostrato un errore nel caso non inserisca correttamente l’indirizzo e-mail in fase di registrazione & \Ob & UCE3\\	

	R1FE4 & All’utente viene mostrato un errore nel caso non inserisca una password corretta in fase di registrazione & \Ob & UCE4\\	
	
	R1FE5 & All'utente viene mostrato un errore nel caso non inserisca correttamente l'indirizzo e-mail al login & \Ob & UCE5 \\
	 
	R1FE6 & All'utente viene mostrato un errore nel caso non inserisca correttamente la password al login & \Ob & UCE6 \\	 
	 
	R1FE7 & All’utente viene mostrato un errore nel caso non inserisca correttamente l’indirizzo e-mail nel recupero password & \Ob & UCE7\\	

	R1FW4 &	Un utente autenticato può accedere alla sua Area Personale & \Ob & UCW4 \\ 
	 
	R1FW4.1 & L’utente autenticato può collegare il proprio profilo Instagram  & \Ob & UCW4.1\\	
	 
	R1FE8 & All’utente autenticato viene mostrato un messaggio d’errore nel caso il collegamento al profilo Instagram non vada a buon fine & \Ob & UCE8\\	
	 
	R1FW4.2 & L’utente autenticato può collegare al proprio profilo l’account TikTok & \Ob & UCW4.2\\		 

	R1FE9 & All’utente autenticato viene mostrato un messaggio d’errore nel caso il collegamento al profilo TikTok non vada a buon fine  & \Ob & UCE9 \\		
	 
	R3FW4.3 & L’utente autenticato può modificare la password con cui accede al sistema & \Fa & UCW4.3\\				
	 
	R3FE15 & All’utente autenticato viene mostrato un errore nel caso non inserisca una password valida, in fase di modifica  & \Fa & UCE15\\			
	  	 	 	
	R1FW5 & L’utente autenticato può suggerire dei profili social da cui fare il crawling dei dati & \Ob & UCW5, UCW6\\		
	 
	R2FE10 & All’utente autenticato viene mostrato un messaggio d’errore nel caso suggerisca un profilo social inesistente & \De & UCE10\\		

	R2FE11 & All’utente autenticato viene mostrato un messaggio d’errore nel caso l’utente suggerito abbia un profilo privato & \De & UCE11\\
	 
	R2FE12 & All’utente autenticato viene mostrato un messaggio d’errore nel caso suggerisca un profilo già presente a sistema & \De & UCE12\\			
	 
	R1FW7 & L’utente può visualizzare la classifica con i locali presenti nel database della piattaforma & \Ob & UCW7\\	
	
	R1FW7.1 & L’utente è in grado di visualizzare i nomi dei locali presenti nella classifica & \Ob & UCW7.1 \\

	R1FW7.2 & All’utente viene data la possibilità di visualizzare i punteggi totali relativi a ciascun locale & \Ob & UCW7.2 \\

	R2FW7.3 & L’utente può visualizzare le varie categorie a cui appartengono i vari locali presenti nella classifica & \De & UCW7.3 \\

	R2FW7.4 & All’utente viene data la possibilità di visualizzare le foto di copertina di ciascun locale & \De & UCW7.4 \\
	 
	R1FW8 & L’utente può filtrare la classifica & \Ob & UCW8\\		
	
	R2FE13 & Nel caso non ci sia alcun risultato compatibile con i filtri applicati, viene mostrato un errore & \De & UCE13\\	
	 
	R1FW8.1 & L’utente può filtrare la classifica dei locali presenti nel database del sistema in base alla zona & \Ob & UCW8.1\\	
	 
	R2FW8.2 & L’utente può filtrare la classifica dei locali presenti nel database del sistema per giorno ed orario di apertura & \De & UCW8.2\\	
	 
	R2FW8.3 & L’utente può filtrare la classifica dei locali presenti nel database del sistema in base al tipo di cucina & \De & UCW8.3\\	
	 
	R2FW8.4 & L’utente può filtrare la classifica dei locali presenti nel database del sistema per fascia di prezzo & \De & UCW8.4\\	 
	 
	R1FW8.5 & L’utente può filtrare la classifica dei locali presenti nel database del sistema in base al punteggio & \Ob & UCW8.5\\	 
	 
	R3FW9 & L’utente può modificare l’ordinamento di visualizzazione della classifica & \Fa & UCW9\\	
	 
	R3FW9.1 & L’utente può modificare la visualizzazione dei risultati della classifica, impostando il peso dei social & \Fa & UCW9.1\\	 
	 
	R3FW9.2 & L’utente può modificare la visualizzazione dei risultati della classifica, impostando il peso dei tipi di contenuto & \Fa & UCW9.2\\	  
	 
	R1FW10 & L’utente può cercare un locale presente nel database del sistema tramite il suo nome & \Ob & UCW10 \\	 
	 
	R2FE14 & All’utente viene mostrato un errore in caso il locale cercato non sia presente nel sistema & \De & UCE14\\	 
	 	 
	R1FW11 & L’utente può visualizzare le informazioni di un locale presente nel sistema & \Ob & UCW11\\	

	R2FW11.1 & L’utente può visualizzare le informazioni generali di un locale & \De & UCW11.1 \\
	R2FW11.2 & L’utente può visualizzare il punteggio totale di un locale & \De & UCW11.2 \\
	R2FW11.3 & L’utente può visualizzare i contenuti relativi ad un locale estratti dai social & \De & UCW11.3 \\
	R2FW11.4 & L’utente può visualizzare l’icona dei preferiti per aggiungere o rimuovere quel locale dalla/nella lista dei preferiti & \De & UCW11.4 \\

	R1FW11.1.1 & L’utente visualizza il nome di un locale & \Ob & UCW11.1.1 \\
	R2FW11.1.2 & L’utente visualizza la posizione di un locale su GoogleMaps & \De & UCW11.1.2 \\
	R2FW11.1.3 & L’utente visualizza la categoria a cui appartiene un locale & \De & UCW11.1.3 \\
	R3FW11.1.4 & L’utente visualizza gli orari di apertura di un locale & \Fa & UCW11.1.4 \\
	R2FW11.1.5 & L’utente visualizza il numero di telefono di un locale & \De & UCW11.1.5 \\
	R2FW11.1.6 & L’utente visualizza un bottone che rimanda al sito web del locale & \De & UCW11.1.6 \\

	R1FW11.2.1 & L’utente visualizza il punteggio totale di un locale & \Ob & UCW11.2.1 \\
	R1FW11.2.2 & L’utente visualizza il punteggio totale dei contenuti multimediali pubblicati sui social, relativi ad un locale & \Ob & UCW11.2.2 \\
	R1FW11.2.3 & L’utente visualizza il punteggio totale dei testi dei post pubblicati sui social relativi ad un locale & \Ob & UCW11.2.3 \\
	R1FW11.2.4 & L’utente visualizza il punteggio totale delle emoticon relative al gradimento di un locale sui social & \Ob & UCW11.2.4 \\

	R2FW11.3.1 & L’utente visualizza le foto pubblicate sui social relative ad un locale & \De & UCW11.3.1 \\
	R2FW11.3.2 & L’utente visualizza i contenuti testuali dei post pubblicati sui social e relativi ad un locale & \De & UCW11.3.2 \\
	R2FW11.3.3 & L’utente visualizza i tag relativi ai post pubblicati sui social e relativi ad un locale & \De & UCW11.3.3 \\ 	 	 	

	R2FW12 & L’utente può aggiungere un locale alla lista dei preferiti & \De &  UCW12\\ 	 	 	

	R2FW13 & L’utente può rimuovere un locale dalla lista dei preferiti & \De & UCW13\\

	R3F1 & L’utente può suggerire delle modifiche da apportare relative alle informazioni di un locale & \Fa & \Di \\

	\hiderowcolors \caption{Requisiti Funzionali}
\end{longtable}

\clearpage