\subsection{Requisiti Funzionali}

\definecolor{darkblue}{cmyk}{99, 99, 0, 71}

\renewcommand{\arraystretch}{1.5}
\rowcolors{2}{gray!25}{white}
\begin{longtable}{ m{0.2\textwidth}<{\centering}  m{0.35\textwidth}<{\centering}  m{0.16\textwidth}<{\centering}  m{0.19\textwidth}<{\centering}}
	\rowcolor{darkblue}
	\textcolor{white}{\textbf{Requisito}} &\textcolor{white}{\textbf{Descrizione}}& \textcolor{white}{\textbf{Classificazione}} & \textcolor{white}{\textbf{Fonti}}\\ 

	R1FW1 & L’utente deve riuscire ad inserire i propri dati personali (nome, cognome, indirizzo e-mail e password) per effettuare la registrazione & \Ob & UCW1, Decisione Interna\\	
	 
	R1FW2 & L’utente deve riuscire ad inserire i propri dati (indirizzo e-mail e password) per effettuare il login & \Ob & UCW2\\	

	R2FW3 & L’utente deve riuscire a recuperare la password, nel caso l’avesse dimenticata & \De & UCW3\\	
	 
	R2FE1 & All’utente viene mostrato un errore nel caso non inserisca correttamente il nome in fase di registrazione & \De & UCE1\\	
	 
 	R2FE2 & All’utente viene mostrato un errore nel caso non inserisca correttamente il cognome in fase di registrazione & \De & UCE2\\	
	 
	R2FE3 & All’utente viene mostrato un errore nel caso non inserisca correttamente l’indirizzo e-mail in fase di registrazione & \De & UCE3\\	

	R2FE4 & All’utente viene mostrato un errore nel caso non inserisca una password corretta in fase di registrazione & \De & UCE4\\	
	
	R2FE5 & All'utente viene mostrato un errore nel caso non inserisca correttamente l'indirizzo e-mail al login & \De & UCE5 \\
	 
	R2FE6 & All'utente viene mostrato un errore nel caso non inserisca correttamente la password al login & \De & UCE6 \\	 
	 
	R2FE7 & All’utente viene mostrato un errore nel caso non inserisca correttamente l’indirizzo e-mail nel recupero password & \De & UCE7\\	

	R1FW4 &	Un utente registrato può accedere alla sua Area Personale & \Ob & UCW4 \\ 
	 
	R2FW4.1 & L’utente autenticato può collegare il proprio profilo Instagram  & \De & UCW4.1\\	
	 
	R2FE8 & All’utente autenticato viene mostrato un messaggio d’errore nel caso il collegamento al profilo Instagram non vada a buon fine & \De & UCE8\\	
	 
	R2FW4.2 & L’utente autenticato può collegare al proprio profilo l’account TikTok & \De & UCW4.2\\		 

	R2FE9 & All’utente autenticato viene mostrato un messaggio d’errore nel caso il collegamento al profilo TikTok non vada a buon fine  & \De & UCE9 \\		
	 
	R3FW4.3 & L’utente autenticato può modificare la password con cui accede al sistema & \Fa & UCW4.3\\				
	 
	R3FE15 & All’utente autenticato viene mostrato un errore nel caso non inserisca una password valida, in fase di modifica  & \Fa & UCE15\\			
	  	 	 	
	R1FW5 & L’utente autenticato può suggerire dei profili social da cui fare il crawling dei dati & \Ob & UCW5, Decisione interna\\		
	 
	R2FE10 & All’utente viene mostrato un messaggio d’errore nel caso suggerisca un profilo social inesistente & \De & UCE10\\		

	R2FE11 & All’utente viene mostrato un messaggio d’errore nel caso l’utente suggerito abbia un profilo privato & \De & UCE11\\
	 
	R2FE12 & All’utente viene mostrato un messaggio d’errore nel caso suggerisca un profilo già presente a sistema & \De & UCE12\\			
	 
	R1FW7 & L’utente può visualizzare la classifica con i locali presenti nel database della piattaforma & \Ob & UCW7\\	
	 
	R1FW8 & L’utente può filtrare la classifica & \Ob & UCW8\\		
	
	R2FE13 & Nel caso non ci sia alcun risultato compatibile con i filtri applicati, viene mostrato un errore & \De & UCE13\\	
	 
	R2FW8.1 & L’utente può filtrare la classifica dei locali presenti nel database del sistema in base alla zona & \De & UCW8.1\\	
	 
	R2FW8.2 & L’utente può filtrare la classifica dei locali presenti nel database del sistema per giorno ed orario di apertura & \De & UCW8.2\\	
	 
	R2FW8.3 & L’utente può filtrare la classifica dei locali presenti nel database del sistema in base al tipo di cucina & \De & UCW8.3\\	
	 
	R2FW8.4 & L’utente può filtrare la classifica dei locali presenti nel database del sistema per fascia di prezzo & \De & UCW8.4\\	 
	 
	R2FW8.5 & L’utente può filtrare la classifica dei locali presenti nel database del sistema in base al punteggio & \De & UCW8.5\\	 
	 
	R2FW9 & L’utente può modificare l’ordinamento di visualizzazione della classifica & \De & UCW9\\	
	 
	R2FW9.1 & L’utente può modificare la visualizzazione dei risultati della classifica, impostando il peso dei social & \De & UCW9.1\\	 
	 
	R2FW9.2 & L’utente può modificare la visualizzazione dei risultati della classifica, impostando il peso dei tipi di contenuto & \De & UCW9.2\\	  
	 
	R2FW10 & L’utente può cercare un locale presente nel database del sistema tramite il suo nome & \Ob & UCW10, Decisione interna\\	 
	 
	R2FE14 & All’utente viene mostrato un errore in caso il locale cercato non sia presente nel sistema & \De & UCE14\\	 
	 
	R2F & L’utente può visualizzare dei locali alternativi, nel caso il locale cercato non sia presente nel sistema & \De ? & Decisione interna ?\\	 
	 
	R1FW11 & L’utente può visualizzare le informazioni di un locale presente nel sistema & \Ob & UCW11\\	 	 	 	

	R2FW12 & L’utente può aggiungere un locale alla lista dei preferiti & \De &  UCW12\\

	R2FW13 & L’utente può rimuovere un locale dalla lista dei preferiti & \De & UCW13\\

	R3F & L’utente può suggerire delle modifiche da apportare relative alle informazioni di un locale & \Fa & Decisione interna\\

	\caption{Requisiti Funzionali}
\end{longtable}

\clearpage