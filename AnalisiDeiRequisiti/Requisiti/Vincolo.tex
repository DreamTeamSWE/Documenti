\subsection{Requisiti di Vincolo}

\definecolor{darkblue}{cmyk}{99, 99, 0, 71}

\renewcommand{\arraystretch}{1.5}
\rowcolors{2}{gray!25}{white}
\begin{longtable}{ m{0.15\textwidth}<{\centering}  m{0.4\textwidth}<{\centering}  m{0.16\textwidth}<{\centering}  m{0.19\textwidth}<{\centering}}
	\rowcolor{darkblue}
	\textcolor{white}{\textbf{Requisito}} &\textcolor{white}{\textbf{Descrizione}}& \textcolor{white}{\textbf{Classificazione}} & \textcolor{white}{\textbf{Fonti}}\\ 

	R1V1 & L’interfaccia utente del sistema dovrà essere sviluppato sfruttando il framework React & \Ob & \Vi{} 2022-01-13 \\	

	R1V2 & Il sistema dovrà funzionare sul browser Chrome dalla versione più recente (97.0.4692) & \Ob & \Ve{} 2022-01-26 \\	
	 
	R1V3 & Il sistema dovrà funzionare sul browser Microsoft Edge dalla versione più recente (96.0.1031.0) & \Ob & \Ve{} 2022-01-26 \\	

	R1V4 & Il sistema dovrà funzionare sul browser Firefox dalla versione più recente (96.0.2) & \Ob & \Ve{} 2022-01-26 \\	
	 
	R1V5 & Il sistema dovrà funzionare sul browser Safari dalla versione più recente (15.3) & \Ob & \Ve{} 2022-01-26 \\	
	 
	R1V6 & Il sistema dovrà sfruttare i servizi offerti da AWS & \Ob & \Ca \\	
	 
	R2V1 & Il sistema dovrà sfruttare il database Amazon Aurora Serverless & \De & \Vi{} 2022-01-13 \\
	
	R1V7 & Il sistema dovrà usare le Lambda di AWS & \Ob & \Ca \\	
	 
	R2V2 & Il sistema dovrà sfruttare le API Gateway & \Ob & \Ca \\	 

	R2V3 & Le API per lo sviluppo del crawler Instagram dovranno essere sviluppate in Python & \De & \Vi{} 2022-01-13 \\	
	 
	R2V4 & Le API per lo sviluppo del crawler TikTok dovranno essere sviluppate in Python & \De & \Vi{} 2022-01-13 \\	
	 
	R1V8 & È necessario sfruttare un’architettura a micro-servizi & \Ob & \Ca \\	
	 
	R1V9 & È necessario non superare la soglia di 2000\$ di credito per i servizi offerti da AWS & \Ob & \Ve{} 2022-01-26 \\	
	
	\hiderowcolors \caption{Requisiti di Vincolo}
\end{longtable}

\pagebreak