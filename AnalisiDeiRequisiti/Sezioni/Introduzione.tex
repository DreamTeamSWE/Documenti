\section{Introduzione}

\subsection{Scopo del Documento}
Lo scopo di questo documento è di descrivere dettagliatamente i requisiti del progetto definendone i casi d’uso individuati nello studio del progetto.

\subsection{Scopo del Prodotto}

L’obiettivo di Sweeat e dell’azienda Zero12 è la creazione di un sistema software costituito da una Webapp. Lo scopo del prodotto è di fornire all’utente una guida dei locali gastronomici sfruttando i numerosi contenuti digitali creati dagli utenti sulle principali piattaforme social (Instagram e TikTok). In questo modo, è possibile realizzare una classifica basata sulle impressioni e reazioni di chiunque usufruisca dei servizi dei locali, non solo da professionisti ed esperti del settore.

\subsection{Glossario}

Per evitare ambiguità relative alle terminologie utilizzate è stato creato un documento denominato “\textit{Glossario}”. Questo documento comprende tutti i termini tecnici scelti dai membri del gruppo e utilizzati nei vari documenti con le relative definizioni. Tutti i termini inclusi in questo glossario, vengono segnalati all’interno del documento con l’apice\textsuperscript{G} accanto alla parola.

\subsection{Riferimenti}

\subsubsection{Riferimenti normativi}
\begin{itemize}
    \item Norme di Progetto v1.0.0;
    \item Verbale Esterno 2021-12-22;
    \item Presentazione del capitolato - Zero12 Progettazione e sviluppo di una Social guida Michelin: \newline \mylink{https://www.math.unipd.it/~tullio/IS-1/2021/Progetto/C4p.pdf}.
\end{itemize}
\subsubsection{Riferimenti informativi}
\begin{itemize}
    \item Casi d'uso - Materiale didattico del corso di Ingegneria del Software: \newline\mylink{https://www.math.unipd.it/~rcardin/swea/2022/Diagrammi\%20Use\%20Case.pdf};
    \item Analisi dei requisiti - Materiale didattico del corso di Ingegneria del Software: \newline \mylink{ https://www.math.unipd.it/~tullio/IS-1/2021/Dispense/T07.pdf};
    \item Regolamento del progetto didattico - Materiale didattico del corso di Ingegneria del Software:\newline \mylink{https://www.math.unipd.it/~tullio/IS-1/2021/Dispense/PD2.pdf}.
\end{itemize}


