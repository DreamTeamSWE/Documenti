\section{Descrizione Generale}

\subsection{Caratteristiche del Prodotto}

A seguito della presentazione del capitolato e dei primi due incontri fatti col proponente è emerso che il prodotto che andremo a realizzare dovrà avere le seguenti caratteristiche:

\subsubsection{Crawling dei dati da Instagram e TikTok}
Il team dovrà effettuare un’analisi sulle API \textsuperscript{G} social di Instagram e TikTok per capire se esse sono sufficienti a raccogliere e analizzare le informazioni, in caso contrario si dovrà individuare una soluzione alternativa per effettuare tali operazioni. In ogni caso si dovranno analizzare esclusivamente profili pubblici ed escogitare una strategia per evitare di essere inseriti nelle black-list \textsuperscript{G} di Instagram e TikTok a causa di queste operazioni.

\subsubsection{Traduzione dei dati ottenuti}
Nella fase di crawling \textsuperscript{G} verranno raccolti dati di diverso tipo quali foto, video, stories, post e commenti. Questi dati andranno poi “tradotti” utilizzando diversi servizi come ad esempio Amazon Transcribe \textsuperscript{G}, Amazon Rekognition \textsuperscript{G} e Amazon Comprehend \textsuperscript{G} in modo da assegnare a ciascun dato un valore quantitativo relativo alla sua positività. Bisognerà inoltre riconoscere tutti i dati raccolti che non sono assolutamente rilevanti, i quali verranno scartati. Particolare attenzione andrà prestata ai commenti, riguardo i quali è richiesto di svolgere un’analisi preliminare al fine di capire se è davvero utile includerli nella raccolta dei dati.

\subsubsection{Realizzazione di un ranking}
Sarà compito del team progettare un sistema di ranking \textsuperscript{G} dei locali presenti nella guida, in particolare bisognerà decidere che peso dare a ciascun tipo di contenuto e fare attenzione a considerare anche dati che apparentemente non sembrano significativi ma che correlati con altri possono assumere un significato ben preciso, come ad esempio due foto postate in successione dallo stesso profilo \textsuperscript{G} in cui nella prima si vede un ristorante e nella seconda una persona felice. Sarà inoltre necessario decidere in base a che criteri strutturare il ranking \textsuperscript{G}, i quali potrebbero essere la regione in cui si trova il locale, il tipo di cucina o altri. Per questa fase non sono state stabilite delle linee guida molto rigide e viene lasciata molta libertà al team per l’implementazione.

\subsubsection{Interfaccia utente}
L’utente \textsuperscript{G} potrà interfacciarsi con la guida in due modi differenti ovvero tramite webapp \textsuperscript{G} o con l’uso di una mobile app \textsuperscript{G}, entrambi gli applicativi dovranno fornire le stesse funzionalità principali quali la visualizzazione del ranking \textsuperscript{G}, la ricerca di uno specifico locale per la consultazione delle informazioni principali e del suo punteggio, la possibilità registrarsi e poter suggerire nuovi profili social da cui andare a effettuare il crawling \textsuperscript{G} dei dati.
Anche per questa fase viene concessa molta libertà di sviluppo al team senza però tralasciare le funzionalità principali.

\subsection{Caratteristiche degli Utenti}

La piattaforma \textsuperscript{G} offre la possibilità sia ad utenti generici che registrati di poter consultare la guida tramite app per smartphone o webapp \textsuperscript{G}. Ciascuna tipologia di utente \textsuperscript{G} ha diverse funzionalità.

\subsubsection{Utente Generico}

Con il termine utente generico \textsuperscript{G} ci si riferisce ad una qualsiasi persona non registrata nel sistema e che può sfruttare le funzionalità di base offerte dalla piattaforma \textsuperscript{G}, ossia:

\begin{itemize}
  \item Visualizzare la lista di locali \textsuperscript{G} data una specifica posizione;
  \item Cercare un determinato locale attraverso la funzionalità di ricerca (impostando anche dei filtri \textsuperscript{G});
  \item Visualizzare un ranking \textsuperscript{G} di locali per zona o in base alla propria posizione;
  \item Consultare il profilo \textsuperscript{G} social di un locale e conoscere gli orari di apertura.
\end{itemize}

\subsubsection{Utente Registrato}

Invece, con il termine utente registrato \textsuperscript{G} ci si riferisce ad una persona registrata nella piattaforma \textsuperscript{G}, che oltre a sfruttare le funzionalità dell’utente generico \textsuperscript{G} può anche:

\begin{itemize}
  \item Riempire il modulo di registrazione ed autenticarsi \textsuperscript{G};
  \item Suggerire nuovi utenti \textsuperscript{G} che potrebbero iscriversi alla piattaforma \textsuperscript{G};
  \item Creare liste personalizzate private \textsuperscript{G} con i locali preferiti \textsuperscript{G} e da visitare in futuro.
\end{itemize}

\subsection{Piattaforme di Esecuzione}

\subsubsection{Web}
Insieme di pagine web accessibili tramite i browser più utilizzati, come Google Chrome, Firefox, Safari, Microsoft Edge o Opera. I browser devono essere aggiornati nelle loro versioni più recenti.

\subsubsection{Android}
L'applicazione verrà installata tramite un file con l’estensione apk e sarà eseguibile nel sistema operativo android per la piattaforma \textsuperscript{G} mobile.

\subsection{Obblighi di Progettazione}
\begin{itemize}
  \item Creare degli API \textsuperscript{G} per il crawling \textsuperscript{G};
  \item Valutare strategie Voice to Text se le informazioni quali testi, commenti e tag non sono sufficienti:
  \item Realizzare una webapp \textsuperscript{G} per offrire il servizio agli utenti che utilizzano un browser;
  \item Realizzare una mobile app \textsuperscript{G} rendendo disponibile la guida ai dispositivi android;
  \item Utilizzo delle seguenti tecnologie: 
  \begin{itemize}
    \item NodeJS \textsuperscript{G} per lo sviluppo degli API \textsuperscript{G} per supportare l’applicativo;
    \item AWS \textsuperscript{G} per l'analisi dei dati raccolti tramite crawling \textsuperscript{G};
    \item Neptune per il database dove memorizzare i dati raccolti;
    \item Utilizzo di un linguaggio nativo per lo sviluppo della mobile app \textsuperscript{G}.
  \end{itemize}
  \item Utilizzo di un'architettura basata a microservizi \textsuperscript{G}.

\end{itemize}