\section{Descrizione Generale}

\subsection{Caratteristiche del Prodotto}

A seguito della presentazione del capitolato\textsuperscript{G} e dei primi incontri fatti con il proponente\textsuperscript{G} è emerso che il prodotto che andremo a realizzare dovrà avere le seguenti caratteristiche:

\subsubsection{Crawling dei dati da Instagram e TikTok}
Il team dovrà effettuare un’analisi mediante le API\textsuperscript{G} social di Instagram e TikTok per capire se esse sono sufficienti a raccogliere e analizzare le informazioni, in caso contrario si dovrà discutere con il proponente sui limiti dei social utilizzati. In ogni caso si dovranno analizzare esclusivamente profili pubblici ed escogitare una strategia per evitare di essere inseriti nelle black-list\textsuperscript{G} di Instagram e TikTok a causa di queste operazioni.

\subsubsection{Traduzione dei dati ottenuti}
Nella fase di crawling verranno raccolti dati di diverso tipo quali foto, video, stories, post e commenti. Questi dati andranno poi “tradotti” utilizzando diversi servizi come ad esempio Amazon Rekognition\textsuperscript{G} e Amazon Comprehend\textsuperscript{G} in modo da assegnare a ciascun dato un valore quantitativo relativo alla sua positività. Bisognerà inoltre riconoscere tutti i dati raccolti che non sono assolutamente rilevanti, i quali verranno scartati. Particolare attenzione andrà prestata ai commenti, riguardo i quali è richiesto di svolgere un’analisi preliminare al fine di capire se è davvero utile includerli nella raccolta dei dati.

\subsubsection{Realizzazione di un ranking}
Sarà compito del team progettare un sistema di ranking\textsuperscript{G} dei locali presenti nella guida, in particolare bisognerà decidere che peso dare a ciascun tipo di contenuto e fare attenzione a considerare anche dati che apparentemente non sembrano significativi ma che correlati con altri possono assumere un significato ben preciso, come ad esempio due foto postate in successione dallo stesso profilo\textsuperscript{G} in cui nella prima si vede un ristorante e nella seconda una persona felice. Sarà inoltre necessario decidere in base a che criteri strutturare il ranking, i quali potrebbero essere la regione in cui si trova il locale, il tipo di cucina o altri. Per questa fase non sono state stabilite delle linee guida molto rigide e viene lasciata molta libertà al team per l’implementazione.

\subsubsection{Interfaccia utente}
L’utente\textsuperscript{G} potrà interfacciarsi con la guida tramite una WebApp\textsuperscript{G} la quale dovrà fornire alcune funzionalità principali quali la visualizzazione del ranking, la ricerca di uno specifico locale per la consultazione delle informazioni principali e del suo punteggio, la possibilità di registrarsi e poter suggerire nuovi profili\textsuperscript{G} social da cui andare a effettuare il crawling dei dati.

\subsection{Caratteristiche degli Utenti}

La piattaforma\textsuperscript{G} offre la possibilità di poter consultare la guida tramite WebApp sia ad utenti non autenticati che ad utenti autenticati\textsuperscript{G}. Ciascuna tipologia di utente potrà avere accesso a funzionalità differenti.

\subsubsection{Utente Non Autenticato}

Con il termine utente non autenticato\textsuperscript{G} ci si riferisce ad una qualsiasi persona non autenticata nel sistema, che può sfruttare le funzionalità di base offerte dalla piattaforma, ossia:

\begin{itemize}
  \item Visualizzare il ranking dei locali\textsuperscript{G} applicando specifici filtri\textsuperscript{G};
  \item Cercare un determinato locale attraverso la funzionalità di ricerca;
  \item Consultare il profilo social di un locale, conoscerne gli orari di apertura ed il numero di telefono, oltre a visitare il sito web (nel caso esista);
  \item Registrarsi nella piattaforma per sfruttare delle funzionalità aggiuntive.
\end{itemize}

\subsubsection{Utente Autenticato}

Invece, con il termine “utente autenticato” ci si riferisce ad una persona registrata nel database e che ha effettuato l'accesso nella piattaforma, la quale, oltre a sfruttare le funzionalità dell’utente generico\textsuperscript{G}, può anche:

\begin{itemize}
  \item Suggerire nuovi profili dai quali andare ad effettuare il crawling dei dati per realizzare il ranking\textsuperscript{G};
  \item Creare liste personalizzate private\textsuperscript{G} con i locali preferiti\textsuperscript{G};
  \item Gestire il profilo personale (modificare la password, collegare l'account Instagram e/o TikTok).
\end{itemize}

% \subsubsection{WebApp}

% Tra le caratteristiche del prodotto, troviamo anche la realizzazione di una WebApp che permetterà all'utente finale di poter fruire dei contenuti presenti nella piattaforma.

% Una WebApp\textsuperscript{G}{} è un'applicazione formata da più pagine web accessibili tramite un browser (ad esempio: Google Chrome, Firefox, Safari, Microsoft Edge ed Opera). Al fine di ottenere la miglior esperienza utente possibile, è consigliabile consultare la piattaforma usando un browser (tra quelli appena elencati) nella sua versione più aggiornata.

\subsection{Obblighi di Progettazione}

\begin{itemize}
  \item Creare e sfruttare delle API\textsuperscript{G} per il crawling\textsuperscript{G} dei dati se quelle già esistenti non sono sufficienti allo scopo del prodotto;
  \item Valutare strategie Voice to Text se le informazioni quali testi, commenti e tag non sono sufficienti;
  \item Realizzare una WebApp\textsuperscript{G} con design responsive per permettere agli utenti di consultare i contenuti presenti nella piattaforma ed usufruire di tutte le sue funzionalità;
  \item Utilizzo di un'architettura a microservizi\textsuperscript{G}{}.
\end{itemize}

\subsubsection{Tecnologie utilizzate}

Per sviluppare la piattaforma verranno utilizzare le seguenti tecnologie:
\begin{itemize}
    \item \textbf{Python\textsuperscript{G}} per la creazione e l'utilizzo dai crawler\textsuperscript{G} che estrapolerà i dati dai contenuti social;
    \item \textbf{HTML5\textsuperscript{G}}: per creare la struttura dell'interfaccia utente;
    \item \textbf{CSS3\textsuperscript{G}}: per lo stile dell'interfaccia utente;
    \item \textbf{React\textsuperscript{G}} per la creazione dell'interfaccia\textsuperscript{G} utente;
    \item \textbf{AWS\textsuperscript{G}} tramite i seguenti servizi:
    \begin{itemize}
    	\item \textbf{Amazon Rekognition} per analizzare e riconoscere attributi\textsuperscript{G}{}, oggetti e testi dei contenuti multimediali;
    	\item \textbf{Amazon Comprehend} per analizzare e riconoscere informazioni presenti nei contenuti testuali;
    	\item \textbf{AWS Lambda\textsuperscript{G}} per il crawler, l'inserimento e l'estrazione dei dati nel/dal database;
    	\item \textbf{Amazon Aurora Serverless\textsuperscript{G}} con \textbf{MySQL} come database relazionale;
    	\item \textbf{Amazon S3\textsuperscript{G}} per l'archiviazione di documenti;
    	\item \textbf{API Gateway\textsuperscript{G}} per consentire alla WebApp di visualizzare i contenuti;
    	\item \textbf{Amazon Amplify\textsuperscript{G}} che consiste in un set di strumenti che permettono di realizzare e far funzionare correttamente la WebApp.
	\end{itemize}
  \end{itemize}