%\section{Limiti dei social utilizzati}
Come richiesto dal proponente, il team dovrà realizzare un'analisi sui limiti dei social utilizzati al fine di capire la realizzabilità di determinati requisiti.

\subsection{Instagram}
\subsubsection{Limiti}
Durante lo sviluppo, il team ha riscontrato alcuni problemi che evidenziano i limiti nell'utilizzo di un crawler su Instagram.
\begin{itemize}
    \item Per effettuare crawling, è necessario utilizzare un account instagram;
    \item Effettuare troppe richieste in poco tempo porta ad alcuni warning che impediscono l'esecuzione del programma;
    \item A seguito dei warning, Instagram può chiedere di inserire un codice di conferma ricevuto via email per poter continuare ad utilizzare l'account con cui si sta facendo crawling. Questo comporta l'intervento di una persona per sbloccare l'account;
    \item Il susseguirsi degli eventi escritti ai punti precedenti porta al ban definitivo dell'account Instagram;
    \item Dato che gli indirizzi IP di aws sono in gran parte presenti nella black-list di Instagram, effettuare le operazioni di crawling da una aws lambda è quasi impossibile perché porta al ban dell'account in tempi brevissimi.
\end{itemize}

\subsubsection{Soluzioni}
Il team ha individuato le seguenti soluzioni ai problemi descritti in precedenza.
\begin{itemize}
    \item Creare innumerevoli account in modo da averne sempre a disposizione a seguito di un ban, gli account per funzionare a dovere dovranno essere stati creati con qualche settimana di anticipo dato che Instagram tende a bloccare più facilemnte un account appena creato;
    \item Introdurre delle sleep di un tempo random da 5 a 15 secondi tra una richiesta e l'altra, in modo da emulare il comportamento di un umano;
    \item Fare girare il crawler esclusivamente in locale data l'impossibilità di farlo su una aws lambda (sarebbe possibile effettuare il crawling su una aws lambda, ma solo utilizzando dei proxy domestici che il team non ha avuto a disposizione);
    \item Provare ad utilizzare una VPN per effettuare il crawling, soluzione che ha non ha avuto l'effetto desiderato, ma che ha portato al rapido ban di un account;
    \item Salvare a database i dati relativi alle location dei post in modo tale da poter fare un controllo nel database prima di effettuare un richiesta ad Instagram.
\end{itemize}

\subsubsection{Conclusioni}
Nonostante i limiti descritti ai punti precedenti, il team è riuscito a raccogliere i seguenti dati analizzando i post di 340 utenti:
\begin{itemize}
    \item più di 5000 location delle quali più della metà relative a ristoranti;
    \item più di 5000 post relativi a ristoranti;
    \item più di 13000 immagini relative a ristoranti.
\end{itemize}
I numeri ottenuti sono limitati dalle politiche anti-crawling di Instagram, ma il codice realizzato sarebbbe comunque in grado di funzionare su più ampia scala con l'aggiunta di alcune risorse come:
\begin{itemize}
    \item sostituzione delle chiamate alle API private di instagram per ottenere dettagli sul geotag di un post con le API di google maps;
    \item esecuzione parallela di più istanze del crawler, oguna con un account Instagram e un proxy domestico differente.
\end{itemize}

\subsection{TikTok}
Per quanto riguarda il crawling con TikTok, il team ha riscontrato numerosi problemi durante la fase di sviluppo del \textit{Proof of Concept}, i quali hanno portato a ritenere queste operazioni come non fattibili a causa dei seguenti motivi:
\begin{itemize}
    \item Tiktok a differenza di Instagram è molto più severo nei confronti di chi effettua crawling;
    \item L'unico crawler open source funzionante individuato dal team, dopo pochissime operazioni di crawling porta al ban dell'ip\textsuperscript{G}. Nonostante ciò il team è riuscito ad aggirare questo problema utilizzando dei proxy\textsuperscript{G} gratuiti e creando un algoritmo che testasse la validità di questi proxy;
    \item Le alternative valide al crawler utilizzato sono tutte a pagamento e quindi non utilizzabili all'interno di questo progetto;
    \item Il crawler utilizzato dal team non può funzionare all'interno di una lambda dato che fa uso del framework playwright\textsuperscript{G}, per questo problema dopo lunghe ricerche il team non è riuscito a trovare una soluzione attuabile;
    \item L'idea di creare un nostro crawler è troppo dispendiosa se messa a confronto con le ore lavorative a disposizione, dato che dopo un po' di tentativi il team si è reso conto che un semplice parser html\textsuperscript{G} non sarà sufficiente a causa dell'autenticazione (la quale necessita la risoluzione di un captcha\textsuperscript{G}) e dell'utilizzo da parte di TikTok di un sistema di signed request\textsuperscript{G};
    \item Per quanto riguarda la versione desktop di TikTok, non è possibile guardare i profili seguiti da un utente. Questo rende irrealizzabile il caso d'uso UCW6;
    \item Difficilmente si riuscirà ad associare un locale ad un post, dato che TikTok non offre la possibilità di mettere un geotag\textsuperscript{G}. Anche ricavando un indirizzo dal testo del post risulterebbe troppo oneroso associare quell'indirizzo ad uno specifico locale, infatti l'unica soluzione pensata dal team è quella di utilizzare le API di Google Maps, le quali offrono un numero di chiamate troppo piccolo nella versione gratuita;
	
	    
    \item A tutto questo si aggiunge il fatto che TikTok non viene utilizzato spesso per caricare contenuti relativi ai ristoranti, c’è quindi il rischio di impiegare molte risorse per ricavare pochissimi dati utili.
\end{itemize}
Dopo aver presentato queste considerazioni relative al rapporto costi/benefici al proponente, ci è stato comunicato che l’analisi svolta risulta sufficiente. A tal proposito,  questa eventualità era già stata presa in considerazione dal proponente stesso, il quale ha confermato al team che il prodotto finale sfrutterà solo Instagram, mentre la componente TikTok potrà essere esclusa.