\subsection{API}
Acronimo di “\textit{Application Programming Interface}”, sono un’interfaccia software. A differenza di un’interfaccia utente, che connette il computer con una persona, le API connettono computer e porzioni di software tra loro.

\subsection{Applicazione}
Software che può essere eseguito sia su mobile che su desktop, permette all’utente di visualizzare dei contenuti e di interagire con essi.

\subsection{Amazon Comprehend}
È un servizio di elaborazione del linguaggio naturale (NLP – “\textit{Natural Language Processing}”), che sfrutta il machine learning per ottenere informazioni scritte o parlate in una lingua. 

\subsection{Amazon Neptune}
È un servizio di database a grafo rapido, affidabile e completamente gestito, che semplifica la creazione e l’esecuzione di applicazioni. 

\subsection{Amazon Rekognition}
Servizio di cloud computing che offre capacità di visione artificiale (CV – “\textit{Computer Vision}”) pre-addestrate e personalizzabili per estrarre informazioni dettagliate da immagini e video. 

\subsection{Amazon Transcribe}

Servizio di cloud computing che utilizza un processo di deep learning, chiamato riconoscimento vocale automatico (ASR – \textit{Automatic Speech Recognition}), per convertire voce in testo in modo rapido e preciso.

\subsection{AWS}
È l’acronimo di “\textit{Amazon Web Services}” ed è un “pacchetto” di servizi di cloud computing offerti da Amazon. 

\clearpage 