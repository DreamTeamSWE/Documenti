\section{Introduzione}

\subsection{Scopo del Documento}
Lo scopo di questo documento è di definire le norme, le convenzioni e le procedure adottate da tutti i membri di \textit{DreamTeam}, in modo da poter definire un metodo di lavoro comune.
Per raggiungere questo scopo ogni membro è tenuto a visionare periodicamente il documento e a rispettare tutte le norme in esso presenti. 
Per la redazione viene adottata una filosofia incrementale, quindi il documento allo stato attuale è incompleto e le norme saranno definite passo passo partendo dalle più urgenti, con l'aspettativa di avere un processo normato prima del suo avvio, considerando che, in generale, ogni norma può essere soggetta a cambiamenti.


\subsection{Scopo del Prodotto}
L’obiettivo di Sweeat e dell’azienda \zd è la creazione di un sistema software costituito da una Webapp. Lo scopo del prodotto è di fornire all’utente una guida dei locali gastronomici sfruttando i numerosi contenuti digitali creati dagli utenti sulle principali piattaforme social (Instagram e TikTok).
In questo modo, è possibile realizzare una classifica basata sulle impressioni e reazioni di chiunque usufruisca dei servizi dei locali, non solo da professionisti ed esperti del settore.


\subsection{Glossario}
Per evitare ambiguità relative alle terminologie utilizzate è stato creato un documento denominato “\textit{Glossario}”. Questo documento comprende tutti i termini tecnici scelti dai membri del gruppo e utilizzati nei vari documenti con le relative definizioni. Tutti i termini inclusi in questo glossario, vengono segnalati all'interno del documento con l'apice \textsuperscript{G} accanto alla parola.

\subsection{Riferimenti}
\subsubsection{Normativi}

\subsubsection{Informativi}