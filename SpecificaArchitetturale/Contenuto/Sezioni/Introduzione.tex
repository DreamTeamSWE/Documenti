\section{Introduzione}

\subsection{Scopo del Documento}
Lo scopo del presente documento è quello di descrivere in maniera coesa, coerente ed esaustiva le caratteristiche architetturali del prodotto \textit{Sweeat} sviluppato dal gruppo \textit{DreamTeam}.


\subsection{Scopo del Prodotto}
L’obiettivo di Sweeat e dell’azienda \zd è la creazione di un sistema software costituito da una Webapp. Lo scopo del prodotto è di fornire all’utente una guida dei locali gastronomici sfruttando i numerosi contenuti digitali creati dagli utenti sulle principali piattaforme social (Instagram e TikTok).
In questo modo, è possibile realizzare una classifica basata sulle impressioni e reazioni di chiunque usufruisca dei servizi dei locali, non solo da professionisti ed esperti del settore.


\subsection{Glossario}
Per evitare ambiguità relative alle terminologie utilizzate è stato creato un documento denominato “\textit{Glossario}”. Questo documento comprende tutti i termini tecnici scelti dai membri del gruppo e utilizzati nei vari documenti con le relative definizioni. Tutti i termini inclusi in questo glossario, vengono segnalati all'interno del documento con l'apice \textsuperscript{G} accanto alla parola.

\subsection{Riferimenti}
\subsubsection{Normativi}
\begin{itemize}
\item \textit{Analisi dei Requisiti v}
\item Presentazione del capitolato - Zero12 Progettazione e sviluppo di una Social guida Michelin:
\newline \mylink{https://www.math.unipd.it/~tullio/IS-1/2021/Progetto/C4p.pdf}
\end{itemize}
\subsubsection{Informativi}
\begin{itemize}
\item Regolamento del progetto didattico - Materiale didattico del corso di Ingegneria del Software:
\newline \mylink{https://www.math.unipd.it/~tullio/IS-1/2021/Dispense/PD2.pdf}
\begin{itemize}
\item \textit{Slides 12, 17.}
\end{itemize}
\item Diagrammi delle classi - Materiale didattico del corso di Ingegneria del Software:
\newline \mylink{https://www.math.unipd.it/\%7Ercardin/swea/2021/Diagrammi\%20delle\%20Classi_4x4.pdf}
\item Design Pattern Strutturali - Materiale didattico del corso di Ingegneria del Software:
\newline \mylink{https://www.math.unipd.it/\%7Ercardin/swea/2021/Design\%20Pattern\%20Strutturali_4x4.pdf}
\begin{itemize}
\item \textit{Slides 4-13, 25-34.}
\end{itemize}
\item Design Pattern Comportamentali - Materiale didattico del corso di Ingegneria del Software:
\newline \mylink{https://www.math.unipd.it/\%7Ercardin/swea/2021/Design\%20Pattern\%20Comportamentali_4x4.pdf}
\begin{itemize}
\item \textit{Slides 32-40.}
\end{itemize}
\item Model-View Patterns - Materiale didattico del corso di Ingegneria del Software:
\newline \mylink{https://www.math.unipd.it/~rcardin/sweb/2022/L02.pdf}
\begin{itemize}
\item \textit{Slides 31-40.}
\end{itemize}
\item Static Factory - Cleaner Code with Static Factory Methods:
\newline \mylink{https://stackify.com/static-factory-methods/}
\end{itemize}