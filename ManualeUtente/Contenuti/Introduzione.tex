% Introduzione Manuale Utente

\subsection{Scopo del documento}

Lo scopo di questo documento è quello di illustrare le funzionalità offerte dalla WebApp Sweeat e, per ciascuna di esse, dare una descrizione dettagliata sul relativo funzionamento.
In questo modo, qualora l’utente cercasse una guida su come sfruttare correttamente la WebApp e relative funzionalità, potrà farlo affidandosi a questo manuale. 

\subsection{Scopo del prodotto}

L’obiettivo di Sweeat e dell’azienda Zero12 è la creazione di un sistema software costituito da una Webapp. Lo scopo del prodotto è di fornire all’utente una guida dei locali gastronomici sfruttando i numerosi contenuti digitali creati dagli utenti sulle principali piattaforme social (Instagram e TikTok). In questo modo, è possibile realizzare una classifica basata sulle impressioni e reazioni di chiunque usufruisca dei servizi dei locali, non solo da professionisti ed esperti del settore.

\subsection{Glossario}

Per evitare ambiguità relative alle terminologie utilizzate è stato creato un documento denominato “\textit{Glossario v1.1.0}”. Questo documento comprende tutti i termini tecnici scelti dai membri del gruppo ed utilizzati nei vari documenti con le relative definizioni. Tutti i termini inclusi in questo glossario, vengono segnalati all’interno del documento con l’apice\glo accanto alla parola. 

\subsection{Riferimenti}
\begin{itemize}
    \item Regolamento del progetto didattico - Materiale didattico del corso di Ingegneria del Software: \newline\mylink{https://www.math.unipd.it/~tullio/IS-1/2020/Dispense/P1.pdf};
    \item AWS Cognito: \newline \mylink{https://aws.amazon.com/it/cognito/};
    \item Griglia dei voti:\newline \mylink{https://www.liceomorgagni.edu.it/sites/www.liceomorgagni.it/files/dipartimentp_griglia_valutazione_1.pdf}.
\end{itemize}