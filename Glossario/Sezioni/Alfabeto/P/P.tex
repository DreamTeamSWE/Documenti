\section{P}

\subsection{Piattaforma} Si intende il software Sweeat, che mostra la guida dei locali gastronomici presenti nel database e permette di eseguire operazioni (come la registrazione, la creazione di liste personalizzate etc.).

\subsection{Processo}
Insieme di attività correlate e coese che trasformano input (bisogni) in output (prodotti) secondo regole prestabilite, consumando risorse. Deve essere sistematico, disciplinato e quantificabile.

\subsection{Product Baseline}
Rappresenta la baseline successiva alla Technology Baseline\textsuperscript{G}. Definisce diagrammi delle classi e di attività e eventuali design pattern. 

\subsection{Profilo} Account registrato su un social network. Nel caso di Sweeat, quando si parla di “profilo”, si intende un account su Instagram e TikTok, da non confondere col termine utente.




\subsection{Proponente}
Ente o azienda che propone un capitolato d’appalto\textsuperscript{G} per un progetto.

\subsection{Pull Request}
Nel un sistema di versionamento integrato da GitHub\textsuperscript{G} è l'operazione alias al merge\textsuperscript{G}.

\clearpage