\section{P}

\subsection{Parser html}
Software per l'analisi automatizzata di pagine HTML. Utile per estrapolare informazioni da una pagina web.

\subsection{PEP 8}
Abbreviazione di Python Enterprise Proposal, è uno standard che offre una serie di linee guida per la scrittura di codice Python leggibile e mantenibile. 

\subsection{Piattaforma} Si intende il software Sweeat, che mostra la guida dei locali gastronomici presenti nel database e permette di eseguire operazioni (come la registrazione, la creazione di liste personalizzate etc.).

\subsection{Politiche anti-crawling}
In relazione a Instagram e TikTok, tutta una serie di azioni e funzionalità che limitano o impediscono totalmente le attività di web crawling.

\subsection{Processo}
Insieme di attività correlate e coese che trasformano input (bisogni) in output (prodotti) secondo regole prestabilite, consumando risorse. Deve essere sistematico, disciplinato e quantificabile.

\subsection{Product Baseline}
Rappresenta la baseline successiva alla Technology Baseline\textsuperscript{G}. Definisce diagrammi delle classi e di attività e eventuali design pattern. 

\subsection{Profilo} Account registrato su un social network. Nel caso di Sweeat, quando si parla di “profilo”, si intende un account su Instagram e TikTok, da non confondere col termine utente.

\subsection{Proof of Concept o PoC}
È un prototipo software che ha lo scopo di dimostrare e verificare la fattibilità dell'idea proposta dal proponente.

\subsection{Proponente}
Ente o azienda che propone un capitolato d'appalto\textsuperscript{G} per un progetto.

\subsection{Proxy}
Indica un tipo di server che funge da intermediario per le richieste da parte dei clienti alla ricerca di risorse su altri server, disaccoppiando l'accesso al web dal browser. Un client si connette al server proxy, richiedendo qualche servizio (ad esempio un file, una pagina web o qualsiasi altra risorsa disponibile su un altro server), e quest'ultimo valuta ed esegue la richiesta in modo da semplificare e gestire la sua complessità. Viene utilizzato per le attività di crawling poichè permette di evitare il ban degli IP, migliorare la sicurezza, accedere a contenuti limitati per zona e permette di aumentare i volumi di dati raccolti. 

\subsection{Pull Request}
Nel sistema di versionamento integrato da GitHub\textsuperscript{G} è l'operazione alias al merge\textsuperscript{G}.

\subsection{Python} 
È un linguaggio di programmazione ad alto livello, orientato agli oggetti, utilizzato per diversi usi, tra cui sviluppare applicazioni distribuite, scripting, computazione numerica e system testing. 


\clearpage