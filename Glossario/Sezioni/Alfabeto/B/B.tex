\section{B}


\subsection{Baseline} 
Equivale al concetto di campo base (ad esempio per delle escursioni in montagna). Rappresenta un risultato concreto associato al concetto di milestone\textsuperscript{G} che risponde in maniera affermativa alla domanda: “Si può fare?”, evitando situazioni di rischio.


\subsection{Black-list} Ossia “lista nera”, è una lista che viene utilizzata per controllare gli accessi effettuati ad una certa risorsa (un sito web piuttosto che una rete). In genere, chi finisce nella “black-list” ha delle funzionalità limitate rispetto gli utenti della lista bianca (white-list) ed è possibile che un utente presente nella black-list non abbia accesso ad alcun servizio offerto da un’ente o un’azienda (ad esempio, è possibile che un utente che finisce in black-list non possa visualizzare i contenuti pubblicati sulla piattaforma social Instagram). 



\subsection{Branch}
In un sistema di versionamento, indica un ramo di sviluppo, quindi una ramificazione dal contenuto del repository considerato principale (detto main branch). 


\clearpage