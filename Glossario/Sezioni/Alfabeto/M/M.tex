\section{M}


\subsection{Media}
Si intende una pubblicazione sul un social media, nel caso di Instagram rappresenta un singolo post di un utente.

\subsection{Meet, Google}
Google Meet è un'applicazione di teleconferenza sviluppata da Google. 

\subsection{Merge}
In un sistema di versionamento, indica l'operazione che permette di unire due branch\textsuperscript{G} di sviluppo separati.

\subsection{Merge Conflicts}
In un sistema di versionamento, indica la situazione in cui un merge\textsuperscript{G} automatico è impossibile dato che ci sono delle modifiche nelle stesse righe che sono in contrasto tra di loro. Richiede un intervento manuale  per risolvere il conflitto.

\subsection{Microservizi} I microservizi sono un approccio per sviluppare e organizzare l’architettura dei software, i quali sono composti da servizi indipendenti di piccole dimensioni e che comunicano tra loro tramite API ben definite. Questi servizi sono controllati da piccoli team autonomi. \\
Le architetture dei microservizi permettono di scalare e sviluppare le applicazioni in modo più rapido e semplice, permettendo di promuovere l’innovazione e accelerare il gap tra creazione e produzione di nuove funzionalità.

\subsection{Milestone}
Indica importanti traguardi temporali intermedi nello svolgimento del progetto. Si traduce dall'inglese in pietra miliare. 

\subsection{Mobile App} Applicazione che può essere eseguita in un dispositivo mobile (ad esempio uno smartphone Android o iOS).

\clearpage