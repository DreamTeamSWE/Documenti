\section{C}

\subsection{CamelCase}
Convenzione per la scrittura di nomi composti da multiple parole, mettendole una di seguito all'altra con l'iniziale di ogni parola maiuscola. 

\subsection{Capitolato (d'appalto)}
Documento tecnico redatto dal cliente in cui vengono specificati i vincoli contrattuali(prezzo e scadenze) per lo sviluppo di un determinato prodotto software.

\subsection{Ciclo di vita}
Il ciclo di vita di un software presenta un'organizzazione in fasi che partono dal concepimento del prodotto software al suo ritiro. La fase più lunga e corposa è quella di manutenzione.



\subsection{Cloud}
In informatica indica una serie di tecnologie che permettono l’elaborazione, l’archiviazione e la memorizzazione di dati, grazie all’utilizzo di risorse hardware e software distribuite nella rete.



\subsection{Commit}
In un sistema di versionamento, rappresenta l'operazione che invia le ultime modifiche al codice all'interno di un repository. Generalmente è sempre data la possibilità di tornare allo stato del codice prima di ogni commit.


\subsection{Committenti}
Persone o gruppo di persone incaricato ad individuare il prodotto da commissionare al fornitore\textsuperscript{G}. Viene utilizzato anche il plurale committenti.



\subsection{Crawler} Detto anche “web crawler”, “spider” o “robot”, è un software che analizza i contenuti presenti nel web o in un database, in modo metodico ed automatizzato. In genere, si tratta di un bot che, una volta impostato ed in maniera del tutto autonoma, acquisisce dei dati utili per integrare o realizzare qualcosa (come un prodotto software). 

\subsection{Crawling} Significa “fare scansione”, è il processo di raccolta delle informazioni e dei contenuti dal web o all’interno di un database e tale operazione viene effettuata da un bot (o crawler). 








\clearpage