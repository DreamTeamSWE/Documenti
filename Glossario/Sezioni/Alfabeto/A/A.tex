\section{A}

\subsection{Amplify, Amazon}
Amazon Amplify consiste in un set di strumenti progettati per sviluppare WebApp ed integrarle con l'ecosistema di AWS. 

\subsection{API}
Acronimo di “\textit{Application Programming Interface}”, sono un’interfaccia software. A differenza di un’interfaccia utente, che connette il computer con una persona, le API connettono computer e porzioni di software tra loro.

\subsection{API Gateway, Amazon}
Servizio per la creazione, la pubblicazione, la gestione, il monitoraggio e la protezione di API REST, HTTP e WebSocket a qualsiasi livello.

\subsection{Aurora Serverless, Amazon}
È una configurazione a dimensionamento automatico on-demand per Amazon Aurora. Questa funzionalità si avvia, si arresta e ridimensiona automaticamente la capacità in base alle esigenze dell'applicazione. Per il suo funzionamento, è sufficiente creare un endpoint del database, specificare l'eventuale intervallo di capacità desiderato e connettere le applicazioni desiderate.  

%\subsection{Amazon Cognito}
%Servizio che offre strumenti di registrazione degli utenti, accesso e controllo degli accessi alle WebApp e per dispositivi mobili, in modo rapido e semplice.

%\subsection{Amazon Neptune}
%È un servizio di database a grafo rapido, affidabile e completamente gestito, che semplifica la creazione e l’esecuzione di applicazioni. 

%\subsection{Amazon Transcribe}
%Servizio di cloud computing che utilizza un processo di deep learning, chiamato riconoscimento vocale automatico (ASR – \textit{Automatic Speech Recognition}), per convertire voce in testo in modo rapido e preciso.

\subsection{AWS}
È l’acronimo di “\textit{Amazon Web Services}” ed è un “pacchetto” di servizi di cloud computing offerti da Amazon. 

\subsection{AWS CLI}
Servizio AWS che offre uno strumento open source per interazione con i servizi AWS utilizzando i comandi nella shell a riga di comando

\clearpage 