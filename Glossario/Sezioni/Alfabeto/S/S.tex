\section{S}

\subsection{S3, Amazon}
È un servizio di archiviazione di oggetti, che offre scalabilità, sicurezza e prestazioni d'avanguardia. 

\subsection{Serverless}
È un framework utilizzabile da linea di comando che permette di semplificare il deploy del codice per AWS Lambda e delle infrastrutture di cloud. Supporta una moltitudine di linguaggi tra cui: Node.js, Typescript, Python, Go e Java.

\subsection{Signed Request}
Un comune sistema anti crawling, dove ogni richiesta HTTP viene “firmata” tramite un algoritmo privato presente nel server. La firma è univoca e ogni richiesta non conforme viene bloccata. È, quindi, necessario l'utilizzo di un browser per accedere alla piattaforma, così da poter generare la firma necessaria per validare la richiesta.

\subsection{Slack}
Applicazione di messaggistica per ambienti enterprise.

\subsection{SQS, Amazon}
Amazon Simple Queue Service (SQS) è un servizio di accodamento messaggi completamente gestito, che consente la separazione e la scalabilità di microservizi, sistemi distribuiti e applicazioni senza server. Tramite SQS è possibile inviare, archiviare e ricevere messaggi tra componenti software a qualsiasi volume, senza perdere messaggi o richiedere la disponibilità di altri servizi.
SQS offre due tipi di code di messaggi. Le code standard offrono throughput massimo, ordinamento semplificato e distribuzione di tipo at-least-once. Le code FIFO di SQS sono progettate per garantire che i messaggi vengano elaborati esattamente una sola volta, nell'ordine in cui sono inviati.

\subsection{Stakeholder}
All'interno di un progetto rappresenta un “portatore di interessi” (investitore, cliente, ecc.) o in generale chiunque sia direttamente o indirettamente coinvolto nel progetto di realizzazione del prodotto.

\clearpage