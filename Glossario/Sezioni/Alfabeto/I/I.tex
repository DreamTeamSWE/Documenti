\section{I}

\subsection{Incrementale, Modello di sviluppo} Il modello di sviluppo incrementale vede il progetto come una successione di rilasci (prima interni e poi esterni) che approssimino sempre più da vicino, quindi per aggiunte successive, il prodotto software atteso. \\ 
Questo modello combina alcuni aspetti del modello a cascata applicati iterativamente. Prevede l’applicazione, scalata nel tempo, di più sequenze complete del modello a cascata. Il modello è stato concepito con l’idea che la prima iterazione porti come risultato la definizione di un prodotto base, una sorta di scheletro che soddisfi solamente i requisiti fondamentali del committente. Questo aiuta il cliente ad enunciare subito le funzionalità principali del software, per poi aggiungere dettagli in seguito (piuttosto che definire subito tutte le caratteristiche del prodotto finito). In quest’ottica, le iterazioni successive servono a raccogliere requisiti sempre più dettagliati e a raffinare maggiormente le funzionalità del prodotto.

\subsection{Indice di Gulpease} Indica la leggibilità di un testo tarato sulla lingua italiana. Rispetto ad altri ha il vantaggio di utilizzare la lunghezza delle parole in lettere anziché in sillabe, semplificandone il calcolo automatico. 

\subsection{Interfaccia utente} Una rappresentazione visiva in grado di mostrare la visuale grafica dell’applicazione, che permette di fare interagire l’applicazione stessa con l’utente.

\subsection{Issue}
Feature offerta dal servizio GitHub\textsuperscript{G} per il tracciamento di idee, miglioramenti, compiti o bug su cui lavorare. Possono essere assegnate a uno o più utenti, etichettate, e catalogate.

\clearpage