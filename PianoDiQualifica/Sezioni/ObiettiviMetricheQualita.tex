\section{Obiettivi e metriche di qualità}

\subsection{Obiettivi di qualità}

\subsubsection{Obiettivi di qualità di processo}

\begin{table}[H]
\rowcolors{2}{gray!25}{white}
    \renewcommand{\arraystretch}{1.5}
    \begin{tabular}{ m{0.1\textwidth}<{\centering}  m{0.2\textwidth}<{\centering}  m{0.3\textwidth}<{\centering}  m{0.3\textwidth}<{\centering} }
        \rowcolor{darkblue}
        \textcolor{white}{\textbf{ID}} &\textcolor{white}{\textbf{Nome}}& \textcolor{white}{\textbf{Descrizione}} & \textcolor{white}{\textbf{Metriche associate}}\\ 
        
        OPC01 & 
        Miglioramento continuo. &
        Capacità del processo di valutare e migliorare costantemente le proprie prestazioni. &
        MPC01: SPICE. \\
        
        \rowcolor{gray!25}
        OPC02 &
        Efficienza nell’utilizzo delle risorse. &
        Assicurare il corretto consumo delle risorse durante le attività di progetto. &
        MPC02: Budgeted cost of work scheduled; \newline
        MPC03: Actual cost of work performed; \newline
        MPC04: Budgeted cost of work performed. \\

        OPC03 &
        Rispetto della pianificazione. &
        Rispettare le scadenze temporali ed i limiti economici descritti all’interno del \textit{PianodiProgetto-v1.0.0}. &
        MPC05: Schedule variance; \newline
        MPC06: Budget variance. \\

    \end{tabular}
    \caption{Obiettivi di qualità di processo}
\end{table}

\subsubsection{Obiettivi di qualità di prodotto}

\begin{table}[H]

\rowcolors{2}{gray!25}{white}
    \renewcommand{\arraystretch}{1.5}
    \begin{tabular}{ m{0.1\textwidth}<{\centering}  m{0.2\textwidth}<{\centering}  m{0.3\textwidth}<{\centering}  m{0.3\textwidth}<{\centering} }
    	\rowcolor{gray!00} {\large{\textbf{Documenti}}}\\ 
        \rowcolor{darkblue}
        \textcolor{white}{\textbf{ID}} &\textcolor{white}{\textbf{Nome}}& \textcolor{white}{\textbf{Descrizione}} & \textcolor{white}{\textbf{Metriche associate}}\\ 

		\rowcolor{gray!00}
        OQP01 &
        Leggibilità dei documenti. &
        I documenti devono essere comprensibili ad utenti con licenza media. &
        MQP01: Indice di Gulpease. \\        
\end{tabular}


\rowcolors{2}{gray!25}{white}
    \renewcommand{\arraystretch}{1.5}
    \begin{tabular}{ m{0.1\textwidth}<{\centering}  m{0.2\textwidth}<{\centering}  m{0.3\textwidth}<{\centering}  m{0.3\textwidth}<{\centering} }
        \rowcolor{gray!00} {\large{\textbf{Software}}}\\ 
        \rowcolor{darkblue}
        \textcolor{white}{\textbf{ID}} &\textcolor{white}{\textbf{Nome}}& \textcolor{white}{\textbf{Descrizione}} & \textcolor{white}{\textbf{Metriche associate}}\\ 

		\rowcolor{gray!00}
        OQP02 &
        Manutenibilità. &
        Livello di capacità del prodotto richiesto per modifiche e correzioni. Il codice prodotto deve permettere di individuare facilmente gli errori. &
        MQP02: Profondità di una gerarchia; \newline
        MQP03: Numero parametri per metodo; \newline
        MQP06: complessità ciclomatica; \newline
        MQP08: Numero di Code smell; \newline
        MQP09: Linee di Commento per Linee di Codice. \\

		\rowcolor{gray!25}        
        OQP03 &
        Funzionalità. &
        Tutti i requisiti richiesti e riportati nel documento \textit{AnalisiDeiRequisiti-v1.0.0} devono essere soddisfatti. &
        MQP05: percentuale requisiti obbligatori soddisfatti. \\

		\rowcolor{gray!00}
        OQP04 &
        Affidabilità. &
        Livello di affidabilità del prodotto di essere in grado di svolgere tutte le funzionalità implementate. &
        MQP04: Code coverage; \newline
        MQP07: Numero di bug; \newline
        MQP10: Branch coverage; \newline
        MQP11: Successo dei test; \newline
        MQP12: Numero di vulnerabilità. \\

    \end{tabular}
    \caption{Obiettivi di qualità di prodotto}
\end{table}

\subsection{Metriche di qualità}

\subsubsection{Metriche di qualità di processo}

\begin{table}[H]
\rowcolors{2}{gray!25}{white}
    \renewcommand{\arraystretch}{1.5}
    \begin{tabular}{ m{0.1\textwidth}<{\centering}  m{0.2\textwidth}<{\centering}  m{0.15\textwidth}<{\centering}  m{0.15\textwidth}<{\centering} m{0.3\textwidth}<{\centering}}
        \rowcolor{darkblue}
        \textcolor{white}{\textbf{ID}} &\textcolor{white}{\textbf{Nome}}& \textcolor{white}{\textbf{Valore tollerato}} & \textcolor{white}{\textbf{Valore ottimo}} & \textcolor{white}{\textbf{Obiettivo}}\\ 
        
        MPC01 &
        SPICE &
        Livello di Capability $\geq$ 2 &
        Livello di Capability $\geq$ 4 &
        OPC01: Miglioramento continuo \\

        MPC02 &
        Budgeted cost of work scheduled &
        $\geq$ 0 &
        $\geq$ 0 &
        OPC02: Efficienza nell’utilizzo delle risorse \\

        MPC03 &
        Actual cost of work performed &
        BCWS &
        BCWS &
        OPC02: Efficienza nell’utilizzo delle risorse\\

        MPC04 &
        Budgeted cost of work performed &
        $\geq$ 0 &
        $\geq$ BCWS &
        OPC02: Efficienza nell’utilizzo delle risorse \\

        MPC05 &
        Schedule variance &
        $\geq$ -15\% &
        0\% &
        OPC03: Rispetto della pianificazione \\

        MPC06 &
        Budget variance &
        $\geq$ -10\% &
        0\% &
        OPC03: Rispetto della pianificazione \\


    \end{tabular}
    \caption{Metriche di qualità di processo}
\end{table}

\subsubsection{Metriche di qualità di prodotto}


\begin{table}[H]
\rowcolors{2}{gray!25}{white}
    \renewcommand{\arraystretch}{1.5}
    \begin{tabular}{ m{0.1\textwidth}<{\centering}  m{0.2\textwidth}<{\centering}  m{0.15\textwidth}<{\centering}  m{0.15\textwidth}<{\centering} m{0.3\textwidth}<{\centering}}
        \rowcolor{darkblue}
        \textcolor{white}{\textbf{ID}} &\textcolor{white}{\textbf{Nome}}& \textcolor{white}{\textbf{Valore tollerato}} & \textcolor{white}{\textbf{Valore ottimo}} & \textcolor{white}{\textbf{Obiettivo}}\\ 
        MQP01 &
        Indice di Gulpease  &
        $\geq$ 40 &
        $\geq$ 70  &
        OQP01: Leggibilità dei documenti \\


        MQP02 &
        Profondità di una gerarchia &
        $\leq$ 3 &
        $\leq$ 2 &
        OQP02: Manutenibilità \\

        MQP03 &
        Numero parametri per metodo &
        $\leq$ 8 &
        $\leq$ 4 &
        OQP02: Manutenibilità \\


        MQP04 &
        Code coverage &
        $\geq$ 70\% &
        100\% &
        OQP04: Affidabilità \\

        MQP05 &
        Percentuale requisiti obbligatori soddisfatti &
        100\% &
        100\% &
        OQP03: Funzionalità \\


        MQP06 &
        Complessità ciclomatica &
        $\leq$ 20 &
        $\leq$ 10 &
        OQP02: Manutenibilità \\

        MQP07 &
        Numero di bug &
        $\leq$ 20 &
        $\leq$ 5 &
        OQP04: Affidabilità \\


        MQP08 &
        Numero di Code smell &
        $\leq$ 40 &
        $\leq$ 10 &
        OQP02: Manutenibilità \\

        MQP09 &
        Linee di Commento per Linee di Codice  &
        $\leq$ 25\% &
        $\leq$ 10\% &
        OQP02: Manutenibilità \\


        MQP10 &
        Branch coverage &
        $\geq$ 70\% &
        100\% &
        OQP04: Affidabilità \\

        MQP11 &
        Successo dei test &
        $\geq$ 80\% &
        100\% &
        OQP04: Affidabilità \\


        MQP12 &
        Numero di vulnerabilità &
        $\leq$ 2 &
        0 &
        OQP04: Affidabilità \\
    \end{tabular}
    \caption{Metriche di qualità di prodotto}
\end{table}