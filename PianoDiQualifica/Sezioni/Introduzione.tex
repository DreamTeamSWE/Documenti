\section{Introduzione}

\subsection{Scopo del Documento}
Questo documento ha il fine di fissare degli standard e degli obiettivi che permettano di quantificare la qualità dei processi e dei prodotti mostrandone l’andamento nel corso dell’intero progetto.
Il documento definirà inoltre le modalità di validazione e verifica al fine di evitare che venga effettuata una verifica di tipo retrospettivo permettendo così di poter rilevare e correggere errori in modo tempestivo e con un costo minore.

\subsection{Scopo del Prodotto}

L’obiettivo di Sweeat dell’azienda Zero12 è la creazione di un sistema software costituito da un’applicazione mobile e da una webapp. Lo scopo del prodotto è di fornire all’utente una guida dei locali gastronomici sfruttando i numerosi contenuti digitali creati dagli utenti sulle principali piattaforme social (Instagram e TikTok). In questo modo è possibile realizzare una classifica basata sulle impressioni e reazioni di chiunque usufruisca dei servizi dei locali, non solo da professionisti ed esperti del settore.

\subsection{Glossario}

Per evitare ambiguità relative alle terminologie utilizzate è stato creato un documento denominato “\textit{Glossario}”. Questo documento comprende tutti i termini tecnici scelti dai membri del gruppo e utilizzati nei vari documenti con le relative definizioni. Tutti i termini inclusi in questo glossario, vengono segnalati all’interno del documento con l’apice \textsuperscript{G} accanto alla parola.

\subsection{Standard di progetto}
Per il progetto Sweeat, il gruppo DreamTeam ha pensato di adottare lo standard \textbf{ISO/IEC 9126} per la parte relativa alla qualità del prodotto, mentre lo standard \textbf{ISO/IEC 15504} – detto anche “\textit{SPICE}” – per la parte relativa alla qualità del processo.
