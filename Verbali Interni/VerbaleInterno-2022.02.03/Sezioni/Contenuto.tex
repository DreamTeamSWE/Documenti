\section{Informazioni Generali}

\begin{itemize}
\item{\textbf{Luogo}}: Meeting Discord
\item{\textbf{Data}}: \D
\item{\textbf{Ora}}: 13 - 14
\item{\textbf{Partecipanti del Gruppo}}: 
	\begin{itemize}
	\item{\EP{},} 
	\item{\FP{},}
	\item{\GC{},}
	\item{\LW{},}
	\item{\MB{},}
	\item{\MG{},}
	\item{\PV{}.}
	\end{itemize} 
\item{\textbf{Segretario}}: \PV{}	
\end{itemize}

\section{Ordine del Giorno}
\begin{itemize}
\item{Stato dei Documenti}
\item{Stato del PoC}
\end{itemize}

\section{Resoconto}
 
\subsection{Parti mancanti ai documenti} 
La maggior parte dei documenti sono pronti e/o in attesa di verifica. 
Sono saltate all'occhio delle mancanze sul \textit{Piano di Progetto} (§4) e delle bozze dimenticate e/o scartate nelle \textit{Norme di Progetto} su le quali ci si concentrerà nei prossimi giorni.
 
\subsection{Difficoltà riscontrate con i servizi di AWS}
Nell'uso dei servizi di AWS sono emerse delle difficoltà nel settaggio del server \textbf{Aurora}, visto che ha utilizzato delle risorse senza nessun utilizzo. Si è pensato di chiedere all'azienda \textit{Zero12} delle delucidazioni sul primo settaggio.

\subsection{Problemi sorti con il crawler di TikTok}
Visti i problemi avuti precedentemente con i crawler di TikTok si stanno facendo tentativi con altri crawler.
Durante questi ulteriori test è sorto un problema : per come funziona TikTok, i vari post non hanno abbinato la posizione del luogo in questione come invece succede su Instagram. Questo ci costringerebbe ad affidarci a delle informazioni non obbligatorie, che possono quindi essere omesse, inserite dagli autori dei post. Si è deciso di chiedere all'azienda \textit{Zero12} come comportarsi a riguardo.
