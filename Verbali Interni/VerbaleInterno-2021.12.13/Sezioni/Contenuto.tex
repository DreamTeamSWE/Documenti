\section{Informazioni Generali}

\begin{itemize}
\item{\textbf{Luogo}}: Meeting Discord
\item{\textbf{Data}}: \D{}
\item{\textbf{Ora}}: 13:00 - 14:00
\item{\textbf{Partecipanti del Gruppo}}: 
	\begin{itemize}
	\item{\EP{},} 
	\item{\FP{},}
	\item{\GC{},}
	\item{\LW{},}
	\item{\MB{},}
	\item{\MG{},}
	\item{\PV{}.}
	\end{itemize} 
\item{\textbf{Segretario}}: \PV{}	
\end{itemize}

\section{Ordine del Giorno}
\begin{itemize}
\item{Punto della situazione sul lavoro svolto;}
\item{Discussione primo incontro formativo con Zero12;}
\item{Discussione dei linguaggi da adottare;}
\item{Discussione del proof of concept;}
\item{Modifiche alla gestione del versionamento.}
\end{itemize}

\section{Resoconto}

\subsection{Proof of concept}
Dopo il primo incontro di formazione con l'azienda Zero12 sono nate varie discussioni sia sui linguaggi da adottare che del probabile proof of concept, non ancora discusso con l'azienda. Perciò, prima di prendere decisioni a riguardo, abbiamo deciso di avere un incontro con l'azienda incentrato su questo.


\subsection{Modifiche al versionamento}
Dopo le passate settimane di lavoro, abbiamo deciso di apportare delle modifiche al versionamento e quindi anche al registro delle modifiche : d'ora in poi il verificatore non verrà più registrato sul registro delle modifiche su una nuova linea bensì sulla stessa linea di una modifica verrà riportato nella descrizione (se verificata). Questo per rendere facilitare la tracciabilità di ciò che è verificato. Inoltre cambia il significato della seconda cifra nel numero di versione : prima il suo aumentare indicava il passaggio di una verifica. Adesso indica una verifica complessiva del documento e/o una verifica complessiva delle modifiche effettuate dall'ultima verifica complessiva. 