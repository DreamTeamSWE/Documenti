\section{Informazioni Generali}

\begin{itemize}
\item{\textbf{Luogo}}: Meeting Discord
\item{\textbf{Data}}: \D{}
\item{\textbf{Ora}}: 13 - 14
\item{\textbf{Partecipanti del Gruppo}}: 
	\begin{itemize}
	\item{\EP{},} 
	\item{\FP{},}
	\item{\GC{},}
	\item{\LW{},}
	\item{\MB{},}
	\item{\MG{},}
	\item{\PV{}.}
	\end{itemize} 
\item{\textbf{Segretario}}: \PV{}	
\end{itemize}

\section{Ordine del Giorno}
\begin{itemize}
\item{Stato di avanzamento dei documenti;}
\item{Discussione sull'incontro con Cardin;}
\item{Tecnologie scelte per il PoC;}
\item{Questione crediti.}
\end{itemize}

\section{Resoconto}

\subsection{Discussione sull'incontro con Cardin}
L'incontro con Cardin ha portato alla luce due  problemi : 
\begin{itemize}
\item Gli schemi UML dei casi d'uso non sono idonei;
\item Il PoC non include tutte le tecnologie del programma finale. 
\end{itemize}
Per gli schemi UML sono già a lavoro i redattori dei Casi d'uso (guardare \textit{Analisi dei Requisiti})
Per le tecnologie del PoC, ne abbiamo discusso e abbiamo deciso di includere le tecnologie riportate nel prossimo paragrafo.

\subsection{Tecnologie scelte per il PoC}
Le tecnologie discusse e approvate per il PoC, dopo l'incontro con Cardin, sono : 
\begin{itemize}
    \item Python per le Lambda;
    \item Amazon S3 per l'archiviazione dei dati;
    \item React per la parte Front End;
    \item Amazon Aurora MySQL per il database.
\end{itemize}

\subsection{Questione dei crediti}
Per la realizzazione del PoC, siccome dobbiamo utilizzare tutte le tecnologie, abbiamo bisogno di fare dei test con Amazon Aurora, Rekognition e Comprehend. Ma questi richiedono dei crediti AWS. Si è deciso di chiedere tramite Slack all'azienda \textit{Zero12} quando saranno in grado di darci tali crediti.

\pagebreak

