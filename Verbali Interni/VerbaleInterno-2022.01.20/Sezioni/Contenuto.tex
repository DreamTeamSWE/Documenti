\section{Informazioni Generali}

\begin{itemize}
\item{\textbf{Luogo}}: Meeting Discord
\item{\textbf{Data}}: 2022-01-20
\item{\textbf{Ora}}: 13 - 14
\item{\textbf{Partecipanti del Gruppo}}: 
	\begin{itemize}
	\item{\EP{},} 
	\item{\FP{},}
	\item{\GC{},}
	\item{\LW{},}
	\item{\MG{}.}
	\end{itemize} 
\item{\textbf{Segretario}}: \GC{}	
\end{itemize}

\section{Ordine del Giorno}
\begin{itemize}
\item{Stato di avanzamento dei documenti;}
\item{Discussione sulle tecnologie utilizzate per lo sviluppo del PoC e dei file JSON prodotti dal crawler;}
\item{Difficoltà riscontrate con il crawler di TikTok.}
\end{itemize}

\section{Resoconto}

\subsection{Stato di avanzamento dei documenti}

La prima tematica affrontata durante questa riunione è lo \textbf{stato di avanzamento} di \textbf{tutti} i documenti (\AdR, \NdP, \PdP, \PdQ e \G): l'obiettivo è quello di approvare i documenti il prima possibile, per poterci concentrare principalmente sullo sviluppo del PoC. Ci siamo prefissati di ultimare le recenti modifiche segnalate dai verificatori e terminare i paragrafi mancanti entro il \textbf{22 Gennaio 2022}.

\subsection{Discussione sulle tecnologie utilizzate per lo sviluppo del PoC e dei file JSON prodotti dal crawler}

La scorsa settimana abbiamo discusso delle diverse tecnologie da utilizzare per lo sviluppo del PoC, tuttavia, in questi sette giorni, abbiamo fatto un po' di scouting e ci siamo resi conto che le tecnologie scelte non erano sufficienti: ad esempio, per la WebApp, non è sufficiente utilizzare React (per la parte front-end) ed il servizio di archiviazione S3, ma è necessario integrare la parte back-end con \textbf{NodeJS} e trovare un servizio di hosting - per fare ciò, abbiamo optato per \textbf{Amazon Amplify}. A tal proposito, abbiamo chiesto all'azienda se questo servizio offerto da AWS potesse essere la soluzione ideale per il nostro progetto e ci dato un responso positivo (via Slack).
\\ \\
Inoltre, abbiamo discusso anche di come far “dialogare” assieme le diverse tecnologie scelte: ad esempio, è stata affrontata la questione di come salvare i risultati prodotti dal crawler nello spazio di archiviazione (Amazon S3), di cosa inserire nel database e di come visualizzare gli attributi\glo nella WebApp per il PoC; il crawler dovrà essere in grado di produrre un file in formato \textbf{JSON}, che conterrà il responso dei file multimediali analizzati – che verrà prodotto dai servizi di AWS (indicati nel \textit{Verbale Interno v1.0.0} del 2022-01-13) – e che verrà salvato in S3.  

\subsection{Difficoltà riscontrate con il crawler di TikTok}

In quest'ultima settimana \MG{} si è occupato di cercare e studiare un crawler di TikTok che fosse in grado di svolgere, perlomeno in minima parte, quanto richiesto dal proponente. Purtroppo sono sorte delle difficoltà, in quanto tutti i crawler trovati sembrano avere dei problemi: non funzionano correttamente, sono poco affidabili e non fanno quanto promesso. 

