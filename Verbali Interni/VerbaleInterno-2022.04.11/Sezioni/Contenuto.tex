\section{Informazioni Generali}

\begin{itemize}
	\item{\textbf{Luogo}}: Meeting Discord
	\item{\textbf{Data}}: \D
	\item{\textbf{Ora}}: 13:00 - 13:40
	\item{\textbf{Partecipanti del Gruppo}}:
	\begin{itemize}
		\item{\EP{};}
		\item{\FP{};}
		\item{\GC{};}
		\item{\LW{};}
		\item{\MG{};}
		\item{\MG{};}
		\item{\PV{}.}
	\end{itemize}
	\item{\textbf{Segretario}}: \PV{}
\end{itemize}

\section{Ordine del Giorno}
\begin{itemize}
	\item{Discussione sullo stato di avanzamento;}
	\item{Prossima data di consegna;}
	\item{Decisione divisione documenti (manuale utente in lingua italiana e manuale API in lingua inglese).}
\end{itemize}

\section{Resoconto}

\subsection{Stato di avanzamento del progetto}

Abbiamo analizzato lo stato di avanzamento della parte front-end e back-end per cercare di capire per quando stimare la prossima data di consegna.

Ci siamo prefissati come data di “collaudo” della WebApp il \textbf{22 Aprile 2022}, da quel momento dovrà esserci una WebApp funzionante e che implementa i requisiti obbligatori discussi nel documento “\AdR{}”.

In base a ciò che avremo prodotto con la WebApp ed a seconda di eventuali ritardi, programmeremo la prossima data di consegna.

\subsection{Divisione documenti da produrre}

In questa riunione abbiamo deciso di dividerci anche i compiti relativi ai documenti da produrre. \\ 
In particolar modo, abbiamo scelto di realizzare il: 
\begin{itemize}
 	\item{\textbf{Manuale utente} in \textbf{lingua italiana}: il motivo è che si presuppone che il destinatario finale conosca la lingua italiana (essendo, la WebApp stessa, realizzata in italiano);}
 	\item{\textbf{Manuale delle API} in \textbf{lingua inglese}: esso verrà realizzato in lingua inglese di modo che possa essere fruito da chiunque (nel caso dovessero esserci degli sviluppi futuri della WebApp e qualcuno volesse sfruttare o ampliare le API realizzate).}
\end{itemize}

Per quanto riguarda il manuale utente, se ne occuperà chi sta sviluppando la parte frontend della WebApp, mentre il manuale delle API verrà prodotto da chi si sta occupando del back-end. \\

Infine, il documento legato alle specifiche tecniche (contenente i diagrammi delle classi e di sequenza), sarà collaborativo e realizzato da entrambe le parti (parte front-end e back-end).
