\section{Informazioni Generali}

\begin{itemize}
	\item{\textbf{Luogo}}: Meeting Discord
	\item{\textbf{Data}}: \D
	\item{\textbf{Ora}}: 13:00 - 13:30
	\item{\textbf{Partecipanti del Gruppo}}:
	\begin{itemize}
		\item{\EP{};}
		\item{\FP{};}
		\item{\GC{};}
		\item{\LW{};}
		\item{\MG{};}
		\item{\PV{}.}
	\end{itemize}
	\item{\textbf{Segretario}}: \PV{}
\end{itemize}

\section{Ordine del Giorno}
\begin{itemize}
	\item{Discussione sullo stato di avanzamento;}
	\item{Prossimo incontro con l'azienda;}
	\item{Discussione sui filtri che dovranno essere implementati;}
\end{itemize}

\section{Resoconto}

\subsection{Prossimo incontro con l'azienda}
Si è deciso di chiedere un incontro all'azienda per poter discutere di alcuni requisiti obbligatori, opzionali e facoltativi (per esempio se implementare o meno il filtro relativo all'orario di apertura, piuttosto che del giorno di apertura, dei locali da mostrare nella classifica), oltre che mostrare l'andamento del progetto e capire se ci sono richieste particolari da implementare.


\subsection{Filtri da implementare}
Si è discusso dei filtri che dovranno essere svolti in primis secondo le priorità indicate nel documento relativo all'Analisi dei Requisiti. \\
I filtri che verranno implementati in prima battuta nella WebApp e che saranno a supporto della classifica sono i seguenti:
\begin{itemize}
	\item{\textbf{Filtra per zona},}
	\item{\textbf{Filtra per tipo di cucina}.}
\end{itemize}

