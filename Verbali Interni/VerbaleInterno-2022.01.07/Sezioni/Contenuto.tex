\section{Informazioni Generali}

\begin{itemize}
\item{\textbf{Luogo}}: Meeting Discord
\item{\textbf{Data}}: 2022-01-07
\item{\textbf{Ora}}: 09 - 10
\item{\textbf{Partecipanti del Gruppo}}: 
	\begin{itemize}
	\item{\EP{},} 
	\item{\FP{},}
	\item{\GC{},}
	\item{\LW{},}
	\item{\MB{}.}
	\end{itemize} 
\item{\textbf{Segretario}}: \GC{}	
\end{itemize}

\section{Ordine del Giorno}
\begin{itemize}
\item{Punto della situazione sul lavoro svolto dai sotto-gruppi;}
\item{Sistemazione Repository GitHub;}
\item{Discussione sul PoC;}
\item{Discussione relativa al colloquio imminente,}
\item{Discussione sulla prima consegna.}
\end{itemize}

\section{Resoconto}

\subsection{Punto della situazione sul lavoro svolto dai sotto-gruppi}

È stato fatto il punto della situazione, per capire l'andamento del lavoro dei sotto-gruppi. In particolare, sono stati presi in esame i documenti per capire cosa c'era da verificare nell'immediato.
Poiché quattro membri del gruppo si stanno occupando del Proof of Concept, si è parlato delle difficoltà riscontrate nel suo sviluppo e nella sua gestione.     

\subsection{Sistemazione Repository GitHub}

È stato sistemato il repository di GitHub, dove erano presenti delle vecchie issue che non erano più state utilizzate/non erano state gestite correttamente, oltre ad etichettare ed assegnare correttamente le issue attualmente aperte.  

\subsection{Discussione sul PoC}

Sono sorte le prime piccole difficoltà inerenti al PoC, in particolare: non è molto chiaro il funzionamento delle \textbf{Lambda di AWS}, per questo abbiamo discusso tra di noi e deciso di chiedere un parere all'azienda \textit{Zero12} tramite la piattaforma di comunicazione \textit{Slack}. 

\subsection{Discussione relativa al colloquio imminente}

Lunedì 10 Gennaio 2022 verrà fatto un breve colloquio \textit{Zoom} con il prof. \textit{Tullio Vardanega} relativo allo stato di avanzamento del progetto, in vista della prima revisione RTB (\textit{Requirements and Technology Baseline}) e ciascun gruppo dovrà esporre le proprie perplessità e difficoltà. Dal momento in cui sono sorti alcuni dubbi, sono state raccolte alcune domande da fare al prof. per capire cosa consegnare, come presentare il PoC e se c'è da realizzare una piccola presentazione con quanto fatto fino ad ora.

Inoltre, prima della data del colloquio, abbiamo deciso di realizzare un'altra riunione interna con tutti i componenti del gruppo per parlare in maniera più approfondita di cosa esporre.

\subsection{Discussione sulla prima consegna}

In questa riunione si è parlato anche della prima consegna: si è deciso che, prima di fissare la data della prima consegna ed al termine della verifica dei documenti attualmente in corso, verrà fatto un incontro con l'azienda \textit{Zero12}, per capire se ciò che è stato fatto fino ad ora rispecchia quanto è stato richiesto.   