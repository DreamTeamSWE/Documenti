\section{Informazioni Generali}

\begin{itemize}
\item{\textbf{Luogo}}: Meeting Discord
\item{\textbf{Data}}: \D
\item{\textbf{Ora}}: 13:00 - 14:00
\item{\textbf{Partecipanti del Gruppo}}: 
	\begin{itemize}
	\item{\EP{},} 
	\item{\FP{},}
	\item{\GC{},}
	\item{\LW{},}
	\item{\MG{}.}
	\end{itemize} 
\item{\textbf{Segretario}}: \GC{}	
\end{itemize}

\section{Ordine del Giorno}
\begin{itemize}
\item{Andamento del PoC}
\begin{itemize}
\item{API Gateway}
\item{API Google Maps}
\end{itemize}
\item{Dati da inserire nel database per il PoC}
\item{Stato dei Documenti}
\end{itemize}

\section{Resoconto}
 
\subsection{Andamento del PoC}

Nelle ultime settimane ci siamo occupati principalmente del PoC ed in questa riunione abbiamo messo in luce alcune problematiche e cambi di ruolo, in particolare: 
\begin{itemize}
\item \EP{} si occuperà delle API Gateway, anziché della parte front-end come stabilito inizialmente, in quanto non erano state prese in carico da nessuno;
\item \LW{} si occuperà della parte front-end al posto di \EP{};
\item a \PV{} è stata assegnata la parte relativa alle \textbf{API di Google Maps}, poiché possono tornare utili per identificare ciascun locale (ad esempio: è possibile identificare un locale in base alle coordinate geografiche) e nessuno se ne stava occupando.
\end{itemize}

Gli altri componenti del gruppo continuano a lavorare su ciò che avevano iniziato: \FP{} e \MB{} si stanno occupando di far funzionare il \textbf{crawler di Instagram} e di interfacciarlo con i servizi di AWS, \GC{} si sta occupando della parte \textbf{front-end}, mentre \MG{} sta continuando ad informarsi e sperimentare con il \textbf{crawler di TikTok}, considerati anche i consigli dati ieri da parte dell'azienda (\textit{Verbale Esterno 2022-01-26}).

Inoltre, si è discusso se avesse senso o meno utilizzare NodeJS per la parte back-end, piuttosto che usare sempre Python (dato che le Lambda sono comunque in Python), oltre al fatto di usare \textbf{AWS Amplify} piuttosto che sfruttare una pipeline CI/CD che carichi la WebApp su S3, sfruttando CloudFront (come suggerito da Michele Massaro di Zero12 in una recente comunicazione avvenuta via \textit{Slack}).

\subsubsection{API Gateway}

Come esposto nel \Ve{} 2022-01-26, le API Gateway di AWS servono per interfacciare la WebApp con l'architettura della piattaforma, preservando la sicurezza dei dati presenti nel database. Nel nostro sistema le API Gateway fungono da intermediario, infatti il loro scopo è duale: mostrano ciò che un utente chiede tramite la WebApp (ad esempio, se l'utente effettua una ricerca o vuole visualizzare le informazioni di un locale) e recuperano le informazioni richieste all'interno del database mediante delle chiamate API Rest, sfruttando le Lambda di AWS.

\subsubsection{API Google Maps}
 
Le \textbf{API di Google Maps} potrebbero tornare utili per il nostro sistema, in quanto permettono di integrare le mappe digitali di Google Maps nella propria piattaforma (nel nostro caso la WebApp), di identificare mediante le coordinate geografiche (per esempio, noi possiamo estrapolare le coordinate, come longitudine e latitudine, con il crawler partendo da un'immagine) e sapere come raggiungere un determinato luogo (per esempio: un ristorante o un bar). \\
L'unico vincolo da considerare sono le chiamate gratuite che si possono fare con queste API di Google Maps (ossia quante volte posso sfruttare la mappa di Google in un mese per progetto), stimate a \textbf{28.500 al mese} (Google mette a disposizione 200\$ al mese di prova per ciascuna azienda - \color{blue}  \underline{\url{https://mapsplatform.google.com/pricing/}}\color{black}).  

\subsection{Dati da inserire nel database per il PoC}

Per il PoC si è deciso di realizzare un database che contenesse le informazioni ottenute dal crawler, ma che non fosse troppo complesso. A tal proposito, si è deciso di realizzare una tabella contenente le seguenti informazioni:

\begin{itemize}
\item Username utente che ha pubblicato l'immagine,
\item Indirizzo dell'immagine analzzata con i servizi di AWS,
\item Testo del post presente nell'immagine,
\item Nome del luogo in cui è stata pubblicata l'immagine,
\item Latitudine,
\item Longitudine,
\item Risultati ottenuti da Comprehend,
\item Risultati ottenuti da Rekognition.
\end{itemize}

Da quanto è emerso ieri nell'incontro con il proponente, si è deciso di includere i risultati ottenuti dai servizi di AWS direttamente nel database, anziché all'interno di file JSON, per maggior praticità (ad esempio, così risulta molto più semplice ed efficiente effettuare delle ricerche, anziché dover cercare informazioni sui file JSON).

\subsection{Stato dei Documenti}

Infine, abbiamo parlato dello stato di avanzamento dei documenti: sebbene siano a buon punto, ci sono delle piccole migliorie/aggiunte da apportare (ad esempio: nel \textit{\PdQ} va terminato il consuntivo, va controllato il preventivo e la parte relativa alle metriche – ). Inoltre, vanno sistemati alcuni termini del \textit{\G} su tutti i documenti.
