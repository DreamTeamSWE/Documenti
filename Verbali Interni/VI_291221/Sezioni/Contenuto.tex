\section{Informazioni Generali}

\begin{itemize}
\item{\textbf{Luogo}}: Meeting Discord
\item{\textbf{Data}}: 29 Dicembre 2021
\item{\textbf{Ora}}: 9:00 - 10:30
\item{\textbf{Partecipanti del Gruppo}}: 
	\begin{itemize}
	\item{\EP{},} 
	\item{\FP{},}
	\item{\GC{},}
	\item{\LW{},}
	\item{\MB{},}
	\item{\MG{},}
	\item{\PV{}}
	\end{itemize} 
\item{\textbf{Segretario}}: \PV{}	
\end{itemize}

\section{Ordine del Giorno}
\begin{itemize}
\item{Punto della situazione sul lavoro svolto;}
\item{AWS;}
\item{Discussione sui crawler;}
\item{Approvazione dei documenti.}
\end{itemize}

\section{Resoconto}

\subsection{AWS}
Seguendo il consiglio dell'azienda Zero12, abbiamo creato l'account AWS del DreamTeam e abbiamo cominciato a guardare i vari servizi offerti gratuiti per il PoC.

\subsection{Crawler}
Si è discusso dei vari crawler, concentrandosi su linguaggio e possibile implementazione con AWS.

\subsection{Data di approvazione documenti}
Si è deciso che entro la data del 7 Gennaio 2022 i documenti dovranno essere pronti per poter essere approvati.
