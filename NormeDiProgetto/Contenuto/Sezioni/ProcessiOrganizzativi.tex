\section{Processi Organizzativi}

\subsection{Gestione di processo}

\subsubsection{Scopo}
Secondo lo standard ISO-12207:1995 la gestione di processo contiene le attività e i compiti generici utili per la gestione dei rispettivi processi. Vengono individuate le seguenti attività:
\begin{enumerate}
\item Inizializzazione e definizione dello scopo;
\item Pianificazione e stima dei tempi, delle risorse, dei costi, assegnazione di compiti e responsabilità;
\item Esecuzione e controllo;
\item Revisione e valutazione;
\item Determinazione della fine del processo.
\end{enumerate}

\subsubsection{Obiettivi}
\begin{itemize}
\item Semplificare e gestire la comunicazione tra i membri del gruppo e l'esterno;
\item Coordinare l'assegnazione dei ruoli e compiti;
\item Monitorare il lavoro del gruppo e pianificare le attività da svolgere;
\item Definire le linee guida generali per la formazione dei membri.
\end{itemize}

\subsubsection{Coordinamento}
L'attività di coordinamento è responsabile della gestione delle comunicazioni interne, esterne e delle riunioni.


\paragraph{Comunicazione}
Le comunicazioni avvengono su due piani diversi: tra i membri del gruppo (interno) e tra i membri del gruppo e uno o più soggetti esterni (esterno). I soggetti esterni si identificano in:
\begin{itemize}
\item \textbf{Proponente}: l'azienda zero12;
\item \textbf{Committenti}:  Prof. Tullio Vardanega e Prof. Riccardo Cardin.
\end{itemize}

\subparagraph{Comunicazione interna}
Il mezzo di comunicazione principale tra membri del gruppo sarà Slack e Telegram.
In particolare per Slack saranno creati dei canali appositi e specifici per argomenti ritenuti di importanza ed i membri del gruppo dovranno intrattenere le discussioni nei canali appositi per evitare confusione e/o ammassi incoerenti di messaggi. La creazione dei canali spetta al Responsabile di progetto e essi saranno destinati a mutare nel tempo per accomodare le esigenze del gruppo.
Telegram verrà usato per la componente più informale delle discussioni, per le decisioni meno importanti o in caso di problemi con il funzionamento di Slack (da considerarsi comunque una possibilità remota). 

Per quanto riguarda la comunicazione interna tramite videochiamata (tipicamente le riunioni) lo strumento principale sarà Discord, scelto poiché è semplice, conosciuto da tutti i membri del gruppo e multipiattaforma. In alternativa sarà utilizzato Zoom, altro strumento con il quale tutti i membri del gruppo hanno già avuto esperienza.

\subparagraph{Comunicazione esterna}
 Per la comunicazione con i soggetti esterni sarà utilizzato un indirizzo email apposito: \textit{dreamteam.unipd@gmail.com}.
Tutti i membri del gruppo avranno accesso alla casella di posta elettronica e saranno tenuti a verificare e notificare il gruppo circa la ricezione di nuovi messaggi mentre la stesura e l'invio di un nuovo messaggio spetterà al Responsabile di progetto.
Prima dell'invio del messaggio esso sarà sottoposto ad una breve verifica e approvazione da parte del gruppo.
Si sono decise le seguenti convenzioni per quanto riguarda la struttura delle e-mail:
\begin{itemize}
\item L'oggetto dovrà terminare con la dicitura "| SWE - Unipd";
\item In generale il corpo dovrà mantenere un tono il più possibile formale ed il contenuto dovrà essere chiaro e conciso;
\item Il corpo dovrà terminare con la firma "DreamTeam".
\end{itemize}

\paragraph{Riunioni}
Le riunione potranno essere interne od esterne. Prima di ogni riunione verrà nominato un segretario che ha lo scopo di far rispettare l'ordine del giorno e dirigere la discussione, tenendo traccia dei punti salienti per poi poter redigere il verbale.

\subparagraph{Riunioni interne}
Alle riunione interne parteciperanno solo i membri del gruppo, per essere considerate valide dovranno essere presenti almeno quattro dei membri del gruppo. Prima di ogni riunione il Responsabile di progetto dovrà:
\begin{itemize}
\item fissare la data e l'orario;
\item definire l'ordine del giorno;
\item nominare il segretario;
\item comunicare tutto quanto detto sopra (o eventuali variazioni) ai membri con ragionevole anticipo.
\end{itemize}

Gli incontri saranno svolti a cadenza settimanale e, in caso di necessità, anche più frequentemente.
Ogni membro può rivolgersi al Responsabile di progetto per richiedere una riunione interna, proponendo l'ordine del giorno, poi se ritenuta necessaria il Responsabile si attiverà per l'organizzazione dell'incontro.

In ogni caso i membri del gruppo sono liberi di indire riunioni informali tra gli interessati, le persone coinvolte gestiranno orari, date e temi di discussione tra di loro come meglio credono senza intervento del Responsabile di progetto.
Generalmente la riunione non sarà considerata valida e non sarà prodotto un verbale.

\subparagraph{Riunioni esterne}
Alle riunioni esterne parteciperanno sia i membri del gruppo che soggetti esterni (proponente e committente). La richiesta di una riunione potrebbe provenire da entrambe le parti (soggetti esterni o Responsabile di progetto) e la piattaforma standard utilizzata sarà GMeet/Zoom, anche se il soggetto esterno è libero di scegliere la piattaforma che più desidera (non sono escluse le riunioni in presenza). In ogni caso il Responsabile di progetto dovrà nominare un segretario, incaricato della stesura del verbale.

\subsection{Processo di pianificazione}
\subsubsection{Scopo}
\subsubsection{Ruoli di progetto}
\paragraph{Responsabile di progetto}
\paragraph{Amministratore di progetto}
\paragraph{Analista}
\paragraph{Progettista}
\paragraph{Programmatore}
\paragraph{Verificatore}
\subsubsection{Gestione dei ticket}

\subsection{Formazione dei membri}
\subsubsection{Formazione interna}
\subsubsection{Formazione esterna}
