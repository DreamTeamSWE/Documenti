\section{Processi Organizzativi}
\subsection{Gestione di processo}
Secondo lo standard ISO-12207:1995 la gestione di processo contiene le attività e i compiti generici utili per la gestione dei rispettivi processi. Vengono individuate le seguenti attività:
\begin{enumerate}
\item Inizializzazione e definizione dello scopo;
\item Pianificazione e stima dei tempi, delle risorse, dei costi, assegnazione di compiti e responsabilità;
\item Esecuzione e controllo;
\item Revisione e valutazione;
\item Determinazione della fine del processo.
\end{enumerate}

\subsubsection{Obiettivi}
\begin{itemize}
\item Semplificare e gestire la comunicazione tra i membri del team e l'esterno;
\item Coordinare l'assegnazione dei ruoli e compiti;
\item Monitorare il lavoro del team e pianificare le attività da svolgere;
\item Definire le linee guida generali per la formazione dei membri.
\end{itemize}

\subsubsection{Coordinamento}
L'attività di coordinamento è responsabile della gestione delle comunicazioni sia interne che esterne e delle riunioni.


\paragraph{Comunicazione}
Le comunicazioni avvengono su due piani diversi: tra i membri del team (interno) e tra i membri del team e uno o più soggetti esterni (esterno). I soggetti esterni si identificano in:
\begin{itemize}
\item \textbf{Proponente}: l'azienda zero12;
\item \textbf{Committenti}:  Prof. Tullio Vardanega e Prof. Riccardo Cardin.
\end{itemize}

\subparagraph{Comunicazione interna}
Il mezzo di comunicazione principale tra membri del team sarà Slack e Telegram.
In particolare per Slack saranno creati dei canali appositi e specifici per argomenti ritenuti di importanza ed i membri del team dovranno intrattenere le discussioni nei canali appositi per evitare confusione e/o ammassi incoerenti di messaggi. La creazione dei canali spetta al Responsabile di progetto e essi saranno destinati a mutare nel tempo per accomodare le esigenze del team.
Telegram verrà usato per la componente più informale delle discussioni, per le decisioni meno importanti o in caso di problemi con il funzionamento di Slack (da considerarsi comunque una possibilità remota). 

Per quanto riguarda la comunicazione interna tramite videochiamata (tipicamente le riunioni) lo strumento principale sarà Discord, scelto poiché è semplice, conosciuto da tutti i membri del team e multipiattaforma. In alternativa sarà utilizzato Zoom, altro strumento con il quale tutti i membri del team hanno già avuto esperienza.

\subparagraph{Comunicazione esterna}
 Per la comunicazione con i soggetti esterni sarà utilizzato un indirizzo email apposito: dreamteam.unipd@gmail.com.
Tutti i membri del team avranno accesso alla casella di posta elettronica e saranno tenuti a verificare e notificare il team circa la ricezione di nuovi messaggi mentre la stesura e l'invio di un nuovo messaggio spetterà al Responsabile di progetto.
Prima dell'invio del messaggio esso sarà sottoposto ad una breve verifica e approvazione da parte del team.
Si sono decise le seguenti convenzioni per quanto riguarda la struttura delle e-mail:
\begin{itemize}
\item L'oggetto dovrà terminare con la dicitura "| SWE - Unipd";
\item In generale il corpo dovrà mantenere un tono il più possibile formale ed il contenuto dovrà essere chiaro e conciso;
\item Il corpo dovrà terminare con la firma "DreamTeam".
\end{itemize}

\paragraph{Riunioni}
Le riunione potranno essere interne od esterne. Prima di ogni riunione verrà nominato un segretario che ha lo scopo di far rispettare l'ordine del giorno e dirigere la discussione, tenendo traccia dei punti salienti per poi poter redigere il verbale.

\subparagraph{Riunioni interne}
Alle riunione interne parteciperanno solo i membri del gruppo, per essere considerate valide dovranno essere presenti almeno quattro dei membri del gruppo. Prima di ogni riunione il responsabile di progetto dovrà:
\begin{itemize}
\item fissare la data e l'orario;
\item definire l'ordine del giorno;
\item nominare il segretario;
\item comunicare tutto quanto detto sopra (o eventuali variazioni) ai membri con ragionevole anticipo.
\end{itemize}
Gli incontri saranno svolti a cadenza settimanale, in caso di necessità anche più di frequente.

Ogni membro può rivolgersi al Responsabile di Progetto per richiedere una riunione interna, proponendo l'ordine del giorno, poi se ritenuta necessaria il Responsabile si attiverà per l'organizzazione dell'incontro.

In ogni caso i membri del team sono liberi di indire riunioni informali in cui partecipano due o tre persone, le persone interessate gestiranno orari, date e temi di discussione tra di loro come meglio credono ma la riunione non verrà considerata valida e non sarà prodotto un verbale.

\subparagraph{Riunioni esterne}
Alle riunioni esterne parteciperanno sia i membri del team che soggetti esterni (proponente e committente), la richiesta di una riunione potrebbe provenire da entrambe le parti (soggetti esterni o Responsabile di progetto) e la piattaforma standard utilizzata sarà GMeet/Zoom anche se il soggetto esterno è libero di scegliere la piattaforma che più desidera (inoltre non sono escluse le riunioni in presenza). In ogni caso il Responsabile dovrà nominare un segretario, incaricato della stesura del verbale.
