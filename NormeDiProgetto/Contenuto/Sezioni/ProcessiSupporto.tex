\section{Processi di Supporto}
\subsection{Documentazione}
\subsubsection{Scopo}
Ogni processo e attività per lo sviluppo del progetto dovrà essere documentata. Nella presente sezione verranno descritte regole e standard da seguire durante il processo di documentazione per l’intero ciclo di vita del software.

\subsubsection{Descrizione}
Vengono presentate decisioni e norme prescelte per:
\begin{itemize}
  \item Stesura;
  \item Verifica;
  \item Approvazione.
\end{itemize}

\subsubsection{Documenti prodotti}
I documenti prodotti sono:
\begin{itemize}
  \item \textbf{Norme di progetto}: documento interno che contiene norme e regole stabilite dal gruppo, che devono essere seguite per l’intera durata del progetto;
  \item \textbf{Glossario}: documento esterno dove sono presenti i termini tecnici usati nella documentazione con le loro definizioni, affinché non ci siano ambiguità e/o incongruenze;
  \item \textbf{Piano di progetto}: documento esterno con la pianificazione delle attività del progetto previste dal gruppo. Contiene la previsione dell’impegno orario dei singoli membri, il preventivo spese e i consuntivi di periodo;
  \item \textbf{Piano di qualifica}: documento esterno che descrive i criteri con cui si valuta la qualità;
  \item \textbf{Analisi dei requisiti}: documento esterno contenente requisiti e caratteristiche del prodotto finale;
  \item \textbf{Verbali}:
  \begin{itemize}
  		\item Interni: resoconti degli incontri del gruppo;
  		\item Esterni: resoconti degli incontri del gruppo con i committenti e/o il proponente.
	\end{itemize}
\end{itemize}

\subsubsection{Sistema software per la preparazione dei documenti}
Tutti i documenti prodotti dal gruppo verranno redatti usando il linguaggio di markup \LaTeX.

\subsubsection{Ciclo di vita di un documento}
Ogni documento passa per i seguenti step:
\begin{itemize}
  \item \textbf{Creazione}: il documento viene creato basandosi su un template comune;
  \item \textbf{Strutturazione}: il documento viene fornito di:
  \begin{itemize}
  		\item Registro delle modifiche;
  		\item Indice dei contenuti.
	\end{itemize}
  \item \textbf{Stesura}: il gruppo redige il documento adottando il metodo incrementale;
  \item \textbf{Revisione}: ogni sezione del corpo del documento è rivista da almeno un membro del gruppo che non sia il redattore della parte in verifica;
  \item \textbf{Approvazione}: se revisionato, il Responsabile di Progetto può stabilire che il documento è valido. Se approvato, può essere rilasciato.
\end{itemize}
Per semplificare le operazioni di verifica dovrà sempre essere reso disponibile il documento  completo (fino alla versione più attuale) in formato PDF.

\subsubsection{Struttura delle directory e dei files}
Per ogni documento si definisce la seguente struttura di directory:\\
\begin{forest}
  for tree={
    font=\ttfamily,
    grow'=0,
    child anchor=west,
    parent anchor=south,
    anchor=west,
    calign=first,
    edge path={
      \noexpand\path [draw, \forestoption{edge}]
      (!u.south west) +(7.5pt,0) |- node[fill,inner sep=1.25pt] {} (.child anchor)\forestoption{edge label};
    },
    before typesetting nodes={
      if n=1
        {insert before={[,phantom]}}
        {}
    },
    fit=band,
    before computing xy={l=15pt},
  }
[NomeDocumento
  [NomeDocumento.tex]
  [NomeDocumento.pdf]
  [Contenuto
    [Sezioni
        [Introduzione.tex]
        [Sezione1.tex]
        [Sezione2.tex]
    ]
    [Immagini
        [Immagine1.png]
        [Immagine2.jpeg]
    ]
    [Intro.tex]
    [RegistroModifiche.tex]
  ]
  [Struttura
    [Command.tex]
    [Impaginazione.tex]
    [Packages.tex]
  ]
]
\end{forest}\\
In particolare:
\begin{itemize}
\item \textbf{NomeDocumento.tex}: importa tutti le parti necessarie per comporre il documento finale;
\item \textbf{NomeDocumento.pdf}: versione del documento in formato PDF;
\item \textbf{Introduzione.tex}:  sezione introduttiva del documento, definisce lo scopo del prodotto e del documento, i riferimenti normativi e informativi e informazioni sul \textit{Glossario};
\item \textbf{Intro.tex}: contiene tutti i comandi per creare la prima pagina del documento;
\item \textbf{RegistroModifiche.tex}: contiene tutti i comandi per creare la tabella del registro delle modifiche;
\item \textbf{Command.tex}: contiene tutti i comandi aggiuntivi creati dal gruppo;
\item \textbf{Impaginazione.tex}: definisce alcune istruzioni per l'impaginazione e si occupa di creare header e footer del documento;
\item \textbf{Packages.tex}: contiene tutti i pacchetti aggiuntivi e necessari per la compilazione.
\end{itemize}



\subsubsection{Struttura di un documento}
\paragraph{Prima pagina}
La prima pagina è composta da:
\begin{itemize}
  \item \textbf{Logo del gruppo};
  \item \textbf{Titolo del documento};
  \item \textbf{Informazioni varie del documento}:
      \begin{itemize}
      \item \textbf{Versione corrente};
      \item \textbf{Approvatori}: indica chi ha approvato il documento. Se non presente, indica che il documento non è ancora stato approvato;
      \item \textbf{Data approvazione};
      \item \textbf{Redattori}: indica chi si è occupato della stesura del documento;
      \item \textbf{Verificatori}: indica chi si è occupato della verifica del documento;
      \item \textbf{Uso}: indica se il documento è dedicato a uso interno o esterno;
      \item \textbf{Distribuzione}: indica a chi viene distribuito il documento;
        \end{itemize}
\item \textbf{Indirizzo e-mail del gruppo}.
\end{itemize}

\paragraph{Registro delle modifiche}
Ogni documento ha il suo registro modifiche che tiene traccia di tutte le modifiche importanti del documento durante il suo ciclo di vita.
Sotto forma di tabella, riporta:
  \begin{itemize}
  		\item Versione del documento dopo la modifica;
  		\item Data della modifica;
  		\item Nome dell’autore della modifica;
  		\item Ruolo dell’autore al momento della modifica;
  		\item Descrizione breve della modifica;
  		\item Nome della persona che si è occupata di verificare la modifica.
	\end{itemize}

\paragraph{Indice}
Presente dopo il registro delle modifiche, l’indice permette di avere una visione completa del documento e di individuare le varie parti, ogni voce è un collegamento ipertestuale alla parte del documento in cui viene trattata.

\paragraph{Struttura delle pagine}
Ogni pagina, a eccezione della prima, è formata da questi elementi:
  \begin{itemize}
  		\item In alto a sinistra si trova una miniatura a colori del logo del gruppo;
  		\item In alto a destra è presente il titolo del documento;
  		\item Sotto i due elementi appena elencati una linea nera continua li separa dal contenuto della pagina;
  		\item Il contenuto della pagina;
  		\item Sul lato destro del piè di pagina è indicato il numero della pagina corrente;
	\end{itemize}
	
\paragraph{Verbali}
I verbali applicano le stesse norme strutturali degli altri documenti con la differenza che non sono soggetti a versionamento. Ogni verbale sia interno che esterno dovrà contenere:
\begin{itemize}
\item Motivo della riunione;
\item Luogo della riunione;
\item Data della riunione;
\item Orario di inizio e fine riunione;
\item Partecipanti della riunione;
\item Resoconto della riunione;
\item Registro delle decisioni, dove si riporta in tabella le decisioni prese dal gruppo durante l’incontro;
\end{itemize}  		

\subsubsection{Normativa tipografica}
\paragraph{Nomi dei documenti}
La struttura generale del nome è la seguente: \\ \\
\centerline{\textbf{[NomeDocumento]-v[X].[Y].[Z]}}\\
in particolare:
\begin{itemize}
\item \textbf{[NomeDocumento]} inizia sempre con la lettera maiuscola. Se presenti più parole, queste saranno attaccate ma distinguibili dalla lettera maiuscola (convenzione \textit{"CamelCase"});
\item \textbf{v[X].[Y].[Z]} rappresenta la versione corrente del documento seguendo lo schema di versionamento presentato in TODO: metti i link;
\end{itemize}
I verbali, in quanto non soggetti a versionamento avranno una struttura del nome diversa, ovvero:\\ \\
\centerline{\textbf{Verbale[Tipologia]-[YYYY].[MM].[DD]}} \\
dove:
\begin{itemize}
\item \textbf{[Tipologia]} intende il tipo del verbale, può essere \textbf{Interno} o \textbf{Esterno};
\item \textbf{[YYYY].[MM].[DD]} indica la data in cui è avvenuto l'incontro, essa segue le specifiche degli elementi testuali definite in TODO: metti i link.
\end{itemize}


\paragraph{Stile di testo}
Gli stili di testi adottati nei documenti sono:
\begin{itemize}
\item \textbf{Grassetto}: per titoli, sottotitoli, e altri termini ritenuti importanti dal redattore;
\item \textbf{Maiuscolo}: per acronimi e iniziali di nomi propri, dei documenti o dei paragrafi;
\item \textbf{Corsivo}: per nomi propri dei membri del gruppo, committenti e proponente e per i nomi dei documenti.
\end{itemize}


\paragraph{Termini di glossario}
I termini che possono risultare ambigui e/o incongruenti sono contrassegnati con una \textsuperscript{G} alla loro prima occorrenza nella sezione d’interesse. Questi termini sono riportati con il loro significato in un documento esterno, il \textit{Glossario}.

\paragraph{Elementi testuali}
I redattori devono seguire le seguenti regole stilistiche:
\begin{itemize}
\item \textbf{Elenchi puntati}: un elenco puntato utilizzerà il simbolo • (pallino). Un successivo annidamento utilizzerà il simbolo - (trattino) e un altro ancora un asterisco (*). Se si tratta di un elenco numerato, i quattro livelli di enumerazione sono ordinati con i numeri arabi divisi da un punto fermo. Ogni voce dell’elenco inizia con una lettera maiuscola e termina con un punto e virgola, tranne l'ultima voce che termina con un punto;

\item \textbf{Formati di data}: Le date usano il formato \textbf{[YYYY].[MM].[DD]} dove:
    \begin{itemize}
    \item \textbf{[YYYY]} corrisponde all’anno;
    \item \textbf{[MM]} corrisponde al mese;
    \item \textbf{[DD]} corrisponde al giorno.
    \end{itemize}

\item \textbf{Orario}: gli orari usano il formato \textbf{[HH]:[MM]} dove:
    \begin{itemize}
    \item \textbf{[HH]} rappresentano le ore;
    \item \textbf{[MM]} rappresentano i minuti.
    \end{itemize}

\item \textbf{Sigle}: Tutte le sigle hanno le iniziali di ogni parola maiuscola tranne preposizioni, congiunzioni e articoli. Le sigle utilizzate sono:
    \begin{itemize}
    \item Relative ai documenti:
        \begin{itemize}
        \item \textbf{Analisi dei Requisiti}: AdR;
        \item \textbf{Piano di Progetto}: PdP;
        \item \textbf{Piano di Qualifica}: PdQ;
        \item \textbf{Glossario}: G;
        \item \textbf{Norme di Progetto}: NdP;
        \item \textbf{Verbali Interni}: VI;
        \item \textbf{Verbali Esterni}: VE.
        \end{itemize}
    
    \item Relative ai ruoli di progetto:
        \begin{itemize}
        \item \textbf{Responsabile di Progetto}: RE;
        \item \textbf{Amministratore}: AM;
        \item \textbf{Analista}: AN;
        \item \textbf{Progettista}: PT;
        \item \textbf{Programmatore}: PR;
        \item \textbf{Verificatore}: VE.
        \end{itemize}
    \end{itemize}
\end{itemize}

\paragraph{Elementi grafici}
Le regole per quanto riguarda l’uso di elementi grafici sono:
\begin{itemize}
\item \textbf{Immagini}: le figure presenti sono centrate rispetto al testo e accompagnate da didascalia;
\item \textbf{Diagrammi UML}: verranno inseriti nel documento tramite delle immagini
\end{itemize}

\paragraph{Metriche}
//Da definire

\paragraph{Strumenti}
Gli strumenti dedicati alla stesura sono:
\begin{itemize}
\item \textbf{\LaTeX}: linguaggio compilato basato sul programma di composizione tipografica Tex;
\item \textbf{Texmaker}: l’editor per la stesura dei documenti;
\item \textbf{Overleaf}: editor per la stesura dei documenti basato sul cloud;
\item \textbf{Draw.io}: utilizzato per la creazione di grafici UML. 
\end{itemize}

\subsection{Gestione della configurazione}
\subsubsection{Scopo}
Lo scopo è di gestire e controllare la produzione di documenti e codice in maniera sistematica. Per ogni oggetto sottoposto a configurazione viene garantito il versionamento e controllo sulle modifiche per permettere il mantenimento dell'integrità del prodotto. 

\subsubsection{Descrizione}
Vengono raggruppati e organizzati tutti i mezzi usati per la configurazione degli strumenti designati alla produzione di documenti e codice, per poter gestire struttura e la disposizione dei file all’interno di repository e anche  quelli per versionamento e coordinamento.

\subsubsection{Versionamento}
Per poter capire lo stato di avanzamento di un prodotto delle attività del progetto è necessario un identificatore. Il formato del codice di versione utilizzato è \\
\centerline{\textbf{v[X].[Y].[Z]}} \\ 
dove :
\begin{itemize}
\item \textbf{X} indica il rilascio pubblico e corrisponde ad una versione approvata dal Responsabile di Progetto. La numerazione parte da 0;
\item \textbf{Y} indica una revisione complessiva del prodotto per verificare che, dopo una modifica, il prodotto sia ancora coeso e consistente. La numerazione parte da 0 e si azzera ad ogni incremento di X;
\item \textbf{Z} viene incrementato ad ogni modifica con relativa verifica. La numerazione parte da 0 e si azzera ad ogni incremento di X o Y.
\end{itemize}

\paragraph{Strumenti}
Per il versionamento si è scelto di utilizzare un repository GitHub, che, a sua volta, implementa il software di controllo versione distribuito Git.

\subsubsection{Struttura del repository}
Il repository utilizzato dal gruppo per la creazione dei documenti contiene una directory per ogni documento denominata NomeDocumento, la cui struttura è approfondita in TODO: metti i link.
Il repository è suddiviso in più branch così definiti:
\begin{itemize}
\item \textbf{main}: il branch principale, che contiene l'ultima versione verificata di ogni documento;
\item \textbf{NomeDocumento}: uno per ogni documento, è dove il documento vive e viene attivamente stilato dai membri del gruppo.
\end{itemize}

\subsubsection{Modifiche al repository}
Non è consentito fare commit direttamente sul branch \textbf{main}, poiché porterebbe ad un elevato rischio di incongruenze e merge conflicts. Potrà essere modificato solo tramite il meccanismo di pull request con verifica obbligatoria, in modo da garantire che sia sempre presente una versione verificata e corretta del documento, anche se incompleta.
Nel branch \textbf{NomeDocumento} invece ogni membro può fare commit a patto che siano relative solo al documento specificato. Ogni commit deve referenziare la issue da cui è derivata e quindi, in generale, potranno effettuare modifiche sul branch solo gli assegnatari di issue che trattano quel documento specifico. Per quanto riguarda cambiamenti minimali (punteggiatura, errori ortografici, ecc.) è permessa la modifica autonoma da parte di qualsiasi membro del gruppo e non è necessario referenziare nessuna issue.

\subsection{Gestione della qualità}
\subsubsection{Scopo}
Lo scopo del processo di gestione della qualità è di assicurare che i requisiti di qualità individuati dagli stakeholder e le esigenze espresse dal proponente vengano rispettate dai prodotti e processi da sviluppare.

\subsubsection{Descrizione}
Il \textit{Piano di Qualifica} è il documento dedicato alla gestione della qualità. In esso sono descritti metriche e standard con le quali misurare e valutare la qualità di prodotti e processi.  

\subsubsection{Attività di processo}
Si possono individuare tre attività principali nel processo di gestione di qualità:
\begin{itemize}
\item \textbf{Pianificazione}: definire obiettivi di qualità, le strategie per raggiungerli e le risorse necessarie; 
\item \textbf{Valutazione}: applicare quanto pianificato, misurando i risultati;
\item \textbf{Reazione}: analizzando i risultati ottenuti con lo scopo di attuare miglioramenti o sanare situazioni non desiderate.
\end{itemize}

\subsubsection{Controllo di qualità}
Per essere sicuri di arrivare alla qualità desiderata, ogni membro deve essere in grado di:
\begin{itemize}
\item Comprendere gli obiettivi da raggiungere;
\item Individuare eventuali errori;
\item Stimare in termini di valore, dimensione e complessità le task;
\item Produrre risultati concreti e quantificabili.
\end{itemize}

\subsubsection{Denominazione metriche}
Per la denominazione delle metriche è stato adottato il formato: \\
\centerline{\textbf{M[Tipologia][Numero]}} \\
dove:
\begin{itemize}
\item \textbf{[Tipologia]}: indica la tipologia a cui si riferisce la metrica, può assumere tre valori:
    \begin{itemize}
    \item \textbf{PD}: relativa ai prodotti; 
    \item \textbf{PR}: relativa ai processi;
    \item \textbf{TS}: relativa ai test.
    \end{itemize}
\item \textbf{[Numero]}: indica il numero progressivo della metrica, parte da 1.
\end{itemize}


\subsection{Verifica}
\subsubsection{Scopo}
\subsubsection{Descrizione}
\subsubsection{Verifica della documentazione}
\subsubsection{Verifica del codice}
\subsubsection{Verifica dei requisiti}
\subsubsection{Test}

\subsection{Validazione}
\subsubsection{Scopo}
\subsubsection{Descrizione}
\subsubsection{Attività}

\subsection{Gestione dei cambiamenti}
\subsubsection{Scopo}
\subsubsection{Descrizione}
\subsubsection{Metodo}
