\section{Processi Primari}
\subsection{Fornitura}
\subsubsection{Scopo}
Lo scopo è scoprire quali strumenti e/o competenze siano necessarie nel progetto, documentare su come si organizza il lavoro e stabilire se il materiale prodotto sia di qualità.

\subsubsection{Descrizione}
In questo processo si scelgono procedure e risorse pensate per lo sviluppo del progetto. Si andrà inoltre a definire come gestire i rapporti con il proponente, comprese consegna e manutenzione prodotto finale.

\subsubsection{Documentazione fornita}
I documenti forniti all’azienda proponente e ai committenti sono :
\begin{itemize}
  \item \textbf{\SdF} contiene, per ogni capitolato, informazioni sul prodotto da sviluppare,le finalità, lati positivi e negativi. Inoltre contiene la scelta del capitolato del gruppo con le sue motivazioni;

  \item \textbf{\AdR} contiene l’analisi dei casi d’uso e dei requisiti;

  \item \textbf{\PdP} contiene la pianificazione preventiva dei tempi, l’analisi dei rischi, il consuntivo di periodo, la data di consegna e i costi previsti;

  \item \textbf{\PdQ} contiene le modalità adottate in verifica e validazione.

\end{itemize}
\subsubsection{Strumenti}
Strumenti utilizzati :
\begin{itemize}
  \item Texmaker per scrivere la documentazione.
\end{itemize}

\subsection{Sviluppo}
\subsubsection{Scopo}
Lo scopo è definire compiti e attività per arrivare al prodotto finale richiesto dal proponente.
\subsubsection{Descrizione}
Sono elencate e dopo trattate le seguenti attività di questo processo :
\begin{itemize}
  \item Analisi dei requisiti;
  \item Progettazione;
  \item Codifica.
\end{itemize}
\subsubsection{Analisi dei requisiti}
\subsubsection{Scopo}
E’ compito di ogni \textit{Analista} scrivere il documento di \textit{Analisi dei Requisiti}. Lo scopo di tale documento è :
\begin{itemize}
  \item aiutare i \textit{Progettisti};
  \item stabilire ciò che si è concordato con il cliente;
  \item fornire una base per chiunque prenda sottomano il prodotto per miglioramenti;
  \item aiutare le revisioni del codice;
  \item fornire riferimenti utili ai \textit{Verificatori};
  \item tracciare il lavoro per stimarne I costi.
\end{itemize}

\subsubsection{Descrizione}
L’obiettivo è la realizzazione dell’architettura del sistema.

\subsubsection{Struttura}
La struttura è destinata a cambiare.
Attualmente, \AdR presenta questa struttura :
\begin{itemize}
  \item Introduzione al documento;
  \item Descrizione generale, dove sono presenti requisiti estrapolati sia dal capitolato d’appalto che dagli incontri effettuati con il proponente (verbali esterni).
\end{itemize}

\subsubsection{Classificazione requisiti}
//TODO

\subsubsection{Classificazione casi d'uso}
La struttura adottata per la classificazione dei casi d'uso è la seguente: \\
\centerline{\textbf{UC[CodiceCasoBase](.[CodiceSottoCaso])*}}
Composta da:
\begin{itemize}
\item \textbf{UC} : acronimo di "use case";
\item \textbf{CodiceCasoBase} : id del caso d'uso generico;
\item \textbf{CodiceSottoCaso} : id opzionale per i sottocasi di un caso d'uso.
\end{itemize}

Ogni caso d'uso è descritto da : 
\begin{itemize}
    \item \textbf{Id} : codice identificativo del caso d'uso, stabilito come enunciato sopra;
    \item \textbf{Nome} : stringa titolo del caso d'uso posta dopo l'id;
    \item \textbf{Diagramma UML} : diagramma per rappresentare graficamente il caso d'uso;
    \item \textbf{Descrizione} : breve descrizione del caso d'uso;
    \item \textbf{Attori} : entità esterne al sistema che interagiscono con esso. Ne esistono due varianti : 
    \begin{itemize}
    \item \textbf{Primario} : interagisce con il sistema per raggiungere un obiettivo;
    \item \textbf{Secondario} : aiuta il primario a raggiungere l'obiettivo. Non utilizzato.
    \end{itemize}
    \item \textbf{Precondizione} : descrive lo stato del sistema prima del verificarsi del caso d'uso;
    \item \textbf{Postcondizione} : descrive lo stato del sistema dopo che si è verificato il caso d'uso;
    \item \textbf{Scenario principale} : elenco numerato che descrive il flusso degli eventi del caso d'uso;
    \item \textbf{Scenario secondario/alternativo} : elenco numerato che descrive il flusso degli eventi del caso d'uso dopo un evento imprevisto che lo ha deviato dal caso principale. Può non esserci o possono esserci più di uno;
    \item \textbf{Estensioni} : utilizzate nei scenari alternativi. Se si verifica una determinata situazione, il caso d'uso collegato all'estensione viene interrotto.
\end{itemize}



\subsubsection{Qualità dei requisiti}
Ciascun requisito deve essere :
\begin{itemize}
  \item Completo, ovvero dettagliato;
  \item Consistente, che non sia in contraddizione con altri requisiti;
  \item Necessario;
  \item Verificabile, ovvero che sia possibile controllare che il sistema lo realizzi.
  \item Tracciabile.
\end{itemize}

\subsection{Progettazione}
\subsubsection{Scopo}
Lo scopo è capire le caratteristiche che il prodotto deve avere per soddisfare i requisiti. La progettazione lo fa suddividendo il problema ai singoli componenti, ottimizzando tempi e risorse assegnate. Al tempo stesso garantendo la qualità del prodotto.

\subsubsection{Descrizione}
L’obiettivo è la realizzazione dell’architettura del sistema.

\subsubsection{Qualità}
E’ compito del \textit{Progettista} definire un’architettura di qualità. Le caratteristiche che essa deve avere sono :
\begin{itemize}
  \item Soddisfare i requisiti indicati nel documento \AdR;
  \item Essere comprensibile, robusto e affidabile;
  \item Presentare componenti semplici, in maniera tale da garantire modularità e riusabilità;
  \item Utilizzare le risorse in maniera efficiente.
\end{itemize}

\subsection{Codifica}
\subsubsection{Scopo}
Compito del \textit{Programmatore}, lo scopo è l’effettiva realizzazione del prodotto software. Si può vedere come la trasformazione in codice dell’architettura dei \textit{Progettisti}.

\subsubsection{Descrizione}
Il codice deve rispettare quanto scritto nella documentazione \PdQ. Qui saranno elencate regole e norme di carattere più generale, utilizzate da ogni linguaggio di programmazione impiegato nel progetto.

\subsubsection{Stile di codifica}
\subsubsection{Metriche}
\subsubsection{Strumenti}
