\section{Processi Primari}
\subsection{Fornitura}
\subsubsection{Scopo}
Lo scopo è scoprire quali strumenti e/o competenze siano necessarie nel progetto, documentare su come si organizza il lavoro e stabilire se il materiale prodotto sia di qualità.

\subsubsection{Descrizione}
\subsubsection{Documentazione fornita}
\subsubsection{Strumenti}

\subsection{Sviluppo}
\subsubsection{Scopo}
\subsubsection{Descrizione}
\subsubsection{Analisi dei requisiti}
\subsubsection{Scopo}
E’ compito di ogni \textit{Analista} scrivere il documento di \textit{Analisi dei Requisiti}. Lo scopo di tale documento è :
\begin{itemize}
  \item aiutare i \textit{Progettisti};
  \item stabilire ciò che si è concordato con il cliente;
  \item fornire una base per chiunque prenda sottomano il prodotto per miglioramenti;
  \item aiutare le revisioni del codice;
  \item fornire riferimenti utili ai \textit{Verificatori};
  \item tracciare il lavoro per stimarne I costi.
\end{itemize}

\subsubsection{Descrizione}
L’obiettivo è la realizzazione dell’architettura del sistema.

\subsubsection{Struttura}
La struttura è destinata a cambiare.
Attualmente, \AdR presenta questa struttura :
\begin{itemize}
  \item Introduzione al documento;
  \item Descrizione generale, dove sono presenti requisiti estrapolati sia dal capitolato d’appalto che dagli incontri effettuati con il proponente (verbali esterni).
\end{itemize}

\subsubsection{Classificazione requisiti}
È stato scelto di adottare la seguente rappresentazione per i requisiti:

\begin{center}
\textbf{R[Importanza][Tipologia][Codice]}
\end{center}
dove:
\begin{itemize}
\item \textbf{Importanza}: rappresenta l’importanza associata al requisito e può assumere uno dei seguenti valori:
\begin{itemize}
	\item \textbf{1}: Requisito \textit{Obbligatorio}, la sua soddisfazione dovrà necessariamente avvenire per garantire una buona funzionalità dell’intero sistema;
	\item \textbf{2}: Requisito \textit{Desiderabile}, la sua soddisfazione non vincola il buon funzionamento del sistema, tuttavia ne fornisce una maggior completezza;
	\item \textbf{3}: Requisito \textit{Facoltativo}, se soddisfatto rende il sistema più completo, ma ciò potrebbe comportare un dispendio di energie con un conseguente aumento dei costi preventivati.
\end{itemize}
\item \textbf{Tipologia}: si riferisce alla tipologia di requisito e può assumere uno dei seguenti valori letterali:
\begin{itemize}
	\item \textbf{V}: requisito di \textit{Vincolo}, descrive i vincoli offerti dal sistema;
	\item \textbf{F}: requisito \textit{Funzionale}, descrive servizi o funzioni offerti dal sistema;
	\item \textbf{P}: requisito \textit{Prestazionale}, descrive i vincoli sulle prestazioni da soddisfare, con il numero di informazioni da manipolare in un certo intervallo di tempo;
	\item \textbf{Q}: requisito di \textit{Qualità}, descrive i vincoli di qualità da realizzare (xxx).
\end{itemize}
\item \textbf{Codice}: identifica in maniera univoca il requisito in forma gerarchica padre/figlio.
Per esplicitare la forma gerarchica, il codice viene rappresentato come segue:
\begin{center}
\textbf{[CodiceBase](.[CodiceSottoCaso])}
\end{center}
dove: 
\begin{itemize}
	\item \textbf{CodiceBase}: fa riferimento al caso d’uso preso in esame e, in combinazione con la Tipologia, definisce un identificatore univoco per il requisito;
	\item \textbf{CodiceSottoCaso}: codice progressivo opzionale, che può includere più livelli, ed identifica un eventuale sottocaso.
\end{itemize}
\end{itemize}

Dopo aver classificato ciascun requisito con un codice, quest’ultimo non potrà più essere cambiato.
Inoltre, ciascun codice verrà accompagnato da una serie di informazioni aggiuntive, che meglio definiranno ciascun requisito, ossia:

\begin{itemize}
	\item \textbf{Descrizione}: breve descrizione completa relativa allo scopo del requisito;
	\item \textbf{Classificazione}: indica l’importanza del requisito e può assumere i valori \textit{Obbligatorio}, \textit{Desiderabile} e \textit{Facoltativo}. Sebbene questa informazione possa sembrare ridondante, ne facilita la lettura;
	\item \textbf{Fonti}: indica le fonti del requisito, ossia possono essere all'interno del Capitolato d’Appalto, nei Verbali Interni, nei Verbali Esterni e nei Casi d’Uso presenti nel documento “\textit{Analisi dei Requisiti vxx.xxx}”.
\end{itemize}

\definecolor{darkblue}{cmyk}{99, 99, 0, 71}

\begin{table}[!htbp]
\renewcommand{\arraystretch}{1.5}
\begin{tabular}{ m{0.20\textwidth}<{\centering}  m{0.25\textwidth}<{\centering}  m{0.25\textwidth}<{\centering}  m{0.2\textwidth}<{\centering}}
	\rowcolor{darkblue}
	\textcolor{white}{\textbf{Requisito}} &\textcolor{white}{\textbf{Descrizione}}& \textcolor{white}{\textbf{Classificazione}} & \textcolor{white}{\textbf{Fonti}}\\ 

	\rowcolor{gray!10} R1F1.1 & L’utente deve riuscire ad inserire i propri dati personali per effettuare la registrazione & Obbligatorio & UC1.1 \\	

\end{tabular}
\end{table}

\subsubsection{Classificazione casi d'uso}
La struttura adottata per la classificazione dei casi d'uso è la seguente: \\
\centerline{\textbf{UC[CodiceCasoBase](.[CodiceSottoCaso])*}}
Composta da:
\begin{itemize}
\item \textbf{UC} : acronimo di "use case";
\item \textbf{CodiceCasoBase} : id del caso d'uso generico;
\item \textbf{CodiceSottoCaso} : id opzionale per i sottocasi di un caso d'uso.
\end{itemize}

Ogni caso d'uso è descritto da : 
\begin{itemize}
    \item \textbf{Id} : codice identificativo del caso d'uso, stabilito come enunciato sopra;
    \item \textbf{Nome} : stringa titolo del caso d'uso posta dopo l'id;
    \item \textbf{Diagramma UML} : diagramma per rappresentare graficamente il caso d'uso;
    \item \textbf{Descrizione} : breve descrizione del caso d'uso;
    \item \textbf{Attori} : entità esterne al sistema che interagiscono con esso. Ne esistono due varianti : 
    \begin{itemize}
    \item \textbf{Primario} : interagisce con il sistema per raggiungere un obiettivo;
    \item \textbf{Secondario} : aiuta il primario a raggiungere l'obiettivo. Non utilizzato.
    \end{itemize}
    \item \textbf{Precondizione} : descrive lo stato del sistema prima del verificarsi del caso d'uso;
    \item \textbf{Postcondizione} : descrive lo stato del sistema dopo che si è verificato il caso d'uso;
    \item \textbf{Scenario principale} : elenco numerato che descrive il flusso degli eventi del caso d'uso;
    \item \textbf{Scenario secondario/alternativo} : elenco numerato che descrive il flusso degli eventi del caso d'uso dopo un evento imprevisto che lo ha deviato dal caso principale. Può non esserci o possono esserci più di uno;
    \item \textbf{Estensioni} : utilizzate nei scenari alternativi. Se si verifica una determinata situazione, il caso d'uso collegato all'estensione viene interrotto.
\end{itemize}

\subsubsection{Qualità dei requisiti}
Ciascun requisito deve essere :
\begin{itemize}
  \item Completo, ovvero dettagliato;
  \item Consistente, che non sia in contraddizione con altri requisiti;
  \item Necessario;
  \item Verificabile, ovvero che sia possibile controllare che il sistema lo realizzi.
  \item Tracciabile.
\end{itemize}
\subsection{Progettazione}
\paragraph{Scopo}
\paragraph{Descrizione}
\paragraph{Qualità}
\subsection{Codifica}
\paragraph{Scopo}
\paragraph{Descrizione}
\paragraph{Stile di codifica}
\paragraph{Metriche}
\paragraph{Strumenti}
