\section{Processi di Supporto}
\subsection{Documentazione}
\subsubsection{Scopo}
Ogni processo e attività per lo sviluppo del progetto devono essere documentate. Nella presente sezione verranno descritte regole e standard da seguire durante il processo di documentazione per l’intero ciclo di vita del software.

\subsubsection{Descrizione}
Vengono presentate decisioni e norme prescelte per :
\begin{itemize}
  \item stesura;
  \item verifica;
  \item approvazione.
\end{itemize}

\subsubsection{Ciclo di vita di un documento}
Ogni documento passa per I seguenti step :
\begin{itemize}
  \item Creazione : il documento viene creato basandosi su un template deciso dal gruppo;
  \item Strutturazione  : il documento viene fornito di :
  \begin{itemize}
  		\item registro delle modifiche;
  		\item indice dei contenuti.
	\end{itemize}
  \item Stesura : il gruppo redige il documento adottando il metodo incrementale;
  \item Revisione : ogni sezione del corpo del documento è rivista da almeno un membro del gruppo che non sia il redattore della parte in verifica;
  \item Approvazione : Se revisionato, il Responsabile di Progetto può stabilire che il documento è valido. Se approvato, può essere rilasciato.

\end{itemize}

\subsubsection{Struttura di un documento}
I documenti prodotti sono :
Ogni documento passa per I seguenti step :
\begin{itemize}
  \item Norme di progetto : documento interno che contiene norme e regole stabilite dal gruppo, che devono essere seguite per l’intera durata del progetto;
  \item Glossario : documento esterno dove sono presenti termini usati nella documentazione con le loro definizioni, se il gruppo lo ritiene necessario, affinché non ci siano ambiguità e/o incongruenze;
  \item Studio di fattibilità : documento interno con analisi,valutazione, lati negativi e positivi di ogni capitolato a disposizione;
  \item Piano di progetto : documento esterno con la pianificazione delle attività del progetto previste dal gruppo. Contiene la previsione dell’impegno orario dei singoli membri, il preventivo spese e I consuntivi di periodo.
  \item Piano di qualifica : documento esterno che espone e descrive i criteri con cui si valuta la qualità;
  \item Analisi dei requisiti : documento esterno che presenta requisiti e caratteristiche del prodotto finale;
  \item Verbali  : due varianti :
  \begin{itemize}
  		\item interni  : resoconti degli incontri del gruppo;
  		\item esterni  : resoconti degli incontri del gruppo con I committenti e/o il proponente.
	\end{itemize}

\end{itemize}
\paragraph{Prima pagina}
La prima pagina è composta da :
\begin{itemize}
  \item Logo del gruppo;
  \item Titolo del documento;
  \item Anno;
  \item Informazioni varie del documento :
  \begin{itemize}
  \item Versione corrente del documento;
  \item Stato approvazione e data approvazione;
  \item Approvatori : indica chi ha approvato il documento. Se non presente, indica che il documento non è ancora stato approvato;
  \item Redattori : indica chi si è occupato della stesura del documento;
  \item Verificatori : indica chi si è occupato della verifica del documento;
  \item Uso : indica se il documento è dedicato a uso interno o esterno;
  \item Distribuzione : indica a chi viene distribuito il documento;Distribuzione : indica a chi viene distribuito il documento;
\end{itemize}
\item Indirizzo e-mail del gruppo.
\end{itemize}

\paragraph{Registro delle modifiche}
Ogni documento ha il suo registro modifiche che tiene traccia di tutte le modifiche importanti del documento durante il suo ciclo di vita.
Sotto forma di tabella, riporta :
  \begin{itemize}
  		\item Versione del documento dopo modifica;
  		\item Data della modifica;
  		\item Nome dell’autore della modifica;
  		\item Ruolo dell’autore al momento della modifica;
  		\item Descrizione breve della modifica;
	\end{itemize}

\paragraph{Indice}
Presente dopo il registro delle modifiche, l’indice permette di avere una visione completa del documento e di orientarsi/individuare le varie parti.

\paragraph{Struttura delle pagine}
Ogni pagina, a eccezione della prima, è formata da questi elementi:
  \begin{itemize}
  		\item in alto a sinistra si trova una miniatura a colori del logo del gruppo;
  		\item in alto a destra è presente il titolo del documento;
  		\item sotto I due elementi appena elencati una linea nera continua li separa dal contenuto della pagina;
  		\item il contenuto della pagina;
  		\item sul lato destro del piè di pagina è indicato il numero della pagina corrente;
	\end{itemize}
	
\paragraph{Verbali}
I verbali applicano le stesse norme strutturali degli altri documenti ma con delle differenze :
  \begin{itemize}
  		\item Non sono evidenziati termini di glossario;
  		\item Sono presenti delle informazioni aggiuntive, tali :
  		\begin{itemize}
  		\item Motivo della riunione;
  		\item Luogo della riunione;
  		\item Data della riunione;
  		\item Orario di inizio e fine riunione;
  		\item Partecipanti della riunione;
  		\item Resoconto della riunione;
  		\item Registro delle decisioni, dove si riporta in tabella le decisioni prese dal gruppo durante l’incontro;
  		\end{itemize}  		
	\end{itemize}
	
\subsubsection{Norme}
\paragraph{Nomi dei file}
I nomi dei file iniziano con la lettera maiuscola. Se presenti più parole, queste saranno attaccate ma distinguibili dalla lettera maiuscola. Nel caso di file e cartelle legati alla struttura del documento, i nomi possono non contenere caratteri maiuscoli.

\paragraph{Stile di testo}
Gli stili di testi adottati nei documenti sono :
\begin{itemize}
\item grassetto : per titoli, sottotitoli e altri termini ritenuti importanti dal Redattore;
\item maiuscolo : acronimi e iniziali di nomi propri, nomi documenti e paragrafi;
\item corsivo : nomi documenti.
\end{itemize}


\paragraph{Termini di glossario}
I termini che possono risultare ambigui e/o incongruenti sono contrassegnati con una \textsuperscript{G} alla loro prima occorrenza nella sezione d’interesse. Questi termini sono riportati con il loro significato in un documento esterno, il \textit{Glossario}.

\paragraph{Elementi testuali}
I redattori devono seguire le seguenti regole stilistiche :
\begin{itemize}
\item Elenchi puntati : un elenco puntato utilizzerà il simbolo • (pallino). Un successivo annidamento utilizzerà (pallino vuoto) e un altro ancora un quadratino. Se si tratta di un elenco numerato, I quattro livelli di enumerazione sono ordinati con i numeri arabi divisi da un punto fermo. Ogni voce dell’elenco inizia con una lettera maiuscola e termina con un punto e virgola;
\item Formati di data : Le date usano il formato [DD]-[MM]-[YYYY] dove :
\begin{itemize}
\item [DD] corrisponde al giorno;
\item [MM] corrisponde al mese;
\item [YYYY] corrisponde all’anno.
\end{itemize}
\item Orario : gli orari usano il formato [HH]:[MM] dove :
\begin{itemize}
\item [HH] sono le ore;
\item [MM] sono I minuti.
\end{itemize}
\item Sigle : Tutte le sigle hanno le iniziali di ogni parola maiuscola tranne preposizioni, congiunzioni e articoli. Le sigle utilizzate sono :
\begin{itemize}
\item relative ai documenti :
\begin{itemize}
\item Analisi dei Requisiti : AdR;
\item Piano di Progetto : PdP;
\item Piano di Qualifica : PdQ;
\item Glossario : G;
\item Studio di Fattibilità : SdF;
\item Norme di Progetto NdP;
\item Verbali Interni : VI;
\item Verbali Esterni : VE.
\end{itemize}

\item relative alle revisioni di progetto previste dai committenti :
\begin{itemize}
\item Revisione dei Requisiti : RR;
\item Revisione di Progettazione : RP;
\item Revisione di Qualifica : RQ;
\item Revisione di Accettazione : RA.
\end{itemize}

\item relative ai ruoli di progetto :
\begin{itemize}
\item Responsabile di progetto : RE;
\item Amministratore : AM;
\item Analista : AN;
\item Progettista : PT;
\item Programmatore : PR;
\item Verificatore : VE.
\end{itemize}
\end{itemize}
\end{itemize}

\paragraph{Elementi grafici}
Le regole per quanto riguarda l’uso di elementi grafici sono :
\begin{itemize}
\item Immagini : le figure presenti sono centrate rispetto al testo e accompagnate da didascalia.
\end{itemize}

\paragraph{Metriche}
//Da definire

\paragraph{Strumenti}
Gli strumenti dedicati alla stesura sono :
\begin{itemize}
\item Latex : linguaggio compilato basato sul programma di composizione tipografica Tex;
\item Texmaker : l’editor scelto per la stesura dei documenti;
\end{itemize}

\subsection{Gestione della configurazione}
\subsubsection{Scopo}
Lo scopo è di gestire la produzione di documenti e codice in maniera ordinata e sistematica : per ogni oggetto modifica normata e versionamento.

\subsubsection{Descrizione}
Vengono raggruppati e organizzati gli strumenti a supporto degli strumenti per la produzione di documenti e codice, per poter gestire struttura e la disposizione dei file all’interno di repository e anche  quelli per versionamento e coordinamento.

\subsubsection{Versionamento}
Per poter capire lo stato di avanzamento di un prodotto delle attività del progetto è necessario un identificatore. Il codice di versione che abbiamo deciso di utilizzare è [X].[Y].[Z] dove :
\begin{itemize}
\item X indica il rilascio pubblico e corrisponde ad una versione approvata dal Responsabile di Progetto. La numerazione parte da 0;
\item Y indica una revisione complessiva del prodotto per verificare che, dopo una modifica, il prodotto sia ancora coeso e consistente. La numerazione parte da 0 e si azzera ad ogni incremento di X;
\item Z viene incrementato ad ogni modifica con relativa verifica. La numerazione parte da 0 e si azzera ad ogni incremento di X o Y.
\end{itemize}

\paragraph{Strumenti}
Per il versionamento si è scelto il sistema distribuito Git.

\subsubsection{Struttura del repository}
//Descrivere le repository
\subsubsection{Comandi base di GitHub}
//???
\subsubsection{Modifiche al repository}
//descrivere come avvengono le modifiche

\subsection{Gestione della qualità}
\subsubsection{Scopo}
Lo scopo è di stabilire una metrica per I servizi nell’ambito della verifica e della validazione, in maniera tale da rispettare I requisiti di qualità individuati dagli stakeholder e le esigenze espresse dal proponente.

\subsubsection{Descrizione}
Il Piano di Qualifica è il documento pensato per la gestione della qualità. In esso sono descritti metriche e modalità con le quali misurare e valutare la qualità di prodotti e processi.  

\subsubsection{Controllo di qualità}
Per essere sicuri di arrivare alla qualità desiderata, ogni membro deve essere in grado di:
\begin{itemize}
\item comprendere gli obiettivi da raggiungere;
\item individuare eventuali errori;
\item stimare in termini di valore, dimensione e complessità I task;
\item impiegare le competenze di ciascun membro del gruppo;
\item produrre risultati concreti e quantificabili;

\end{itemize}
\subsubsection{Tecniche}
//???


\subsection{Verifica}
\subsubsection{Scopo}
Lo scopo è definire come bisogna attuare il processo di verifica, per accertarsi che non ci siano errori e che il prodotto sia completo.

\subsubsection{Descrizione}
La verifica viene applicato ad ogni processo in esecuzione. In questo processo ci si affida all’analisi e ai test. L’analisi si divide in due tipologie :
\begin{itemize}
\item statica, che non richiede l’esecuzione dell’oggetto in verifica, perciò è applicabile a qualsiasi elemento;
\item dinamica, che richiede l’esecuzione dell’oggetto in verifica, applicabile perciò solo al codice.
\end{itemize}

\subsubsection{Verifica della documentazione}
Si utilizza un’analisi statica. Si possono utilizzare strumenti automatici oppure si può fare a mano attraverso due metodi :
\begin{itemize}
\item Walkthrough, che consiste in un controllo completo del documento;
\item Inspection, che consiste nel controllo di un punto mirato del documento.
\end{itemize}

\subsubsection{Verifica del codice}
Si utilizza sia l’analisi statica che quella dinamica. In particolare :
\begin{itemize}
\item con la statica si controlla la bontà del codice, sia dal punto di vista della correttezza che dell’ordine;
\item con la dinamica si controlla la presenza o meno di bug durante l’esecuzione del prodotto.
\end{itemize}

\subsubsection{Verifica dei requisiti}
Vengono applicate Walkthrough e Inspection per controllare validità e coerenza con quanto dichiarato e descritto all’interno dell’Analisi dei Requisiti.

\subsubsection{Test}
I test sono l’attività fondamentale dell’analisi dinamica. Servono sia per dimostrare che il programma funzioni e svolga ciò per cui è stato sviluppato. I test si dividono in quattro categorie, in base all’oggetto in verifica e allo scopo : 
\begin{itemize}
\item Test d’unità : I test sono l’attività fondamentale dell’analisi dinamica. Servono sia per dimostrare che il programma funzioni e svolga ciò per cui è stato sviluppato. I test si dividono in quattro categorie, in base all’oggetto in verifica e allo scopo.
\item Test d’integrazione : Verificano la correttezza delle interfacce. Si vuole stabilire il corretto funzionamento delle varie componenti, dopo che hanno passato il test d’unità, aggregandole man mano e verificando il funzionamento nel complesso.
\item Test di sistema : Verifica l’applicazione nella sua interezza. Venendo dopo il test d’integrazione, lo scopo è verificare che I componenti non solo sono compatibili ma che lo scambio di dati tra interfacce e le varie interazioni siano conformi. Inoltre così si controlla che i requisiti siano stati soddisfatti.
\item Test di regressione : Verifica l’applicazione dopo modifiche al sistema. Lo scopo è controllare nuove funzionalità non testate e al tempo stesso garantire che le funzioni precedentemente implementate e testate siano ancora di qualità e non subiscano alterazioni ai loro comportamenti.
\end{itemize}

\subsection{Validazione}
\subsubsection{Scopo}
Lo scopo è stabilire se il prodotto è in grado di soddisfare ciò per il quale è stato creato. Dopo la Validazione, è certo che il software soddisfi e sia conforme ai requisiti del proponente.

\subsubsection{Descrizione}
Questo processo avviene dopo il processo di verifica, prende il prodotto appena verificato e lo restituisce garantendo che sia conforme ai requisiti del proponente.

\subsubsection{Attività}
Il responsabile di Progetto controlla i risultati ottenuti e può :
\begin{itemize}
\item accettare e approvare il prodotto;
\item rifiutare e chiedere un’ulteriore verifica con delle nuove indicazioni;
\end{itemize}


//Da rimuovere?
\subsection{Gestione dei cambiamenti}
\subsubsection{Scopo}
\subsubsection{Descrizione}
\subsubsection{Metodo}
