\section{Processi Primari}
\subsection{Fornitura}
\subsubsection{Scopo}
Lo scopo è scoprire quali strumenti e/o competenze siano necessarie nel progetto, documentare su come si organizza il lavoro e stabilire se il materiale prodotto sia di qualità.

\subsubsection{Descrizione}
In questo processo si scelgono procedure e risorse pensate per lo sviluppo del progetto. Si andrà inoltre a definire come gestire i rapporti con il proponente, comprese consegna e manutenzione prodotto finale.

\subsubsection{Documentazione fornita}
I documenti forniti all’azienda proponente e ai committenti sono :
\begin{itemize}
  \item \textbf{\SdF} contiene, per ogni capitolato, informazioni sul prodotto da sviluppare,le finalità, lati positivi e negativi. Inoltre contiene la scelta del capitolato del gruppo con le sue motivazioni;

  \item \textbf{\AdR} contiene l’analisi dei casi d’uso e dei requisiti;

  \item \textbf{\PdP} contiene la pianificazione preventiva dei tempi, l’analisi dei rischi, il consuntivo di periodo, la data di consegna e i costi previsti;

  \item \textbf{\PdQ} contiene le modalità adottate in verifica e validazione.

\end{itemize}
\subsubsection{Strumenti}
Strumenti utilizzati :
\begin{itemize}
  \item \textbf{Texmaker} per scrivere la documentazione.
\end{itemize}

\subsection{Sviluppo}
\subsubsection{Scopo}
Lo scopo è definire compiti e attività per arrivare al prodotto finale richiesto dal proponente.
\subsubsection{Descrizione}
Sono elencate e dopo trattate le seguenti attività di questo processo :
\begin{itemize}
  \item \textbf{Analisi dei requisiti};
  \item \textbf{Progettazione};
  \item \textbf{Codifica}.
\end{itemize}
\subsubsection{Analisi dei requisiti}
\subsubsection{Scopo}
E’ compito di ogni \textit{Analista} scrivere il documento di Analisi dei Requisiti. Lo scopo di tale documento è :
\begin{itemize}
  \item aiutare i \textit{Progettisti};
  \item stabilire ciò che si è concordato con il cliente;
  \item fornire una base per chiunque prenda sottomano il prodotto per miglioramenti;
  \item aiutare le revisioni del codice;
  \item fornire riferimenti utili ai \textit{Verificatori};
  \item tracciare il lavoro per stimarne i costi.
\end{itemize}

\subsubsection{Descrizione}
L’obiettivo è la realizzazione dell’architettura del sistema.

\subsubsection{Struttura}
La struttura è destinata a cambiare.
Attualmente, \AdR presenta questa struttura :
\begin{itemize}
  \item \textbf{Introduzione} al documento;
  \item \textbf{Descrizione generale}, dove sono presenti requisiti estrapolati sia dal capitolato d’appalto che dagli incontri effettuati con il proponente (verbali esterni).
\end{itemize}

\subsubsection{Classificazione requisiti}

\subsubsection{Classificazione casi d'uso}

\subsubsection{Qualità dei requisiti}
Ciascun requisito deve essere :
\begin{itemize}
  \item \textbf{Completo}, ovvero dettagliato;
  \item \textbf{Consistente}, che non sia in contraddizione con altri requisiti;
  \item \textbf{Necessario};
  \item \textbf{Verificabile}, ovvero che sia possibile controllare che il sistema lo realizzi;
\end{itemize}

\subsection{Progettazione}
\subsubsection{Scopo}
Lo scopo è capire le caratteristiche che il prodotto deve avere per soddisfare i requisiti : la progettazione lo fa suddividendo il problema ai singoli componenti, ottimizzando tempi e risorse assegnate e al tempo stesso garantendo la qualità del prodotto.

\subsubsection{Descrizione}
L’obiettivo è la realizzazione dell’architettura del sistema.

\subsubsection{Qualità}
E’ compito del \textit{Progettista} definire un’architettura di qualità. Le caratteristiche che essa deve avere sono :
\begin{itemize}
  \item Soddisfare i requisiti indicati nel documento \AdR;
  \item Essere comprensibile, robusto e affidabile;
  \item Presentare componenti semplici, in maniera tale da garantire modularità e riusabilità;
  \item Utilizzare le risorse in maniera efficiente.
\end{itemize}

\subsection{Codifica}
\subsubsection{Scopo}
Compito del \textit{Programmatore}, lo scopo è l’effettiva realizzazione del prodotto software. Si può vedere come la trasformazione in codice dell’architettura dei Progettisti.

\subsubsection{Descrizione}
Il codice deve rispettare quanto scritto nella documentazione \PdQ. Qui saranno elencate regole e norme di carattere più generale, utilizzate da ogni linguaggio di programmazione impiegato nel progetto.

\subsubsection{Stile di codifica}
\subsubsection{Metriche}
\subsubsection{Strumenti}
