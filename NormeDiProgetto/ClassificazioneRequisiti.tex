% Classificazione Requisiti - Norme di progetto
\documentclass[a4paper]{article}
\usepackage[italian]{babel}
\usepackage[T1]{fontenc}
\usepackage[utf8]{inputenc}
\usepackage{graphicx}
\usepackage[margin=1in]{geometry}
\usepackage{makecell}
%\usepackage[svgnames,table]{xcolor}
\usepackage[table]{xcolor}
\begin{document}

\subsection{Classificazione Requisiti}

È stato scelto di adottare la seguente rappresentazione per i requisiti:

\begin{center}
\textbf{R[Importanza][Tipologia][Codice]}
\end{center}
dove:
\begin{itemize}
\item \textbf{Importanza}: rappresenta l’importanza associata al requisito e può assumere uno dei seguenti valori:
\begin{itemize}
	\item \textbf{1}: Requisito \textit{Obbligatorio}, la sua soddisfazione dovrà necessariamente avvenire per garantire una buona funzionalità dell’intero sistema;
	\item \textbf{2}: Requisito \textit{Desiderabile}, la sua soddisfazione non vincola il buon funzionamento del sistema, tuttavia ne fornisce una maggior completezza;
	\item \textbf{3}: Requisito \textit{Facoltativo}, se soddisfatto rende il sistema più completo, ma ciò potrebbe comportare un dispendio di energie con un conseguente aumento dei costi preventivati.
\end{itemize}
\item \textbf{Tipologia}: si riferisce alla tipologia di requisito e può assumere uno dei seguenti valori letterali:
\begin{itemize}
	\item \textbf{V}: requisito di \textit{Vincolo}, descrive i vincoli offerti dal sistema;
	\item \textbf{F}: requisito \textit{Funzionale}, descrive servizi o funzioni offerti dal sistema;
	\item \textbf{P}: requisito \textit{Prestazionale}, descrive i vincoli sulle prestazioni da soddisfare, con il numero di informazioni da manipolare in un certo intervallo di tempo;
	\item \textbf{Q}: requisito di \textit{Qualità}, descrive i vincoli di qualità da realizzare (xxx).
\end{itemize}
\item \textbf{Codice}: identifica in maniera univoca il requisito in forma gerarchica padre/figlio.
Per esplicitare la forma gerarchica, il codice viene rappresentato come segue:
\begin{center}
\textbf{[CodiceBase](.[CodiceSottoCaso])}
\end{center}
dove: 
\begin{itemize}
	\item \textbf{CodiceBase}: fa riferimento al caso d’uso preso in esame e, in combinazione con la Tipologia, definisce un identificatore univoco per il requisito;
	\item \textbf{CodiceSottoCaso}: codice progressivo opzionale, che può includere più livelli, ed identifica un eventuale sottocaso.
\end{itemize}
\end{itemize}

Dopo aver classificato ciascun requisito con un codice, quest’ultimo non potrà più essere cambiato.
Inoltre, ciascun codice verrà accompagnato da una serie di informazioni aggiuntive, che meglio definiranno ciascun requisito, ossia:

\begin{itemize}
	\item \textbf{Descrizione}: breve descrizione completa relativa allo scopo del requisito;
	\item \textbf{Classificazione}: indica l’importanza del requisito e può assumere i valori \textit{Obbligatorio}, \textit{Desiderabile} e \textit{Facoltativo}. Sebbene questa informazione possa sembrare ridondante, ne facilita la lettura;
	\item \textbf{Fonti}: indica le fonti del requisito, ossia possono essere all'interno del Capitolato d’Appalto, nei Verbali Interni, nei Verbali Esterni e nei Casi d’Uso presenti nel documento “\textit{Analisi dei Requisiti vxx.xxx}”.
\end{itemize}

\definecolor{darkblue}{cmyk}{99, 99, 0, 71}

\begin{table}[!htbp]
\renewcommand{\arraystretch}{1.5}
\begin{tabular}{ m{0.20\textwidth}<{\centering}  m{0.25\textwidth}<{\centering}  m{0.25\textwidth}<{\centering}  m{0.2\textwidth}<{\centering}}
	\rowcolor{darkblue}
	\textcolor{white}{\textbf{Requisito}} &\textcolor{white}{\textbf{Descrizione}}& \textcolor{white}{\textbf{Classificazione}} & \textcolor{white}{\textbf{Fonti}}\\ 
	%1.0.0& XX.12.21& \shortstack{ \\ XX} &\shortstack{ \\ \RE{} } & Approvazione per il rilascio\\

	\rowcolor{gray!10} R1F1 & L’utente deve riuscire ad inserire i propri dati personali per effettuare la registrazione & Obbligatorio & UC1.1 \\	

\end{tabular}
\end{table}

% Da mettere sotto la linea di fine pagina
%(xxx) riferimento: Per una descrizione dettagliata sui vincoli di qualità presi in considerazione si veda il documento \textit{Piano di Qualifica vxx.xxx} in § : Obiettivi e metriche di qualità.

\end{document}