\section{Introduzione}

\subsection{Scopo del Documento}
Lo scopo di questo documento è di fornire una documentazione dettagliata delle API realizzate per il progetto \textit{Sweeat}, specificando per ciascuna API i suoi endpoint, le informazioni necessarie ad effettuare una richiesta e e la struttura dei messaggi di risposta.


\subsection{Scopo del Prodotto}
L’obiettivo di Sweeat e dell’azienda \zd è la creazione di un sistema software costituito da una Webapp. Lo scopo del prodotto è di fornire all’utente una guida dei locali gastronomici sfruttando i numerosi contenuti digitali creati dagli utenti sulle principali piattaforme social (Instagram e TikTok).
In questo modo, è possibile realizzare una classifica basata sulle impressioni e reazioni di chiunque usufruisca dei servizi dei locali, non solo da professionisti ed esperti del settore.


\subsection{Glossario}
Per evitare ambiguità relative alle terminologie utilizzate è stato creato un documento denominato “\textit{Glossario}”. Questo documento comprende tutti i termini tecnici scelti dai membri del gruppo e utilizzati nei vari documenti con le relative definizioni. Tutti i termini inclusi in questo glossario, vengono segnalati all'interno del documento con l'apice \textsuperscript{G} accanto alla parola.