\section{Informazioni Generali}

\begin{itemize}
\item{\textbf{Luogo}}: Meeting Google Meet
\item{\textbf{Data}}: \D
\item{\textbf{Ora}}: 18:00 - 18:30
\item{\textbf{Partecipanti dell'azienda Zero12}}: 
	\begin{itemize}
	\item{Michele Massaro.} 
	\end{itemize} 
\item{\textbf{Partecipanti del Gruppo}}: 
	\begin{itemize}
	\item{\EP{},} 
	\item{\FP{},}
	\item{\GC{},}
	\item{\LW{},}	
	\item{\MB{},}
	\item{\PV{}.}
	\end{itemize} 
\item{\textbf{Segretario}}: \GC{}
\end{itemize}

\section{Motivo della riunione}
\begin{itemize}
	\item {Allineamento progetto parte Front-end,}
	\item {Allineamento progetto parte Back-end.}
\end{itemize}

\section{Resoconto}

Lo scopo di questa riunione è stato, principalmente, quello di mostrare all'azienda lo stato di avanzamento della WebApp, illustrando nel dettaglio le varie funzionalità che abbiamo realizzato in vista della prossima consegna (\textit{Product Baseline}). 

\subsection{Front-end}
Per quanto riguarda la parte front-end, ci siamo allineati con \textit{Michele Massaro} di \textit{Zero 12} e gli abbiamo mostrato la nuova veste grafica della WebApp e le diverse funzionalità che sono state implementate:

\begin{itemize}
\item \textbf{Ricerca},
\item \textbf{Registrazione},
\item \textbf{Login},
\item \textbf{Recupero Password},
\item \textbf{Modifica password},
\item \textbf{Calcolo punteggio totale per ciascun locale}.
\end{itemize}

Inoltre, abbiamo analizzato nel dettaglio tutte le pagine che sviluppate in questa fase di progetto: 

\begin{itemize}
\item \textbf{Home}, con ricerca e migliori locali presenti a sistema;
\item \textbf{Risultati}, ossia la contenente i risultati ottenuti dalla ricerca e le informazioni che vengono mostrate per ciascun locale;
\item \textbf{Pagina di dettaglio}, contenente i post Instagram relativi ad un locale ed ottenuti mediante il crawler;
\item \textbf{Login}, con il form da compilare per accedere al sistema;
\item \textbf{Registrazione}, con il form da compilare per registrarsi al sistema;
\item \textbf{Recupero Password}, con il form da compilare nel caso l'utente generico abbia smarrito la password e non riesca più ad accedere al sistema;
\item \textbf{Area Personale}, con le diverse sezioni: Lista dei Preferiti, Suggerisci Account e Dati Personali (con possibilità di cambio password).
\end{itemize}

Infine, abbiamo chiarito alcuni dubbi in merito alla struttura con cui avevamo implementato la parte del MVVM; l'azienda ci ha dato delle conferme e dei pareri, che ci hanno permesso di migliorare ulteriormente quanto svolto.

\subsection{Back-end}


\pagebreak
