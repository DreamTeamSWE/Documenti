\section{Informazioni Generali}

\begin{itemize}
\item{\textbf{Luogo}}: Meeting Google Meet
\item{\textbf{Data}}: \D{}
\item{\textbf{Ora}}: 17:00 - 17:30
\item{\textbf{Partecipanti dell'azienda Zero12}}: 
	\begin{itemize}
	\item{Michele Massaro.} 
	\end{itemize} 
\item{\textbf{Partecipanti del Gruppo}}: 
	\begin{itemize}
	\item{\EP{},} 
	\item{\FP{},}
	\item{\GC{},}
	\item{\LW{},}	
	\item{\MB{},}
	\item{\PV{}.}
	\end{itemize} 
\item{\textbf{Segretario}}: \GC{}
\end{itemize}

\section{Motivo della riunione}
\begin{itemize}
	\item{Punto della situazione;}
	\item {Presentazione parte back-end e discussione;}
	\item {Presentazione parte front-end e discussione;}
	\item {Discussione su dettagli e fine del progetto.}
\end{itemize}

\section{Resoconto}
\subsection{Back-end}
Sono stati mostrati i database, sia quelli con dati prelevati direttamente da Instagram tramite il crawler che quelli elaborati tramite AWS. Sono stati mostrati anche alcuni degli algoritmi utilizzati per ottenere tali dati.
Si è discusso del problema più grosso del back-end, ovvero il costante ban degli account Instagram. Gli ultimi sono stati bannati anche per ip e perciò si è parlato delle probabili soluzioni a questo problema, come il delay tra una richiesta e l'altra,
oppure l'utilizzo di ancora più account per allegerire il peso del singolo e evitare di attirare l'attenzione per colpa di troppe richieste al server. 


\subsection{Front-end}
E' stato mostrato lo scheletro del sito : 
\begin{itemize}
	\item Home : esempio della pagina iniziale dove è presente la barra ricerca e appariranno i vari locali con le informazioni più generali;
	\item Area Personale : esempio della pagina dove si possono vedere i propri dati personali e proporre un nuovo account da seguire;
	\item Classifica : esempio della pagina dopo una ricerca effettuata, con i vari filtri per variare i risultati.
	Abbiamo inoltre parlato di come intendiamo applicare i filtri e di come questi siano pensati per influire sul ranking dei locali andando a cambiare il peso dato a testo, immagini e emoji.
\end{itemize}


\subsection{Ulteriori dettagli e scopo del progetto}
Si è discusso molto della possibilità di raggiungimento di certi requisiti, viste le difficoltà avute nelle attività di crawling per via del ban continuo. Nonostante ciò,
Michele ha voluto far presente che la nostra ricerca in materia, i nostri tentativi e i documenti su cui tutto questo verrà riportato sono un punto molto importante del progetto, essendo lo scopo 
dell'azienda Zero12 vedere cosa fosse possibile fare con tali tecnologie.




\pagebreak
