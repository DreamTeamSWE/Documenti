\section{Informazioni Generali}

\begin{itemize}
\item{\textbf{Luogo}}: Meeting Google Meet
\item{\textbf{Data}}: \D{}
\item{\textbf{Ora}}: 17:00 - 17:30
\item{\textbf{Partecipanti dell'azienda Zero12}}: 
	\begin{itemize}
	\item{Michele Massaro.} 
	\end{itemize} 
\item{\textbf{Partecipanti del Gruppo}}: 
	\begin{itemize}
	\item{\EP{},} 
	\item{\FP{},}
	\item{\GC{},}
	\item{\LW{},}	
	\item{\MB{},}
	\item{\PV{}.}
	\end{itemize} 
\item{\textbf{Segretario}}: \GC{}
\end{itemize}

\section{Motivo della riunione}
\begin{itemize}
	\item {Punto della situazione;}
	\item {Presentazione parte back-end e discussione;}
	\item {Presentazione parte front-end e discussione;}
	\item {Discussione su dettagli e scopo del progetto.}
\end{itemize}

\section{Resoconto}
\subsection{Back-end}
È stato mostrato il database su cui si appoggia il progetto, il quale contiene sia i dati “grezzi” ottenuti dal crawler (p.es. nome del profilo che ha pubblicato i contenuti analizzati) sia i dati elaborati tramite le tecnologie offerte da AWS e sfruttate per il nostro progetto (Rekognition, Comprehend etc.).  \\

Abbiamo mostrato al proponente anche il funzionamento degli algoritmi realizzati in Python per la parte Back-end della WebApp, ossia le nostre “Lambda”, come quelli inserire i contenuti estrapolati da Instagram ed analizzati con le tecnologie di AWS nel db. \\ 

Si è discusso del problema più grosso del back-end, ovvero il costante \textbf{ban} degli account Instagram. Gli ultimi account Instagram creati sono stati bannati anche tramite indirizzo IP e, perciò, si è discusso delle possibili soluzioni a questo problema, come il delay tra una richiesta e l'altra, oppure la creazione e l'utilizzo di altri account IG per allegerire il peso del singolo ed evitare di attirare l'attenzione per colpa di troppe richieste al server. 

\subsection{Front-end}
È stato mostrato lo scheletro della WebApp: 
\begin{itemize}
	\item {\textbf{Home}: esempio della pagina iniziale dove è presente la barra di ricerca, la barra di navigazione e la sezione dove compariranno locali d'esempio con le informazioni più generali;}
	\item {\textbf{Area Personale}: esempio della pagina dove è possibile consultare i propri dati personali inseriti in fase di registrazione e proporre un nuovo profilo Instagram, da cui la piattaforma potrà effettuare il crawling dei dati;}
	\item {\textbf{Classifica}: una pagina d'esempio con i risultati ottenuti una volta effettuata una ricerca, nella quale sono presenti anche i vari filtri per modificare i contenuti e l'ordine dei risultati.}
\end{itemize}
Inoltre, abbiamo discusso di come intendiamo applicare i filtri sui risultati della classifica e di come alcuni di essi modificheranno l'ordinamento della classifica stessa, poiché tramite questi ultimi sarà possibile dare più peso ad un tipo (testo, immagini o emoji) di contenuto piuttosto che un altro.

\subsection{Ulteriori dettagli e scopo del progetto}
Viste le difficoltà riscontrate sino ad ora, primo fra tutti l'attività di crawling dopo i numerosi ban, si è discusso molto sulla soddisfacibilità dei requisiti definiti. \\
Nonostante ciò, \textit{Michele Massaro di Zero 12} ha voluto far presente che la nostra ricerca, i nostri tentativi ed i documenti prodotti in merito al progetto, hanno molto valore, in quanto, l'azienda stessa, voleva capire cosa eravamo in grado di produrre con le tecnologie proposte ed i \textbf{limiti} del progetto stesso.

\pagebreak
