\section{Informazioni Generali}

\begin{itemize}
\item{\textbf{Luogo}}: Meeting Google Meet
\item{\textbf{Data}}: \D
\item{\textbf{Ora}}: 18:00 - 18:30
\item{\textbf{Partecipanti dell'azienda Zero12}}: 
	\begin{itemize}
	\item{Michele Massaro.} 
	\end{itemize} 
\item{\textbf{Partecipanti del Gruppo}}: 
	\begin{itemize}
	\item{\EP{},} 
	\item{\FP{},}
	\item{\GC{},}
	\item{\LW{},}	
	\item{\MB{},}
	\item{\PV{}.}
	\end{itemize} 
\item{\textbf{Segretario}}: \PV{}
\end{itemize}

\section{Motivo della riunione}
\begin{itemize}
	\item {Allineamento progetto;}
	\item {Dubbi ultime implementazioni;}
 	\item {Test;}
  	\item {Domande su presentazione finale.}
\end{itemize}

\section{Resoconto}

\subsection{Dubbi ultime implementazioni}
Siccome non sappiamo bene quanto tempo possono prenderci le ultime implementazioni, abbiamo chiesto all'azienda su cosa concentrarci. Per quanto riguarda il sistema dei preferiti, se non riusciamo a terminarlo in tempo, la risposta è stata che basta togliere la grafica relativa ai preferiti. Tutt'altra cosa invece per i filtri, dove hanno suggerito di tentare un workaround se non riusciamo a implementarli, essendo più o meno importanti per questo tipo di applicazione.

\subsection{Test}
E' stato chiesto se fossero necessari dei test particolari ma è stato risposto che non sono richiesti.

\subsection{Presentazione finale}
Visto l'arrivo della presentazione finale, si è chiesto quali fossero le cose importanti da mostrare durante la suddetta. L'indicazione che ci è stata data è di mostrare il flusso, quindi il movimento dell'utente all'interno dell'applicazioni, e il funzionamento con qualche schemino semplice che mostri i collegamenti. Non serve mostrare codice. 

\pagebreak
