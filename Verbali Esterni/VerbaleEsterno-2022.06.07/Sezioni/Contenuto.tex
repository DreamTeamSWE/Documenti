\section{Informazioni Generali}

\begin{itemize}
\item{\textbf{Luogo}}: Meeting Google Meet
\item{\textbf{Data}}: \D
\item{\textbf{Ora}}: 18:00 - 18:30
\item{\textbf{Partecipanti dell'azienda Zero12}}: 
	\begin{itemize}
	\item{Michele Massaro.} 
	\end{itemize} 
\item{\textbf{Partecipanti del Gruppo}}: 
	\begin{itemize}
	\item{\EP{},} 
	\item{\FP{},}
	\item{\GC{},}
	\item{\LW{},}	
	\item{\MB{},}
	\item{\PV{}.}
	\end{itemize} 
\item{\textbf{Segretario}}: \PV{}
\end{itemize}

\section{Motivo della riunione}
\begin{itemize}
	\item {Allineamento progetto;}
	\item {Dubbi ultime implementazioni;}
 	\item {Test;}
  	\item {Domande su presentazione finale.}
\end{itemize}

\section{Resoconto}

\subsection{Dubbi ultime implementazioni}
Non abbiamo idea di quanto tempo sarà necessario per realizzare le ultime implementazioni, perciò abbiamo chiesto a \textit{Michele Massaro di Zero12} su cosa concentrarci. Per quanto riguarda il sistema dei preferiti, poiché sappiamo se riusciremo a terminare la funzionalità per aggiungere e rimuovere un locale sulla/nella lista dei preferiti per tempo. Michele ci ha detto che è preferibile concentrarsi su altre funzionalità di maggior rilievo, come i filtri, e tenere i preferiti come secondari. Per quanto riguarda i filtri, ci hanno suggerito un workaround per l'implementazione, essendo una parte importante per questa tipologia di applicazione.

\subsection{Test}
È stato chiesto se fossero necessari dei test particolari ma è stato risposto che non sono richiesti.

\subsection{Presentazione finale}
In vista della presentazione finale, abbiamo chiesto quali fossero le cose importanti da mostrare durante la suddetta. L'indicazione che ci è stata data è di mostrare il flusso del crawler, oltre al funzionamento con qualche schema semplice che mostri il valore della WebApp. Non serve mostrare codice. 

\pagebreak
