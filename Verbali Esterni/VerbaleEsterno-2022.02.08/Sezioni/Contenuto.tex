\section{Informazioni Generali}

\begin{itemize}
\item{\textbf{Luogo}}: Meeting Google Meet
\item{\textbf{Data}}: \D{}
\item{\textbf{Ora}}: 14:00 - 14:30
\item{\textbf{Partecipanti dell'azienda Zero12}}: 
	\begin{itemize}
	\item{Michele Massaro,} 
	\item{Matteo Depascale.}
	\end{itemize} 
\item{\textbf{Partecipanti del Gruppo}}: 
	\begin{itemize}
	\item{\EP{},} 
	\item{\FP{},}
	\item{\GC{},}
	\item{\LW{},}	
	\item{\MB{},}
	\item{\PV{}.}
	\end{itemize} 
\item{\textbf{Segretario}}: \GC{}
\end{itemize}

\section{Motivo della riunione}
\begin{itemize}
\item{Configurazione Amazon Aurora Serverless}
\end{itemize}

\section{Resoconto}

\subsection{Configurazione Amazon Aurora Serverless}

Questa riunione esterna è stata fatta per cercare di configurare al meglio un database con \textbf{Amazon Aurora Serverless} ed evitare di sprecare risorse inutilmente, che altrimenti avremmo dovuto pagare con i crediti offerti dall'azienda. In particolare, \textit{Matteo Depascale} di Zero12 ci ha spiegato brevemente il funzionamento di Aurora Serverless, oltre ad indicarci passo passo come configurare in maniera efficiente il nostro database dalla console di AWS. Abbiamo così creato il nostro primo Database Serverless con MySQL che sfrutteremo per il PoC (attualmente stavamo usando un semplice Database Relazionale con piano gratuito, in quanto non avevamo ancora a disposizione i crediti di AWS). \\

Come ribadito anche da Michele Massaro di Zero12, Amazon Aurora Serverless è una configurazione a dimensionamento automatico on demand per Amazon Aurora. Si avvia, si arresta e ridimensiona automaticamente la capacità in base alle esigenze dell'applicazione; quindi consente di eseguire il proprio database nel cloud senza doverne gestire la capacità. Inoltre, si integra perfettamente con tutto l'ecosistema AWS, che stiamo usando per tutto il nostro progetto.

\pagebreak
