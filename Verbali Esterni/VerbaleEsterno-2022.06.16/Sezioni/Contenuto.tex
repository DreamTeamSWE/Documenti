\section{Informazioni Generali}

\begin{itemize}
\item{\textbf{Luogo}}: Meeting Google Meet
\item{\textbf{Data}}: \D
\item{\textbf{Ora}}: 18:00 - 18:25
\item{\textbf{Partecipanti dell'azienda Zero12}}: 
	\begin{itemize}
	\item{Michele Massaro.} 
	\end{itemize} 
\item{\textbf{Partecipanti del Gruppo}}: 
	\begin{itemize}
	\item{\EP{},} 
	\item{\FP{},}
	\item{\GC{},}
	\item{\LW{},}	
	\item{\MB{},}
	\item{\MG{}.}
	\end{itemize} 
\item{\textbf{Segretario}}: \GC
\end{itemize}

\section{Motivo della riunione}
\begin{itemize}
	\item {Dimostrazione della WebApp;}
	\item {Test di accettazione.}
\end{itemize}

\section{Resoconto}

Questa è stata l'ultima riunione esterna prima dell'ultima revisione. Per questo motivo, abbiamo mostrato la versione finale della WebApp e verificato che quanto richiesto inizialmente fosse stato implementato. 

\subsection{Dimostrazione della WebApp}

Questa è stata l'ultima riunione esterna prima del collaudo. Per questo motivo, abbiamo mostrato la versione finale della WebApp e verificato che quanto richiesto inizialmente fosse stato implementato. Nella prima parte della riunione abbiamo mostrato interamente la WebApp con tutte le relative funzionalità. In particolare, ci siamo soffermati su: 

\begin{itemize}
    \item Filtri nella pagina della classifica,
    \item Gestione dei preferiti.
\end{itemize}

Abbiamo mostrato all'azienda il loro funzionamento: prima selezionando la località e/o il raggio, il tipo di cucina poi, dopo aver cliccato su “Applica filtri”, verrà renderizzata la classifica filtrata in base alle esigenze dell'utente. \\
Analogamente, abbiamo mostrato anche la gestione dei preferiti: è possibile aggiungere un locale dalla lista dei preferiti direttamente dalla pagina della ricerca o dalla classifica (tramite il cuore, nel caso non sia colorato), mentre è possibile rimuovere il locale dai preferiti (tramite il cestino) sia dall'area personale che dalle card presente nella pagina di ricerca o dalla classifica (tramite il cuore, nel caso sia rosso).

\subsection{Test di Accettazione}

Oltre a ciò, abbiamo verificato che i test di accettazione concordati fossero tutti soddisfatti.

Infine, abbiamo verificato i requisiti richiesti analizzando nella parte di Frontend: registrazione, login, ricerca, classifica, filtri, gestione dei preferiti, recupero password, pagina di dettaglio. Per la parte di BackEnd, abbiamo appurato la correttezza del crawler: l'abbiamo avviato ed abbiamo mostrato il corretto funzionamento, andando ad analizzare i contenuti estrapolati da un profilo aggiunto al sistema in tempo reale. Michele, in rappresentanza dell'azienda ci ha comunicato che è stato soddisfatto ciò che richiedeva il capitolato. 

\pagebreak
