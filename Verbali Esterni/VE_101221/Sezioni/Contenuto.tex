\section{Informazioni Generali}

\begin{itemize}
\item{\textbf{Luogo}}: Meeting Google Meet
\item{\textbf{Data}}: 10 Dicembre 2021
\item{\textbf{Ora}}: 14 - 17
\item{\textbf{Partecipanti del Gruppo}}: 
	\begin{itemize}
	\item{\EP{},} 
	\item{\FP{},}
	\item{\MB{},}
	\item{\MG{},}
	\item{\PV{}}
	\end{itemize} 
\item{\textbf{Segretario}}: \PV{}	
\end{itemize}

\section{Ordine del Giorno}
\begin{itemize}
\item{}
\item{}
\item{}
\item{}
\end{itemize}

\section{Resoconto}

\subsection{Punto della situazione sul lavoro svolto dai sotto-gruppi}

Dopo aver analizzato il punto della situazione e l'andamento del lavoro svolto dai sotto-gruppi, è stato verificato il glossario e si è discusso degli standard di riferimento e delle relative metriche. In particolare, è stato deciso di utilizzare lo standard ISO/IEC 9126 per la parte relativa alla Qualità del Prodotto e lo standard ISO/IEC 15504 - detto anche “SPICE”, per la parte relativa alla Qualità del Processo.

\subsection{Proposte migliorative}

Analizzando il lavoro svolto da tutti i componenti del gruppo, è emerso che la figura del responsabile era un po' carente ed i compiti di ciascun componente del gruppo non erano ben definiti. Per questi motivi, abbiamo deciso di cambiare gli incarichi di ciascun sotto-gruppo, eleggendo come responsabile della settimana a \FP{}. Inoltre, avendo già realizzato della documentazione, abbiamo deciso chi potesse verificare quanto redatto. 

\subsection{Incontro di formazione con l'Azienda}

Il gruppo ha scelto una delle date proposte dall'azienda per il primo incontro di formazione, che verterà sulla parte architetturale di AWS e si svolgerà tramite un meeting di Google Meet.

\subsection{Gestione Coordinamento del Gruppo}

È stata modificata la struttura interna dell'account di GitHub (in particolare, è stato creato un nuovo account con il nome di: “\textit{DreamTeamSWE}”) ed è stato creato un repository ufficiale. \\
Si è discusso del sistema di ticketing e project management mediante le issue di GitHub, per una miglior gestione del lavoro.   